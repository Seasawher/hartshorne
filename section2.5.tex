\bfsection{2.5 加群の層}

\bfsubsection{定義-$\calo_X$加群の逆像}
\barquo{
正確には、任意の$\calo_X$加群$\calf$および任意の$\calo_Y$加群$\scrg$に対して自然な群の同型
\[
\Hom_{\calo_X}(f^*\scrg, \scrf ) \cong \Hom_{\calo_Y}(\scrg, f_*\scrf )
\]
が存在する。
}
\begin{proof}
  詳しくはBosch\cite{Bosch} section 6.9 Proposition 3 を参照のこと。
\end{proof}


\bfsubsection{命題 5.1}
\barquo{
(e) 任意の$A$加群$M$に対して$f^*(\wt{M}) \cong (M \ts_A B) \ \wt{}$
}
\begin{proof}
  詳しくはBosch\cite{Bosch} Section 6.9 Cor.4を参照のこと。
\end{proof}


\bfsubsection{定義-連接層}
\barquo{
$(X,\calo_X)$をスキームとする。$\calo_X$加群$\calf$が準連接であるとは、$X$が開アファイン部分集合$U_i = \Spec A_i$で覆うことができ、各$i$に対して$A_i$加群$M$が存在し、$\calf|_{U_i} \cong \wt{M_i}$となるときをいう。さらに$M_i$が有限生成$A_i$加群ととれるとき$\calf$を連接という。
}
\begin{rem}
  連接層の定義に疑問。$X$が局所Noehterでないときにこれは正しいのか?
\end{rem}


\bfsubsection{命題5.8 (c)}
\barquo{
$X$がNoetherであるか、$f$が準コンパクトかつ分離的であると仮定する。このとき$\scrf$が$\calo_X$加群の準連接層とすると、$f_*\scrf$は$\calo_Y$加群の準連接層となる。
}
\begin{rem}
  $X$がNoetherなら、任意のスキーム$Y$と射$f \colon X \to Y$について$f$は準コンパクトかつquasi-separatedである。これはG\"ortz Wedhorn\cite{GW} Cor 10.24 と Remark 10.2から従うことである。さらに同書の Cor 10.27 より、$f$が準コンパクトかつquasi-separatedであるならば$f_*$は準連接性を保つことがわかる。つまり$X$がNoetherという仮定や準コンパクトかつ分離的という仮定は緩めることができる。
\end{rem}


\bfsubsection{命題 5.8}
\barquo{
(a) 問題は$X$および$Y$双方について局所的なものなので、$X$および$Y$を双方アファインとして良い。 (中略) (c) 問題は$Y$についてのみ局所的なものなので、$Y$をアファインと仮定して良い。
}
\begin{proof}
局所的という言葉の意味が詳しく書かれていないようなのでBosch\cite{Bosch} section 6.9の記述に沿って補足を行う。$f \colon X \to Y$がスキームの射、$\calg$が$\calo_Y$加群だったとする。このとき任意の$U \opsub X$と$f(U) \subset V \opsub Y$に対して
\[
(f^* \calg)|_U \cong f^*_U(\calg|_V)
\]
が成り立つ。ただし$f_U \colon U \to V$は$f$を制限して得られるものである。この事を以て、$f^*\scrg$は$X$についても$Y$についても局所的だといっているのである。

一方で$\calf$が$\calo_X$加群とするとき、任意の$V \opsub Y$について
\[
(f_* \calf)|_V \cong (f_V)_* (\calf_{f^{-1}(V)})
\]
でしかない。ただし$f_V \colon f^{-1}(V) \to V$は$f$の制限である。この事を以て、$f_*  \scrf$は$Y$についてしか局所的でないとしたのである。

命題5.8の証明についてなお不明点がある場合はLiu\cite{Liu} section 5.1 Proposition 1.14も参照のこと。
\end{proof}




\bfsubsection{定義 Serreのねじり層}
\barquo{
任意の$n \in \Z$に対し、層$\calo_X(n)$を$S(n) \ \wt{}$と定義する。$\calo_X(1)$をSerreのねじり層と呼ぶ。
}
\begin{rem}
  $S(n)$とは何だろうか。前後の記述にあたったが、どうも定義されていないように見える。一般に、次数付き環$S$の上の次数付き加群$M = \oplus_{n \in \Z} M_n$に対して、$M(m)$とは
  \[
  M(m)_d = M_{m+d}
  \]
  で定義されるような次数つき$S$加群のことである。たとえば Bosch\cite{Bosch} section 9.2 に詳しく説明があるので参照のこと。
\end{rem}
