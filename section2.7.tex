\bfsection{2.7 射影的射}

\bfsubsection{定理7.1 直前}
\barquo{
さて、$X$を$A$上の任意のスキームとし、$\vp \colon X \to \P^n_A$を$X$から$\P^n_A$への$A$射とする。このとき$\scrl = \vp^*(\calo(1))$は$X$上の可逆層で、大域切断$s_0, \cdots , s_n$ (ここで$s_i = \vp^*(x_i)$, $s_i \in \grG(X,\scrl)$) は層$\scrl$を生成する。
}
\begin{rem} ${}$
  \begin{description}
    \item[$\vp^*(x_i)$とは何だろうか?]  Daniel Murfet氏のPDF\cite{RisingSea}に従って補足する。$f \colon X \to Y$がスキーム間の射で$\scrf$が$\calo_Y$-加群であるとする。このとき随伴であることからcanonicalな射$\eta \colon \scrf \to f_* f^* \scrf$がある。これにより、$V \opsub Y$に対して環の射$\eta_V \colon \grG(V, \scrf) \to \grG(f^{-1}(V), f^*\scrf)$が誘導される。
    そこで$s \in  \grG(V, \scrf)$に対して$f^*(s) = \eta_V(s)$と定めるのである。
    \item[大域切断$s_0, \cdots , s_n$が層$\scrl$を生成するのは何故?]
    茎を考えるとわかる。
  \end{description}

\end{rem}
