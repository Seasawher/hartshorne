\documentclass[10pt]{jsarticle}%文字サイズが10ptのjsarticle %エンジンはpLaTeX

%%%%%%%%%%%%%%%%%%%%%%%%%%%%%%%%%%%%%%%%%%%%%%%%%%%%%%%
%%  パッケージ                                        %%
%%%%%%%%%%%%%%%%%%%%%%%%%%%%%%%%%%%%%%%%%%%%%%%%%%%%%%%
%使用しないときはコメントアウトしてください
\usepackage{amsthm}%定理環境
\usepackage{framed}%文章を箱で囲う
\usepackage{amsmath,amssymb}%数式全般
\usepackage[dvipdfmx]{graphicx}%図の挿入
%\usepackage{tikz}%描画
\usepackage{titlesec}%見出しの見た目を編集できる
\usepackage[dvipdfmx, usenames]{color}%色をつける
%\usepackage{tikz-cd}%可換図式
%\usepackage{mathtools}%数式関連
\usepackage{amsfonts}%数式のフォント
\usepackage[all]{xy}%可換図式
\usepackage{mathrsfs}%花文字
\usepackage{comment}%コメント環境
\usepackage{picture}%お絵かき
\usepackage{url}%URLを出力
\usepackage[dvipdfmx]{hyperref}
\usepackage{pxjahyper}%日本語しおりの文字化けを防ぐ

%%%%%%%%%%%%%%%%%%%%%%%%%%%%%%%%%%%%%%%%%%%%%%%%%%%%%%%
%%  表紙                                             %%
%%%%%%%%%%%%%%%%%%%%%%%%%%%%%%%%%%%%%%%%%%%%%%%%%%%%%%%
\makeatletter
\def\thickhrulefill{\leavevmode \leaders \hrule height 1pt\hfill \kern \z@}
\renewcommand{\maketitle}{\begin{titlepage}%
    \let\footnotesize\small
    \let\footnoterule\relax
    \parindent \z@
    \reset@font
    \null\vfil
    \begin{flushleft}
      \huge \@title
    \end{flushleft}
    \par
    \hrule height 4pt
    \par
    \begin{flushright}
      \LARGE \@author \par
    \end{flushright}
    \vskip 60\p@
    \vfil\null
    \begin{flushright}
        {\small \@date}%
    \end{flushright}
  \end{titlepage}%
  \setcounter{footnote}{0}%
}
\makeatother

%%%%%%%%%%%%%%%%%%%%%%%%%%%%%%%%%%%%%%%%%%%%%%%%%%%%%%%
%%  sectionの修飾                                     %%
%%%%%%%%%%%%%%%%%%%%%%%%%%%%%%%%%%%%%%%%%%%%%%%%%%%%%%%
\titleformat{\section}[block]
{}{}{0pt}
{
  \colorbox{black}{\begin{picture}(0,10)\end{picture}}
  \hspace{0pt}
  \normalfont \Large\bfseries
  \hspace{-4pt}
}
[
\begin{picture}(100,0)
  \put(3,18){\color{black}\line(1,0){300}}
\end{picture}
\\
\vspace{-30pt}
]

%%%%%%%%%%%%%%%%%%%%%%%%%%%%%%%%%%%%%%%%%%%%%%%%%%%%%%%
%%  太字section                                     %%
%%%%%%%%%%%%%%%%%%%%%%%%%%%%%%%%%%%%%%%%%%%%%%%%%%%%%%%
\newcommand{\bfsubsection}[1]{\subsection*{\textbf{#1}}}
\newcommand{\bfsection}[1]{\section*{\textbf{#1}}}

%%%%%%%%%%%%%%%%%%%%%%%%%%%%%%%%%%%%%%%%%%%%%%%%%%%%%%%
%%  番号付き定理環境                                  %%
%%%%%%%%%%%%%%%%%%%%%%%%%%%%%%%%%%%%%%%%%%%%%%%%%%%%%%%
%注:defというコマンドはもうある
\theoremstyle{definition}%定理環境のアルファベットを斜体にしない
\renewcommand{\proofname}{\textgt{証明}}%proof環境の修正

%%%%%%%%%%%%%%%%%%%%%%%%%%%%%%%%%%%%%%%%%%%%%%%%%%%%%%%
%%  番号なし定理環境                                  %%
%%%%%%%%%%%%%%%%%%%%%%%%%%%%%%%%%%%%%%%%%%%%%%%%%%%%%%%
%一部箱付き
\newtheorem*{lemma}{補題}
\newtheorem*{proposition}{命題}
\newtheorem*{definition}{定義}
\newcommand{\lem}[1]{\begin{oframed} \begin{lemma} #1 \end{lemma} \end{oframed}}%箱付きほだい
\newcommand{\prop}[1]{\begin{oframed} \begin{proposition} #1 \end{proposition} \end{oframed}}%箱付きめいだい

\newtheorem*{claim}{主張}
\newtheorem*{sol}{解答}
\newtheorem*{prob}{問題}
\newtheorem*{quo}{引用}
\newtheorem*{rem}{注意}



%%%%%%%%%%%%%%%%%%%%%%%%%%%%%%%%%%%%%%%%%%%%%%%%%%%%%%%
%%  左側に線を引く                                  %%
%%%%%%%%%%%%%%%%%%%%%%%%%%%%%%%%%%%%%%%%%%%%%%%%%%%%%%%
%leftbar環境の定義
\makeatletter
\renewenvironment{leftbar}{%
%  \def\FrameCommand{\vrule width 3pt \hspace{10pt}}%  デフォルトの線の太さは3pt
  \renewcommand\FrameCommand{\vrule width 1pt \hspace{10pt}}%
  \MakeFramed {\advance\hsize-\width \FrameRestore}}%
 {\endMakeFramed}
\newcommand{\exbf}[2]{ \begin{leftbar} \textbf{#1} #2 \end{leftbar} }%左線つき太字
\newcommand{\barquo}[1]{\begin{leftbar} \begin{quo} #1 \end{quo} \end{leftbar}}%左線つき引用



%%%%%%%%%%%%%%%%%%%%%%%%%%%%%%%%%%%%%%%%%%%%%%%%%%%%%%%
%%  色をつける                                      %%
%%%%%%%%%%%%%%%%%%%%%%%%%%%%%%%%%%%%%%%%%%%%%%%%%%%%%%%
\newcommand{\textblue}[1]{\textcolor{blue}{\textbf{#1}}}

%%%%%%%%%%%%%%%%%%%%%%%%%%%%%%%%%%%%%%%%%%%%%%%%%%%%%%%
%%  よく使う記号の略記                                 %%
%%%%%%%%%%%%%%%%%%%%%%%%%%%%%%%%%%%%%%%%%%%%%%%%%%%%%%%
%一般
\newcommand{\f}[2]{\frac{#1}{#2}}%分数
\newcommand{\kakko}[1]{\langle #1 \rangle}%鋭角かっこ%\angleはもうある

%集合
\newcommand{\setmid}[2]{\left\{ #1 \mathrel{} \middle| \mathrel{} #2 \right\}}%集合の内包記法
\newcommand{\sm}{\setminus}%集合差
\newcommand{\single}{\{ 0 \}}%0のシングルトン
\newcommand{\tm}{\times}%直積

%解析
\newcommand{\abs}[1]{\left \lvert #1 \right \rvert}%絶対値
\newcommand{\norm}[1]{\left \lVert #1 \right \rVert}%ノルム
\newcommand{\transpose}[1]{\, {\vphantom{#1}}^t\!{#1}}%行列の転置
\newcommand{\I}{\sqrt{-1}}%虚数単位。\iは既にある。
\newcommand{\del}{\partial}%偏微分の記号

%位相
\newcommand{\clsub}{\subset_{\text{closed}}}%閉部分集合
\newcommand{\opsub}{\subset_{\text{open}}}%開部分集合
\newcommand{\loc}{\subset_{\text{loc. closed}}}%局所閉部分集合

%装飾
\newcommand{\wt}[1]{\widetilde{#1}}%わいどちるだあ
\newcommand{\ol}[1]{\overline{#1}}%オーバーライン
\newcommand{\wh}[1]{\widehat{#1}}%ワイドハット
\newcommand{\la}{\overleftarrow}%上付き左矢印
\newcommand{\ra}{\overrightarrow}%上付き右矢印
\newcommand{\sh}{\sharp}%シャープ
\newcommand{\fl}{\flat}%ふらっと

%論理
\newcommand{\To}{\Rightarrow}%ならば%自然変換
%\newcommand{\xto}[1]{\xrightarrow}%上側文字付き右向き矢印
\newcommand{\st}{\; \; \text{s.t.} \; \;}%空白付きsuch that

%代数
\newcommand{\ts}{\otimes}%テンソル積

%その他
\newcommand{\vartm}{\times^{\text{Var}}}%多様体の圏における直積。集合の直積と区別するとき用。

%%%%%%%%%%%%%%%%%%%%%%%%%%%%%%%%%%%%%%%%%%%%%%%%%%%%%%%
%%       演算子                                       %%
%%%%%%%%%%%%%%%%%%%%%%%%%%%%%%%%%%%%%%%%%%%%%%%%%%%%%%%
%log型
\DeclareMathOperator{\rank}{rank}%行列の階数
\DeclareMathOperator{\re}{Re}%実部
\DeclareMathOperator{\gal}{Gal}%Galois群
\DeclareMathOperator{\Hom}{Hom}%射の集合
\DeclareMathOperator{\tr}{Trace}%トレース
\DeclareMathOperator{\Aut}{Aut}%自己同型群
\DeclareMathOperator{\trdeg}{tr\text{.}deg}%超越次数
\DeclareMathOperator{\Frac}{Frac}%商体をとる操作
\renewcommand{\Im}{\operatorname{Im}}%写像の像。Abel圏の像対象。虚部が出力できなくなった。
\DeclareMathOperator{\im}{im}%写像の像
\DeclareMathOperator{\Ker}{Ker}%写像の核。Abel圏の核対象。
\DeclareMathOperator{\coker}{coker}%余核%対象のほう
\DeclareMathOperator{\Coker}{Coker}%余核%射のほう
\DeclareMathOperator{\Spec}{Spec}%スペクトル
\DeclareMathOperator{\Sing}{Sing}%Singular point.特異点の集合。歌ってるわけではないぞ
\DeclareMathOperator{\Supp}{Supp}%台
\DeclareMathOperator{\ann}{ann}%アナイアレーター
\DeclareMathOperator{\Ass}{Ass}%素因子
\DeclareMathOperator{\ord}{ord}%おーだー
\DeclareMathOperator{\height}{ht}%素イデアルの高度。\htはもうある
\DeclareMathOperator{\coht}{coht}%素イデアルの余高度
\DeclareMathOperator{\Lan}{Lan}%左Kan拡張
\DeclareMathOperator{\Ran}{Ran}%右Kan拡張

%limit型
\DeclareMathOperator*{\llim}{\varprojlim}%極限。逆極限。射影極限。
\DeclareMathOperator*{\rlim}{\varinjlim}%余極限。順極限。入射極限。

%%%%%%%%%%%%%%%%%%%%%%%%%%%%%%%%%%%%%%%%%%%%%%%%%%%%%%%
%%  黒板太字(blackboard bold)                         %%
%%%%%%%%%%%%%%%%%%%%%%%%%%%%%%%%%%%%%%%%%%%%%%%%%%%%%%%
\newcommand{\bba}{{\mathbb A}}
\newcommand{\bbb}{{\mathbb B}}
\newcommand{\bbc}{{\mathbb C}}
\newcommand{\bbd}{{\mathbb D}}
\newcommand{\bbe}{{\mathbb E}}
\newcommand{\bbf}{{\mathbb F}}
\newcommand{\bbg}{{\mathbb G}}
\newcommand{\bbh}{{\mathbb H}}
\newcommand{\bbi}{{\mathbb I}}
\newcommand{\bbj}{{\mathbb J}}
\newcommand{\bbk}{{\mathbb K}}
\newcommand{\bbl}{{\mathbb L}}
\newcommand{\bbm}{{\mathbb M}}
\newcommand{\bbn}{{\mathbb N}}
\newcommand{\bbo}{{\mathbb O}}
\newcommand{\bbp}{{\mathbb P}}
\newcommand{\bbq}{{\mathbb Q}}
\newcommand{\bbr}{{\mathbb R}}
\newcommand{\bbs}{{\mathbb S}}
\newcommand{\bbt}{{\mathbb T}}
\newcommand{\bbu}{{\mathbb U}}
\newcommand{\bbv}{{\mathbb V}}
\newcommand{\bbw}{{\mathbb W}}
\newcommand{\bbx}{{\mathbb X}}
\newcommand{\bby}{{\mathbb Y}}
\newcommand{\bbz}{{\mathbb Z}}

%%%%%%%%%%%%%%%%%%%%%%%%%%%%%%%%%%%%%%%%%%%%%%%%%%%%%%%
%%  よく使う黒板太字                                  %%
%%%%%%%%%%%%%%%%%%%%%%%%%%%%%%%%%%%%%%%%%%%%%%%%%%%%%%%
\newcommand{\Z}{\bbz}
\newcommand{\A}{\bba}
\newcommand{\Q}{\bbq}
\newcommand{\R}{\bbr}
\newcommand{\C}{\bbc}
\newcommand{\F}{\bbf}
\newcommand{\N}{\bbn}
\renewcommand{\P}{\bbp}%パラグラフ記号が出力できなくなった

%%%%%%%%%%%%%%%%%%%%%%%%%%%%%%%%%%%%%%%%%%%%%%%%%%%%%%%
%%  カリグラフィー                                %%
%%%%%%%%%%%%%%%%%%%%%%%%%%%%%%%%%%%%%%%%%%%%%%%%%%%%%%%
%大文字しかどうせ使わない
\newcommand{\cala}{\mathcal{A}}
\newcommand{\calb}{\mathcal{B}}
\newcommand{\calc}{\mathcal{C}}
\newcommand{\cald}{\mathcal{D}}
\newcommand{\calf}{\mathcal{F}}
\newcommand{\calo}{\mathcal{O}}

%%%%%%%%%%%%%%%%%%%%%%%%%%%%%%%%%%%%%%%%%%%%%%%%%%%%%%%
%%  ギリシャ文字(Greek letters)小文字                 %%
%%%%%%%%%%%%%%%%%%%%%%%%%%%%%%%%%%%%%%%%%%%%%%%%%%%%%%%
%コマンドが5字以上のもの
\newcommand{\gra}{{\alpha}}
\newcommand{\grg}{{\gamma}}
\newcommand{\grd}{{\delta}}
\newcommand{\gre}{{\epsilon}}
\newcommand{\grt}{{\theta}}
\newcommand{\grk}{{\kappa}}
\newcommand{\grl}{{\lambda}}
\newcommand{\grs}{{\sigma}}
\newcommand{\gru}{{\upsilon}}
\newcommand{\gro}{{\omega}}

\newcommand{\ve}{{\varepsilon}}
\newcommand{\vp}{{\varphi}}

%%%%%%%%%%%%%%%%%%%%%%%%%%%%%%%%%%%%%%%%%%%%%%%%%%%%%%%
%%  ギリシャ文字(Greek letters)大文字                 %%
%%%%%%%%%%%%%%%%%%%%%%%%%%%%%%%%%%%%%%%%%%%%%%%%%%%%%%%
%コマンドが5字以上のもの
\newcommand{\grG}{{\Gamma}}
\newcommand{\grD}{{\Delta}}
\newcommand{\grT}{{\Theta}}
\newcommand{\grL}{{\Lambda}}
\newcommand{\grS}{{\Sigma}}
\newcommand{\grU}{{\Upsilon}}
\newcommand{\grO}{{\Omega}}

%%%%%%%%%%%%%%%%%%%%%%%%%%%%%%%%%%%%%%%%%%%%%%%%%%%%%%%
%%  フラクトゥール                                  %%
%%%%%%%%%%%%%%%%%%%%%%%%%%%%%%%%%%%%%%%%%%%%%%%%%%%%%%%
\newcommand{\fraka}{\mathfrak{a}}
\newcommand{\frakb}{\mathfrak{b}}
\newcommand{\frakm}{\mathfrak{m}}
\newcommand{\frakn}{\mathfrak{n}}
\newcommand{\frakp}{\mathfrak{p}}
\newcommand{\frakq}{\mathfrak{q}}

\newcommand{\frakA}{\mathfrak{A}}
\newcommand{\frakB}{\mathfrak{B}}
\newcommand{\frakT}{\mathfrak{T}}

\newcommand{\Top}{\mathfrak{Top}}%開部分集合全体のなす有向集合
\newcommand{\Ab}{\mathfrak{Ab}}%Abel群のなす圏

%%%%%%%%%%%%%%%%%%%%%%%%%%%%%%%%%%%%%%%%%%%%%%%%%%%%%%%
%%  花文字                                          %%
%%%%%%%%%%%%%%%%%%%%%%%%%%%%%%%%%%%%%%%%%%%%%%%%%%%%%%%
%大文字しかどうせ使わない
\newcommand{\scra}{\mathscr{A}}
\newcommand{\scrf}{\mathscr{F}}
\newcommand{\scrg}{\mathscr{G}}
\newcommand{\scrh}{\mathscr{H}}
\newcommand{\scrs}{\mathscr{S}}

%%%%%%%%%%%%%%%%%%%%%%%%%%%%%%%%%%%%%%%%%%%%%%%%%%%%%%%
%%  太字                                            %%
%%%%%%%%%%%%%%%%%%%%%%%%%%%%%%%%%%%%%%%%%%%%%%%%%%%%%%%
\newcommand{\Sh}{\textbf{Sh}}%層の圏
\newcommand{\PSh}{\textbf{PSh}}%前層の圏

\newcommand{\bfc}{\textbf{C}}
\newcommand{\bfd}{\textbf{D}}
\newcommand{\bfe}{\textbf{E}}


\begin{document}

\title{R.ハーツホーン『代数幾何学』\\高橋宣能 松下大介 訳}
\author{\url{https://seasawher.github.io/kitamado/} \\ @seasawher}
\date{\today}
\maketitle


\bfsection{1.1 アファイン多様体}
\bfsubsection{例1.1.1}
\barquo{
特にこの例はHausdorffでない。
}
\begin{proof}
  演習1.7(d)と例1.4.7によりわかる。既約な集合の空でない二つの開集合は、つねに空でない共通部分を持つという事実(Goerts Wedhorn\cite{GW}Prop 1.15.)を用いてもよい。
\end{proof}


\bfsubsection{定義 例1.1.1直後}
\barquo{
位相空間$X$の空でない部分集合$Y$が既約であるとは、$Y$において閉であるような二つの真部分集合$Y_1$, $Y_2$を使って和集合$Y = Y_1 \cup Y_2$と表すことができないということである。空集合は既約と見なさない。
}
\begin{rem}
  \textblue{定義として奇妙に思える。}$Y \subset X$という状況をわざわざ仮定しているのに$X$が全く出てこないのだ。$Y \subset X$という状況を生かしたいならば
『$Y \subset X$が既約部分集合であるとは、$Y \subset V_1 \cup V_2$なる$V_1, V_2 \clsub X$があれば、$Y \subset V_1$または
$Y \subset V_2$が成り立つことである』
と定義すればよい。このとき$Y$に$X$からの相対位相を入れた位相空間が既約であることと$Y$が$X$の既約部分集合であることは同値となる。証明はやさしいので読者への演習問題とする。
\end{rem}


\bfsubsection{例1.1.3}
\barquo{
既約な空間の空でない開部分集合はいずれも既約かつ稠密である。
}
\begin{proof}
  $X$は既約とする。$\emptyset \subsetneq U \opsub X$としよう。

 $U \subset F \clsub X$を任意にとる。$F \cup (X - U)=X$より、$X$の既約性と$U \neq \emptyset$により$F =X$である。したがってとくに$F = \overline{U}$として、$X = \overline{U}$が成り立つ。

$U = Y_1 \cup Y_2$なる$Y_i \clsub U$が与えられたとする。このとき$Y_i = F_i \cap U$なる$F_i \clsub X$がある。このとき$U \subset F_1 \cup F_2$で、$U$は稠密なので$F_1 \cup F_2 = X$である。よって$X$の既約性により$\exists j \; F_j =X$である。よって$Y_j = U$も成り立つ。よって既約性がいえた。
\end{proof}

\begin{comment}
\bfsubsection{命題1.2}
\barquo{
任意のイデアル$\fraka \subset A$について$I(Z(\fraka)) = \sqrt{\fraka}$, すなわち$\fraka$の根基である。
}
\begin{rem}
これは次のZariskiの補題からも示すことができる。
\begin{claim}
  $k$は体、$B$は$k$上の代数として有限生成で、$B$が体だとすると、$B/k$は有限次拡大。
\end{claim}
\end{rem}
\end{comment}

\bfsubsection{命題1.5}
\barquo{
ここで$Z=(Y - Y_1)^{-}$としよう。すると$Z=Y_2 \cup \cdots \cup Y_r$, また$Z=Y'_2 \cup \cdots \cup Y'_r$である。
}
\begin{proof}
  ここで$Z=(Y - Y_1)^{-}$は$(Y - Y_1)$の$Y$における閉包を表していることを注意しておく。見慣れない記号を使いたくないので、以下オーバーラインを使う。また、一般に位相空間$B \subset C \subset X$があるとき$B$の$C$における閉包は$X$における閉包をオーバーラインで表したとき$\overline{B} \cap C$に等しいことは今後断り無く使う。

  誤解の無いように、この証明では以下オーバーラインは$Y$における閉包を表すものと約束する。

  $Y - Y_1 \subset Y_2 \cup \cdots \cup Y_r$で、右辺は$Y$の閉部分集合なので$\overline{Y - Y_1} \subset Y_2 \cup \cdots \cup Y_r$はあきらか。逆を示そう。

  閉包作用素は有限個の和とは交換し、共通部分とは交換しないことを思い出すと
  \begin{align*}
    \overline{Y - Y_1} &= \overline{( Y_2 \cup \cdots \cup Y_r) \cap Y_1^c} \\
    &= \overline{( Y_2 \cap Y_1^c) \cup \cdots \cup (Y_r \cap Y_1^c)} \\
    &= \overline{Y_2 \cap Y_1^c} \cup \cdots \cup  \overline{Y_r \cap Y_1^c}
  \end{align*}
  ここで$j \geq 2$について$Y_j \cap Y_1^c \opsub Y_j$なので、$Y_j$の既約性により例1.1.3から$\overline{Y_j \cap Y_1^c} \cap Y_j =Y_j$である。よって$Y_j \subset \overline{Y_j \cap Y_1^c}$が成り立つ。ゆえに$Y_2 \cup \cdots \cup Y_r \subset \overline{Y - Y_1}$である。
\end{proof}



\bfsubsection{命題1.10}
\barquo{
$Z_0 \subset Z_1 \subset \cdots \subset Z_n$が$Y$の異なる既約閉部分集合の列ならば、$\overline{Z}_0 \subset \overline{Z}_1 \subset \cdots \subset \overline{Z}_n$は$\overline{Y}$の異なる既約閉部分集合の列だから
}
\begin{proof}
  まず注意しておくべきことがある。$Y$は準アファイン多様体なので、あるアファイン多様体$X$の開部分集合である。以下、命題1.10の証明において閉包はこの$X$におけるものであるとして固定する。また、$Y \opsub X$なので実際のところ$\overline{Y} = X$である。

まず$\overline{Z}_0 \subset \overline{Z}_1 \subset \cdots \subset \overline{Z}_n$が異なる集合からなるということを示そう。

$Z_i \clsub Y$なので$\exists Y_i \clsub X \; s.t. \; Z_i = Y_i \cap Y$である。このとき
\[
\overline{Z_i} \subset \overline{Y_i \cap Y} \subset \overline{Y_i} \cap \overline{Y} = Y_i
\]
が成り立つ。したがって$i < j$なら
\begin{align*}
Z_j \setminus Z_i &= Z_j \setminus Y_i \\
&\subset Z_j \setminus \overline{Y_i \cap Y} \\
&= Z_j \setminus \overline{Z_i} \\
&\subset \overline{Z_j} \setminus \overline{Z_i}
\end{align*}
が成り立つことがわかる。

また既約性は、$Z_i \subset Y$が既約なので$Z_i \subset X$も既約であり、したがって例1.1.4より従う。
\end{proof}


\bfsubsection{命題1.10}
\barquo{
$\dim Y \leq \dim \overline{Y}$である。特に$\dim Y$は有限なので
}
\begin{proof}
  これは定理1.8Aを用いて
  \[
  \dim Y \leq \dim \overline{Y} = \dim A(\overline{Y}) = \dim k[x_1, \cdots ,x_m] / I(\overline{Y}) \leq \dim k[x_1, \cdots ,x_m] = m < \infty
  \]
  と示される。
\end{proof}


\bfsubsection{命題1.10}
\barquo{
鎖$P=\overline{Z}_0 \subset \cdots \subset \overline{Z}_n$も極大となる(1.1.3).
}
\begin{rem}
  次の補題をまず示す。なお、名前は一般的なものではない。
\end{rem}
\lem{
(\textbf{閉既約性補題}) \\
$X$は位相空間、$Y \opsub X$、$W \clsub X$かつ$W$は$X$からの相対位相により既約とし、$Y \cap W \neq \emptyset$であるものとする。このとき$X$における閉包をオーバーラインで表すことにすると
$W = \overline{Y \cap W}$が成り立つ。
}
\begin{proof}
  まず$\overline{Y \cap W} \subset W$はあきらかである。逆を示そう。$Y \opsub X$により$Y \cap W \opsub W$である。すると交わりが空でないという仮定から、例1.1.3により$Y \cap W$は$W$で稠密。したがって$W = \overline{Y \cap W} \cap W \subset \overline{Y \cap W}$である。これで逆がいえた。
\end{proof}
\begin{proof}
  補題から引用部分を示す。列に対して$X$の既約閉部分集合$V$があり
  \[
\overline{Z}_0 \subset \cdots \subset \overline{Z}_i \subset V \subset \overline{Z}_{i+1} \subset \cdots \subset \overline{Z}_n
  \]
  を満たすとしよう。このとき$Y$との共通部分を考えると、$Z_i \clsub Y$なので列
  \[
  Z_0 \subset \cdots \subset Z_i \subset V \cap Y \subset Z_{i+1} \subset \cdots \subset Z_n
  \]
  を得る。極大性により、$V \cap Y$は$Z_i$または$Z_{i+1}$に等しいので、$X$における閉包をとることにより、補題から、$V$は$\overline{Z}_i$または$\overline{Z}_{i+1}$に等しいことが判る。これで極大性がいえた。
\end{proof}
\begin{rem}
  この部分について、『極大となる』という言葉の解釈を間違えているために、不要な議論に陥っているという指摘をいただいた。その指摘が正しいことは著者(@seasawher)も賛同するところだが、致命的ではないので修正はせず、ここで指摘するにとどめる。
\end{rem}


\bfsubsection{命題1.10}
\barquo{
$\overline{Z}_i$は$\frakm$に含まれる素イデアルに対応するので$\operatorname{height} \frakm = n$である。
}
\begin{proof}
  $\height \frakm \geq n$はあきらかであるので逆を示す。いま素イデアル$P_i \subset A(X)$が存在して
  \[
  P_r \subsetneq P_{r-1} \subsetneq \cdots \subsetneq P_0 = \frakm
  \]
  が成り立つと仮定する。このとき$A(X) = k[x_1,\cdots,x_m]/I(X)$のイデアルを$k[x_1,\cdots,x_m]$に持ち上げると
  \[
  I(X) \subset P_r \subsetneq P_{r-1} \subsetneq \cdots \subsetneq P_0 = \frakm
  \]
  となる。おのおの零点を考える。$Z(P_i)=W_i$とすると$W_i \clsub X$であり$W_i$は既約で
  \[
\{ P \} \subsetneq W_1 \subsetneq \cdots \subsetneq W_{r-1} \subsetneq W_r \subset X
  \]
  が成り立つ。$Y$との共通部分を考えると
  \[
  \{ P \} \subset W_1 \cap Y \subset \cdots \subset W_{r-1} \cap Y \subset W_r \cap Y \subset Y
  \]
  である。ここで$W_i \cap Y$は$W_i \cap Y \opsub W_i$により既約集合である。また閉既約性補題により、$\overline{W_i \cap Y} = W_i$であるので、この列は相異なる集合からなり
  \[
  \{ P \} \subsetneq W_1 \cap Y \subsetneq \cdots \subsetneq W_{r-1} \cap Y \subsetneq W_r \cap Y \subset Y
  \]
  が成り立つ。よって、$\dim Y = n$により$r \leq n$がいえる。したがって$\height \frakm \leq n$である。
\end{proof}


\bfsubsection{演習問題 1.2}
$\vp \colon k[x,y,z] \to k[t]$を$\vp(x)=t$、$\vp(y)=t^2$、$\vp(z)=t^3$で定めるとあきらかに$I(Y)=\Ker \vp = (x^2-y,x^3-z)$となるので$I(Y)$は素イデアルで、よって$Y$はアファイン多様体。そして$\dim Y=\dim A(Y)=\dim k[t]=1$である。



\bfsubsection{演習問題 1.7}
(a)、(c)はあきらか。(b)と(d)を示す。

(b) 位相空間$X$はNoether的であると仮定する。$X$の閉部分集合の族$\mathcal{A}$であって、有限交叉性を持つものが与えられたとする。$\calb = \setmid{F_1 \cap \cdots \cap F_n}{n \in \N , F_i \in \cala}$と定める。このとき$\calb$は有限個の共通部分をとる操作に関して閉じているので、Noether性により$\calb$には最小元$F_0$がある。$\cala$の定義により$F_0 \neq \emptyset$である。$F_0 = \bigcap \cala$である。$\cala$は任意だったから、$X$のコンパクト性がいえた。

(d) 位相空間$X$はHausdorffかつNoetherであったと仮定する。このとき任意の$x \in X$に対して$\calf_x = \setmid{F \clsub X}{\text{$F$は$x$の近傍}}$は全体集合$X$を含むので空でない。Noether性により、$\calf_x$の極小元$F_0$があるが、$\calf_x$は有限個の共通部分をとる操作に関して閉じているので$F_0$は最小元である。Hausdorff性により、$\bigcap \calf_x = \{ x\} = F_0$であるから、$\{x \}$は$X$の開部分集合。$x$は任意だったから、
$X$は離散集合。(b)より$X$はコンパクトなので、有限集合でもある。


\bfsubsection{演習問題 1.10}
\barquo{
稠密な開部分集合$U$であって、$\dim U < \dim X$となっている例を挙げよ。
}
\begin{rem}
通常のユークリッド空間の開部分集合の場合、たとえば$U \opsub \R^n$かつ$V \opsub \R^m$であって$U$と$V$が同相ならば$n=m$といった性質が成り立つ。いわば、開部分集合というのはだいたい全体と同じものなのだが、多様体の組み合わせ次元に関しては不幸にしてそういう性質は成り立たない。
\end{rem}





\bfsubsection{演習問題 1.10}
\begin{description}
  \item[(a)]$Y$の閉既約部分集合の鎖
  \[
  Z_0 \subsetneq Z_1 \subsetneq \cdots \subsetneq Z_m
  \]
  が与えられたとする。このとき$X$での閉包をとると$X$の閉既約な部分集合の列
  \[
  \ol{Z_0} \subset \ol{Z_1} \subset \cdots \subset \ol{Z_m}
  \]
  を得る。各$Z_i$は$Y$の閉部分集合だったので、これはすべて相異なる。よって$m \leq \dim X$であり、したがって$\dim Y \leq \dim X$である。
  \item[(b)]$\sup \dim U_i \leq \dim X$は(a)によりあきらか。逆を示そう。
  $X$の閉既約な部分集合の鎖
  \[
    Z_0 \subsetneq Z_1 \subsetneq \cdots \subsetneq Z_m
  \]
  が与えられたとする。族$U_i$は$X$を被覆するので$U_i \cap Z_0 \neq \emptyset$なる$i$が存在する。この$i$について
  \[
    Z_0 \cap U_i \subsetneq Z_1 \cap U_i \subsetneq \cdots \subsetneq Z_m \cap U_i
  \]
  が成り立ち、これは$U_i$における閉既約な部分集合の鎖であることが補題よりわかる。よって$m \leq \dim U_i$である。したがって$\dim X \leq \sup \dim U_i$である。
  \item[(c)] $X=\{P,Q\}$、$\calf = \{ X, \emptyset , \{ Q \}\}$、$U = \{P \}$とすればよい。ただし$\calf$は閉集合族を表す。
  \item[(d)] 次元が有限であることにより、$Y$の閉既約な部分集合の鎖
  \[
  Z_0 \subsetneq Z_1 \subsetneq \cdots \subsetneq Z_m
  \]
  であって、$m = \dim Y$であるものがとれる。$Y \clsub X$より$Z_i \clsub X$かつ$Z_i$は既約集合なので
  \[
  Z_0 \subsetneq Z_1 \subsetneq \cdots \subsetneq Z_m \subset X
  \]
  なる$X$の閉既約な集合の列を得る。$\dim X = \dim Y$より、$X = Y$でなくてはならない。
  \item[(e)]$X = \N$、$U_N = \setmid{n \in X}{n < N}$、$\calf = \{\emptyset \} \cup \{ U_N\}_{N \geq 1}$とすればよい。
\end{description}





\bfsubsection{演習問題 1.11}
$I(Y)$が素イデアルであること: $\vp \colon k[x,y,z] \to k[t]$を$\vp(x)=t^3$、$\vp(y)=t^4$、$\vp(z)=t^5$により定めると、$\Ker \vp = I(Y)$である。$\Im \vp$は整域なので$I(Y)$は素イデアル。

$I(Y)$が高さ2であること: 面倒な計算を行うと$\Ker \vp = (xz-y^2,x^3-yz,z^2-x^2y)$であることが判る。$xz-y^2$はUFDであるような係数環$k[x,z]$における素元$\frakp = (x)$に関するEisenstein多項式であるから既約、したがって素元である。よって、$0 \subsetneq (xz-y^2) \subsetneq I(Y)$という素イデアルの列が得られるので、高さは2以上である。一方多項式環$k[x,y,z]$の次元は3なので、もし$I(Y)$の高さが3なら極大イデアルでなければならない。しかし$\Im \vp = k[t^3,t^4,t^5]$は体ではないのでこれは矛盾。

$I(Y)$が2元で生成されないこと: $I(Y)$が2元で$S=k[x,y,z]$上生成されるとする。このとき$\pi \colon S \to S/(x)$による像を考えると
\begin{align*}
  \pi(I(Y)) &= (I(Y)+(x)) /(x) \\
  &= ((y^2,yz,z^2)+(x)) /(x) \\
  &= ((y,z)^2+(x)) /(x)  \\
\intertext{が成り立つ。よって}
\pi(I(Y)) / \pi((y,z)^3) &\cong k\pi(Y^2) + k\pi(YZ) + k\pi(Z^2)
\end{align*}
が成り立つ。このとき右辺は$k$加群として次元が3である。左辺は$S$加群であるが、$(x,y,z)$の元による作用は0なので、$I(Y)$が$S$上2元で生成されることから、$k$上でも2元で生成されることが従う。よって左辺の$k$加群としての次元は2以下。これは矛盾。


\newpage

\bfsection{1.2 射影多様体}
\bfsubsection{命題2.1 直前}
\barquo{
$T$を$S$の斉次元からなる任意の集合として, $T$の零点集合を
\[
Z(T) = \setmid{P \in \P^n}{\text{すべての$f \in T$について$f(P)=0$}}
\]
と定義する. $\fraka$が$S$の斉次イデアルのとき, $T$を$\fraka$の中の斉次元すべての集合として$Z(\fraka)=Z(T)$と定める.
}
\begin{rem}
  $H(S)$を$S$の斉次イデアル全体とし、$S^h$を$S$の斉次元全体とする。冪集合を以下$P$で表すことにし、上付きスターで逆像を、下付きスターで像を表すものとする。

  斉次イデアルに対してその斉次元全体を対応させる写像$H(S) \to P(S^h)$を$s$とする。また斉次元の集合からそれが生成する斉次イデアルを対応させる写像$P(S^h) \to H(S)$を$g$で表す。

  零点集合を対応させる写像を、定義域の違いにより$\widetilde{Z} \colon P(S^h) \to P(\P^n)$と$Z \colon H(S) \to P(\P^n)$との二つの記号により区別する。このとき定義から、次の図式が可換。
  \[
  \xymatrix{
  H(S) \ar[r]^Z \ar[d]_s & P(\P^n) \\
  P(S^h) \ar[ur]_{\widetilde{Z}}
  }
  \]
\end{rem}

\lem{
  次の図式は可換である。
  \[
  \xymatrix{
  H(S) \ar[r]^s \ar[dr]_1 & P(S^h) \ar[d]^g \\
  & H(S)
  }
  \]
}

\begin{proof}
  斉次イデアル$\fraka$をとる。斉次イデアルは斉次元によって生成することができるので、$\fraka \cap S^h$は生成元を含む。よって$gs(\fraka) \supset \fraka$である。逆はあきらかだから$gs = 1$がいえた。
\end{proof}
\lem{
($Z'$と$\widetilde{Z}$の基本関係式) \\
  $\pi \colon \A^{n+1} \setminus \single \to \P^n$を自然な商写像とし、$Z'$をアファイン空間での零点集合を対応させる写像、$i$、$j$を包含写像とすると次は可換。
  \[
  \xymatrix{
  P(S^h) \ar[rr]^{\widetilde{Z}} \ar[d]_{j_*} & & P(\P^n) \ar[d]^{\pi^*} \\
  P(S) \ar[r]^{Z'} & P(\A^{n+1}) \ar[r]^{i^*} & P(\A^{n+1} \setminus \single)
  }
  \]
  すなわち$T \subset S^h$に対して
  \[
  Z'(T) \setminus \single = \pi^{-1} (\widetilde{Z}(T))
  \]
  が成り立つ。
}
\begin{proof}
  ぐっとにらめばわかる。
\end{proof}
\prop{
($Z$と${\widetilde Z}$の同一性) \\
次は可換。
\[
\xymatrix{
{} & H(S) \ar[r]^Z & P(\P^n) \\
H(S) \ar[ur]^1 \ar[r]^s & P(S^h) \ar[u]_g \ar[ur]_{\widetilde Z}&
}
\]
}

\begin{proof}
  未知なのは$Zg = {\widetilde Z}$の部分である。$T \subset S^h$のとき
  \begin{align*}
    \pi^{-1}({\widetilde Z}(T)) &=  Z'(T) \setminus \single\\
    &=  Z'(gT) \setminus \single &(\text{$Z'$はイデアルと生成元を区別しない}) \\
    &\subset  Z'(sg(T)) \setminus \single \\
    &= \pi^{-1} ({\widetilde Z}(sgT))
  \end{align*}
  が成り立つ。$\pi$は全射なので
  \[
  {\widetilde Z}(T) \subset {\widetilde Z}(sg (T))
  \]
  が結論される。一方で$T \subset sg (T)$により${\widetilde Z}(sg (T)) \subset {\widetilde Z}(T)$はあきらかなので
  \[
  {\widetilde Z}(T) = {\widetilde Z}(sg(T)) = Z(g(T))
  \]
  がわかる。
\end{proof}
\prop{
($Z'$と$Z$の基本関係式) \\
  次は可換。
  \[
  \xymatrix{
  H(S) \ar[rr]^{Z} \ar[d] & & P(\P^n) \ar[d]^{\pi^*} \\
  P(S) \ar[r]^{Z'} & P(\A^{n+1}) \ar[r]^{i^*} & P(\A^{n+1} \setminus \single)
  }
  \]
  すなわち、斉次イデアル$\fraka \subset S$に対して
  \[
  \pi^{-1}(Z(\fraka)) = Z'(\fraka) \setminus \single
  \]
  が成り立つ。
}

\begin{proof}
  $g(T)= \fraka$なる$T$をとると
  \begin{align*}
\pi^{-1}(Z(\fraka)) &= \pi^{-1}(Zg(T)) \\
&= \pi^{-1}({\widetilde Z}(T)) \\
&= Z'(T) \setminus \single \\
&= Z'(\fraka) \setminus \single
  \end{align*}
  であることがわかる。
\end{proof}


\bfsubsection{命題2.2 直前}
\barquo{
$Y$を$\P^n$の任意の部分集合として, $Y$の$S$における斉次イデアルを$\setmid{f \in S}{\text{$f$は斉次ですべての$P \in Y$について$f(P)=0$}}$と定め, $I(Y)$と書く.
}
\begin{rem}
  \textblue{誤訳かと思われる。}原著では

If $Y$ is any subset of $\P^n$, we define the homogeneous ideal of $Y$ in $S$,
denoted $I(Y)$, to be the ideal generated by $\{ f \in  S \mid f \; \text{is homogeneous and} \;
f(P) = 0 \; \text{for all} \; P \in Y \}$.

となっている。$I(Y)$は$\setmid{f \in S}{\text{$f$は斉次ですべての$P \in Y$について$f(P)=0$}}$により生成されるイデアルとするのが正しい。ここで、
\[
{\widetilde I}(Y) = \setmid{f \in S}{\text{$f$は斉次ですべての$P \in Y$について$f(P)=0$}}
\]
と表すことにする。定義により次は可換。
\[
\xymatrix{
P(\P^n) \ar[r]^I \ar[dr]_{\wt{I}}& H(S) \\
{} & P(S^h) \ar[u]_g
}
\]
\end{rem}
\lem{
($I'$と${\widetilde I}$の基本関係式) \\
  次は可換。
 \[
 \xymatrix{
 P(\P^n) \ar[d]_{\pi^*} \ar[rr]^{\wt{I}} & {} & P(S^h)  \\
P(\A^{n+1} \setminus \single) \ar[r]^{i_*} & P(\A^{n+1}) \ar[r]^{I'} & P(S) \ar[u]^{j^*}
 }
 \]
  すなわち$Y \subset \P^n$に対して次が成り立つ。
  \[
  I'(\pi^{-1}(Y))\cap S^h = {\widetilde I}(Y)
  \]
}

\begin{proof}
  ぐっとにらめばわかる。
\end{proof}
\lem{
  $Y \subset \P^n$について、$I'(\pi^{-1}(Y))$は斉次イデアルである。
}

\begin{proof}
  いま$f \in I'(\pi^{-1}(Y))$とし、$f= f_0 + \cdots + f_k$と斉次元$f_i \; (\deg f_i = i)$の和に分解されたとする。このとき$y=(y_o, \cdots , y_n) \in \pi^{-1}(Y)$ならば、
  \[
  \forall \grl \in k \setminus \single \quad f(\grl y_0, \cdots , \grl y_n) = 0
  \]
  である。したがって
  \[
  \forall \grl \in k \setminus \single \quad f_0 + \grl f_1(y) + \cdots + \grl^k f_k(y) = 0
  \]
  ということになる。$k$は無限体なので$f_0 = f_1(y) = \cdots = f_n(y) = 0$でなくてはならない。$y \in \pi^{-1}(Y)$は任意だったから$f_0, \cdots , f_k \in I'(\pi^{-1}(Y)) \cap S^h$であり、ゆえに示したいことがいえた。
\end{proof}
\prop{
($I'$と$I$の基本関係式) \\
  次は可換。
  \[
  \xymatrix{
  P(\P^n) \ar[d]_{\pi^*} \ar[rr]^I & {} & H(S) \ar[d] \\
P(\A^{n+1} \setminus \single) \ar[r]^{i_*} & P(\A^{n+1}) \ar[r]^{I'} & P(S)
  }
  \]
  すなわち$I(Y)=I'(\pi^{-1}(Y))$である。
}

\begin{proof}
  $I'\pi^*$の値域が$H(S)$に含まれることがわかっているので
  \begin{align*}
    I &= g {\widetilde I} \\
    &= gsI'\pi^* \\
    &= I'\pi^*
  \end{align*}
  という計算でわかる。
\end{proof}



\prop{
  ($I$と${\widetilde I}$の同一性) \\
  次は可換。
  \[
  \xymatrix{
    P(\P^n) \ar[r]^I \ar[dr]_{\widetilde I} & H(S) \ar[d]^s \ar[dr]^1  \\
    {} & P(S^h) \ar[r]^g & H(S)
    }
\]
}

\begin{proof}
  $I'$と$I$の基本関係式からあきらか。
\end{proof}

\bfsubsection{命題 2.2}
\barquo{
素直に調べれば$\vp(Y) = Z(T')$であることがわかる。
}
\begin{rem}
  次の補題をまず示す。
\end{rem}

\lem{
  (零点変換公式$\gra$) \\
  次は可換。
  \[
  \xymatrix{
  P(S^h) \ar[d]^{\gra_*} \ar[r]^{\widetilde Z} & P(\P^n) \ar[r]^{i^*} & P(U) \ar[d]^{\vp_*} \\
  P(A) \ar[rr]_{Z'} & {} & P(\A^n)
  }
  \]
  すなわち
  \[
  Z'(\gra (T)) = \vp ({\widetilde Z}(T) \cap U)
  \]
が成り立つ。
}

\begin{proof}
  $T \subset S^h$と$x \in \A^n$に対して
  \begin{align*}
    x \in Z'(\gra (T)) &\iff \forall f \in T \quad \gra f (x)=0 \\
    &\iff f(1,x) = 0 \\
    &\iff \pi(1,x) \in \wt{Z}(T) \cap U \\
    &\iff \vp^{-1}(x) \in \wt{Z}(T) \cap U \\
    &\iff x \in \vp^{-1}(\wt{Z}(T) \cap U)
  \end{align*}
  であるから示したいことがいえた。
\end{proof}
\begin{proof}
  零点変換公式から引用部を示そう。$Y \clsub U$より、
  \begin{align*}
    \vp(Y) &= \vp(\overline{Y} \cap U) \\
    &= \vp(\wt{Z}(T) \cap U) \\
    &= Z'(\gra (T)) \\
    &= Z'(T')
  \end{align*}
  だからいえた。
\end{proof}


\bfsubsection{命題 2.2}
\barquo{
$\vp^{-1}(W) = Z(\beta(T')) \cap U$であることは簡単に確かめられる.
}
\begin{rem}
  次の二つの補題を示せば十分である。
\end{rem}
\lem{
  $\gra \beta = 1$が成り立つ。
}

\begin{proof}
  どうみてもあきらか。
\end{proof}

\lem{
(零点変換公式$\beta$) \\
  次は可換。
  \[
  \xymatrix{
P(U) & P(\P^n) \ar[l]_{i^*} & P(S^h) \ar[l]_{\wt Z} \\
P(\A^n) \ar[u]^{\vp^*} & {} & P(A) \ar[ll]_{Z'} \ar[u]_{\beta_*}
  }
  \]
  すなわち$\vp^{-1}Z'(T') = Z(\beta(T')) \cap U$である。
}

\begin{proof}
単純に計算するだけで示せる。
  \begin{align*}
    \vp^* Z' &= \vp^* Z' \gra_* \beta_* \\
&= \vp^* (\vp_* i^* \wt{Z}) \beta_* \\
&= i^* {\wt Z} \beta_*
  \end{align*}
\end{proof}




\bfsubsection{演習問題 2.1}
斉次元$f \in S$について$\deg f > 0$は$f \in I'(0)$を意味することに注意する。斉次イデアル$\fraka$について
\begin{align*}
  I(Z(\fraka)) \cap I'(0) &= I'(\pi^{-1} Z(a)) \cap I'(0) \\
  &= I'(\pi^{-1} Z(a) \cup \single) \\
  &\subset I'(Z'(\fraka)) \\
  &= \sqrt{\fraka}
\end{align*}
が成り立つことから判る。



\bfsubsection{演習問題 2.2}
\begin{description}
  \item[(i) $\to$ (ii)] $Z(\fraka)=\emptyset$により$Z'(\fraka)\setminus \single = \pi^{-1}(Z(\fraka))=\emptyset$である。したがって$Z'(\fraka)$は$\single$または$\emptyset$なので、零点定理により$\sqrt{\fraka} = I'(Z'(\fraka)) = S_+ \text{ または } S$である。
  \item[(ii) $\to$ (iii)] 仮定より$S_1 \subset {\sqrt \fraka}$である。よってある$k \in \N$が存在してすべての$i$について$x_i^k \in \fraka$となる。このとき$d = (n+1)(k-1)+1$とすると$S_d \subset \fraka$が成立する。
  \item[(iii) $\to$ (i)] $Z(\fraka) \subset Z(S_d) = \emptyset$よりあきらか。
\end{description}



\bfsubsection{演習問題 2.3}
\begin{description}
  \item[(a)] あきらか
  \item[(b)] ${\wt I}(Y_1) \supset {\wt I}(Y_2)$より、$I(Y_1) \supset {\wt I}(Y_2)$である。$I(Y_1)$はイデアルなので$I(Y_1) \supset I(Y_2)$である。
  \item[(c)] 次のように計算すればわかる。
  \begin{align*}
    I(Y_1 \cup Y_2) &= I'(\pi^{-1}(Y_1 \cup Y_2)) \\
    &= I'(\pi^{-1}Y_1 \cup \pi^{-1}Y_2)) \\
    &= I'(\pi^{-1}Y_1) \cap I'(\pi^{-1}Y_2) \\
    &= I(Y_1) \cap I(Y_2)
  \end{align*}
\end{description}


\bfsubsection{演習問題 2.4}
\begin{description}
  \item[(a)]$Z(S_+) = Z(S)= \emptyset$なので$S_+$と$S$のどちらかを除く必要があるが、$I(\emptyset)= S$なので除かれるべきなのは$S_+$である。あとはアファイン空間のときと同様。
  \item[(b)]『斉次イデアル$\fraka$が素イデアルであることは、任意の斉次元$f$、$g$について$fg \in \fraka$ならば$f \in \fraka$または$g \in \fraka$であることと同値である』ことに注意すれば、例1.4と同様。
  \item[(c)]$I(\P^n)=0$よりあきらか。
\end{description}



\bfsubsection{演習問題 2.5}
\begin{description}
  \item[(a)]例1.4.7と同様。
  \item[(b)]命題1.5よりあきらか。
\end{description}






\bfsubsection{演習問題 2.6}
\begin{rem}
  まず次の二つの補題を用意しておく。この補題を使わなくても十分証明できるが、証明の見通しをよくするために用意した。
\end{rem}
\lem{
  (随伴性) \\
  次が成り立つ。
  \begin{description}
    \item[(1)] 写像$f \colon X \to Y$があるとき$A \subset X$、$B \subset Y$に対して
    \[
    f(A) \subset B \iff A \subset f^{-1}(B)
    \]
    \item[(2)] $Z \colon H(S) \to P(\P^n)$と$I \colon P(\P^n) \to H(S)$を考える。斉次イデアル$\fraka$と$Y \subset \P^n$が与えられたとき
    \[
Y \subset Z(\fraka) \iff \fraka \subset I(Y)
    \]
    同様のことが$Z'$、$I'$や${\wt Z}$、${\wt I}$についても成り立つ。
  \end{description}
}
\begin{proof}
  (2)の$Z$、$I$の場合だけが非自明である。$Y \subset Z(\fraka)$とすると
  \begin{align*}
    Y \subset Z(\fraka) &\Rightarrow Y \subset {\wt Z}s(a) \\
    &\Rightarrow s(\fraka) \subset {\wt I}(Y)  \\
    &\Rightarrow \fraka \subset g{\wt I}(Y) \\
    &\Rightarrow \fraka \subset I(Y)
  \end{align*}
  となる。逆に$\fraka \subset I(Y)$と仮定すると
  \begin{align*}
    \fraka \subset I(Y) &\Rightarrow s(\fraka) \subset sI(Y) \\
    &\Rightarrow s(a) \subset {\wt I}(Y) \\
    &\Rightarrow Y \subset {\wt Z}(s(\fraka)) \\
    &\Rightarrow Y \subset Z(\fraka)
   \end{align*}
\end{proof}

\lem{
  (イデアル変換公式$\beta$) \\
$U = U_0$、$\vp = \vp_0$とおく。このとき次が可換。
\[
\xymatrix{
P(U) \ar[d]_{\vp_*} \ar[r]_{i_*} & P(\P^n) \ar[r]^{\wt{I}} & P(S^h)  \ar[d]_{\beta^*} \\
P(\A^n) \ar[rr]^{I'}  & {} & P(A)
}
\]
すなわち$W \subset U$に対して
\[
\beta^{-1}({\wt I}(W)) = I'(\vp(W))
\]
が成り立つ。
}

\begin{proof}
  随伴性により零点変換公式$\beta$に帰着することができる。$W \subset U$、$W' \subset A$に対して、
  \begin{align*}
    W' \subset \beta^* {\wt I} i_*(W)
    &\iff \beta_*(W') \subset  {\wt I} i_*(W) \\
    &\iff i_*(W) \subset {\wt Z} \beta_*(W') \\
    &\iff W \subset i^*{\wt Z} \beta_*(W') \\
    &\iff W \subset \vp^* Z' (W') \\
    &\iff \vp_*(W) \subset Z'(W') \\
    &\iff W' \subset I'\vp_*(W)
  \end{align*}
  したがって示したいことがいえた。
\end{proof}


\begin{proof}
演習2.6の証明を続ける。$A_i = k[x_0, \cdots , \widehat{x_i} , \cdots , x_n]$とする。$I' \colon P(\A^n) \to P(A_i)$として$A(Y_i) = A_i/I'(Y_i)$と定義する。

以下、$Y \cap U_i \neq \emptyset $なる$i$についてだけ考える。$i=0$として一般性を失わない。そうして$A_0 = A$と書く。このとき$x_0 \notin I(Y)$なので$x_0 \neq 0 \; \text{in} \; S(Y)$であることに注意する。したがって局所化$S(Y)_{x_0}$を考えることができる。このとき
\[
S(Y)_{x_0} = S(Y) \otimes_S S_{x_0} = S_{x_0} / I(Y)S_{x_0}
\]
である。自然な写像$S_{x_0} \to S(Y)_{x_0}$を$\sigma$とおく。

$S_{x_0}$の次数$0$の元からなる部分環を$R$とおく。環準同型$\psi \colon A \to R$を
\[
\psi(f)(x_0, \cdots , x_n) = f\left( \f{x_1}{x_0}, \cdots , \f{x_n}{x_0} \right) = x_0^{- \deg f}\beta f
\]
により定める。$\psi$はあきらかに全射である。

イデアル変換公式$\beta$を$Y \cap U_0$に適用することにより$I'(Y_0) = \beta^{-1}(\wt{I}(Y \cap U_0))$が成り立つことが判る。したがって$f \in A$に対して
\begin{align*}
  f \in \Ker (\sigma \psi) &\iff \psi(f) \in I(Y)S_{x_0} \\
  &\iff x_0^{- \deg f} \beta f \in I(Y)S_{x_0} \\
  &\iff  \exists d \geq 0 \quad x_0^{d} \beta f \in I(Y) \cap S^{h} = {\wt I}(Y) \\
  &\iff \beta f \in \wt{I}(Y \cap U_0)  &(\text{$U_0$の外では$x_0$は$0$である}) \\
  &\iff f \in I'(Y_0)
\end{align*}
が成り立つことが判る。よって$A(Y_0) \cong \sigma(R)$である。よって$\psi$を$\psi(x_0) = x_0$で延長して、
\[
A(Y_0)[x_0,x_0^{-1}] \cong \sigma(R)[x_0,x_0^{-1}] \cong S(Y)_{x_0}
\]
である。

よって、整域の商体をとる操作を$\Frac$で表すことにすると
\begin{align*}
  \dim S(Y) &= \trdeg_k \Frac(S(Y)) &(\text{定理1.8Aより})\\
  &= \trdeg_k \Frac(S(Y)_{x_0}) \\
  &= \trdeg_k \Frac(A(Y_0)[x_0,x_0^{-1}]) \\
  &= \trdeg_k \Frac(A(Y_0)) + 1 \\
  &= \dim A(Y_0) + 1 \\
  &= \dim Y_0 + 1 &(\text{命題1.7より})
\end{align*}
が結論される。

ここまで$i = 0$としてきたが、同様のことが$Y \cap U_i \neq \emptyset$なる任意の$i$について成り立つ。よってそのような$i$について
\[
\dim Y_i = \dim S(Y) - 1
\]
である。ゆえに
\begin{align*}
  \dim Y &= \sup \dim (Y \cap U_i) &(\text{演習1.10より}) \\
  &= \sup_i \dim Y_i &(\text{$\vp_i$は同相写像}) \\
  &= \dim S(Y) - 1
\end{align*}
が結論される。
\end{proof}



\bfsubsection{演習問題 2.7}
\begin{rem}
  次の補題を用意しておく。
\end{rem}
\lem{
  位相空間$X$があり、$Y \subset X$、$U \opsub X$であったとする。このとき
  \[
  \ol{Y} \cap U \subset \ol{Y \cap U}
  \]
  が成り立つ。したがってとくに$\ol{Y} \cap U = \ol{Y \cap U} \cap U$である。この等式は、$X$での閉包をとってから$U$に制限することと、$U$へ制限してから$U$で閉包をとることが同じだと主張していることに注意。
}
\begin{proof}
『したがってとくに』以降はあきらか。前半を示す。

$x \in \ol{Y} \cap U$が与えられたとする。すると$x \in \ol{Y}$なので、ある有向点列$\{x_{\gra}\} \subset Y$が存在して$x_{\gra} \to x$である。$x \in U$より、$U$は$x$の開近傍であるので、ある$N$が存在して$\gra \geq N$ならば$x_{\gra} \in U$が成り立つ。よって$\{ x_{\gra} \}_{ \gra \geq N } \subset Y \cap U$は$x$
に収束する有向点列なので$x \in \ol{Y \cap U}$が結論される。
\end{proof}
\begin{proof}
  演習2.7の証明に戻る。
  \begin{description}
    \item[(a)] $\P^n$は射影多様体なので
    \[
    \dim \P^n = \dim S(\P^n) -1 = \dim S - 1 = n
    \]
    \item[(b)] 演習2.6の記法を用いて、適当な$i$を選べば
    \begin{align*}
      \dim \ol{Y} &= \dim \vp_i(\ol{Y} \cap U_i) &(\text{演習2.6より}) \\
      &= \dim \vp_i (\ol{Y \cap U} \cap U) &(\text{開集合への制限と閉包の可換性より}) \\
      &= \dim \ol{\vp_i (Y \cap U_i)} \\
      &= \dim \vp_i(Y \cap U_i) &(\text{命題1.10より}) \\
      &= \dim (Y \cap U_i) \\
      &\leq \dim Y
    \end{align*}
    逆は明らかなので$\dim \ol{Y} = \dim Y$である。
  \end{description}
\end{proof}

\bfsubsection{演習問題 2.9}
\barquo{
この例を用いて、$f_1, \cdots , f_r$が$I(Y)$を生成しても$\beta(f_1), \cdots , \beta(f_r)$が$I(\ol{Y})$を生成するとは限らないことを示せ。
}
\begin{proof}
  一般に、$A = k[x_1, \cdots , x_n]$, $B = A[w]$として、$f_1, \cdots , f_r \in A$とするとき
  \[
  \sum_{i=1}^r (\beta f_i) B \subset \beta \left( \sum_{i=1}^r f_i A \right) \cdot B
  \]
  は成り立つが、逆の包含は言えない。なぜかというと、$\gra \colon B \to A$は全射環準同型であるが、$\beta \colon A \to B$は環準同型ではないからだ。$\beta$は単射であり積を保つが、和を保たないのである。より正確にいうと$F = \sum_{i=1}^r f_i$のとき
  \[
  \beta \left(\sum_{i=1}^r f_i \right) = \sum_{i=1}^r w^{\deg F - \deg f_i} \beta f_i
  \]
  である。したがって、$g_1, \cdots , g_r \in A$があるとき$H = \sum_{i=1}^r f_i g_i$とすると
  \begin{align*}
\beta \left( \sum_{i=1}^r f_i g_i \right) = \sum_{i=1}^r w^{\deg H - \deg (f_i g_i)} \beta(f_i)\beta( g_i)
  \end{align*}
  である。このとき$\deg H - \deg (f_i)$は必ずしも$0$以上ではないので、
  \[
 w^{\deg H - \deg (f_i g_i)} \beta( g_i) \in B
  \]
  とは限らない。たとえば$r=2$, $f_1 = x^4y^3 + 1$, $f_2 = x^3y^4 + 1$, $g_1 = y$, $g_2 = -x$としてみればわかる。当たり前だが、$r=1$のときにはこのような問題は起こらない。
\end{proof}



\bfsubsection{演習問題 2.9}
\begin{description}
  \item[(a)] $X = \vp_0^{-1}(Y)$とする。$g \colon P(S^h) \to H(S)$を斉次元の集合に対してそれが生成する斉次イデアルを対応させる写像とし、$s \colon H(S) \to P(S^h)$を斉次イデアルに対してその斉次元の全体を対応させる写像とする。$\ol{X} = Z(I(X))$より、
  \begin{align*}
  I(\ol{X}) &= I(Z(I(X))) \\
  &= \sqrt{I(X)} &(Z(I(X)) = \ol{X} \neq \emptyset \text{による}) \\
  &= I(X) &(\text{斉次元をとって各自確認せよ})
  \end{align*}
  がわかる。だから$I(\ol{X})$でなく$I(X)$を考えればよい。

  $I(X) \supset g(\beta (I(Y)))$を示す: $g \in I(Y)$と$P \in X \subset U_0$が与えられたとする。このとき
  \begin{align*}
    \beta(g)(P) &= \left( x_0^e g\left( \f{x_1}{x_0}, \cdots , \f{x_n}{x_0} \right) \right) (P) \\
    &= p_0^e g(\vp_0(P)) \\
    &= 0
  \end{align*}
  であるから、示せた。

  $I(X) \subset g(\beta (I(Y)))$を示す: $f \in I(X) \cap S^h$と$Q \in Y$が与えられたとする。このとき
  \begin{align*}
    (\gra f)(Q) &= f(1, x_1, \cdots , x_n)(Q) \\
    &= f(\vp_0^{-1}(Q)) \\
    &= 0
  \end{align*}
  であるから$\gra f \in I(Y)$である。そうすると
  \begin{align*}
    f &= x_0^{\deg f} f\left( 1, \f{x_1}{x_0} , \cdots , \f{x_n}{x_0} \right) \\
    &= x_0^{\deg f - \deg \gra(f)} \beta(\gra (f))
  \end{align*}
  より$f \in g(\beta(I(Y)))$である。ゆえに$s(I(X)) \subset g(\beta(I(Y)))$であるが、これはすなわち$I(X) \subset g(\beta(I(Y)))$を意味する。
  \item[(b)] $A=k[x,y,z]$, $S=k[w,x,y,z]$としてこの上でイデアルを考えることにする。

  $I(Y)$の生成元: 演習問題1.2により、$I(Y)=(y-x^2,z-x^3)$である。

  $I(\ol{Y})$の生成元: $X = \vp_0^{-1}(Y) = \setmid{(1,t,t^2,t^3) \in \P^3}{t \in k}$とおく。このとき$I(\ol{Y}) = I(X)$である。実は$I(X)=(yw-x^2, zw-xy, y^2-xz)$が成り立つことを確認する。$(yw-x^2, zw-xy, y^2-xz) \subset I(X)$はあきらかであるので、逆を示せばよい。$J = (yw-x^2, zw-xy, y^2-xz)$とおき、割り算を実行する。まず$yw-x^2$で割ると
  \[
  S = (yw-x^2)S + xk[w,y,z] + k[w,y,z]
  \]
  を得る。次に$y^2-xz$で割ると
  \[
S = (yw-x^2, y^2-xz)S + xk[w,y] + k[w,y,z]
  \]
  を得る。さらに$zw-xy$で割ると
  \begin{align*}
  S &= J + zk[w,z] + xk[w] + k[w,y,z] \\
  &= J + xk[w] + k[w,y,z]
\end{align*}
  を得る。最後に$z^2w-y^3 = z(zw-xy) -y(y^2-xz) \in J$で割ると
  \[
  S = J + xk[w] + y^2k[w,z] + yk[w,z] + k[w,z]
  \]
  を得る。ゆえに$f \in I(X) \cap S^h$とすると
  \[
  f(w,x,y,z) + xf_0(w) + y^2f_1(w,z) + yf_2(w,z) + f_3(w,z) \; \in J
  \]
  なる$f_0, \cdots , f_3$がある。よって任意の$t \in k$と$s \in k \setminus \single$について
  \[
  tsf_0(s) + t^4sf_1(s,t^3s) + t^2sf_2(s,t^3s) + f_3(s,t^3s) = 0
  \]
  が成り立つことがいえる。この式で、$t$についての次数が$1$かそうでないかを見ることにより$tsf_0(s)=0$が言える。$k$は無限体なのでこれは$f_0$が多項式として$0$であることを意味する。また残った部分について$t$についての次数を$\bmod 3$で分けることができて$t^4sf_1(s,t^3s)=t^2sf_2(s,t^3s)=f_3(s,t^3s) = 0$がいえる。一つの変数を止めて$1$変数多項式だとみなすごとに零多項式なので、これは$f_1 = f_2 = f_3 =0$ということである。したがって$f \in J$がいえた。よって$I(X) \cap S^h \subset J$であり、ゆえに$I(X) \subset J$である。

  $I(Y)$の生成元を$\beta$で送ったものが$I(\ol{Y})$を生成するとは限らないこと: $(\beta(y-x^2), \beta(z-x^3)) = (yw-x^2, zw^2-x^3)$である。これは$\setmid{(0,0,s,u)}{s,t \in k \setminus \single}$を零点にもつが、$I(\ol{Y}) = J$はこの点を零点として持たない。よって$J$は$(yw-x^2, zw^2-x^3)$と一致しない。
\end{description}





\bfsubsection{演習問題 2.12}
\barquo{
点$P=(a_0, \ldots , a_n)$を$\rho_d(P) = ( M_0(a), \ldots , M_N(a))$ (単項式$M_j$に$a_i$を代入して得られる点)に送ることにより
}
\begin{rem}
\textblue{誤植であると思われる。}正しくは$\rho_d(P) = ( M_0(P), \ldots , M_N(P))$である。そのあとのコメントは意図不明。
\end{rem}



\bfsubsection{演習問題 2.12}
\begin{description}
  \item[(a)] $\grt$は斉次元の次数を$d$倍するので$\fraka$は斉次イデアル。また送る先が整域なので$\fraka$は素イデアル。
  \item[(b)] まず$\rho_d(\P^n) \subset Z(\fraka)$を示そう。こちら側は極めて易しい。$T = k[y_0, \cdots , y_N]$, $S=k[x_0, \cdots , x_n]$とおく。$y \in \rho_d(\P^n)$とする。そうすると$y=\rho_d(P)$なる$P \in \P^n$が存在する。このとき$f \in \fraka \cap T^h$に関して$f(y)=f(\rho_d(P))=(\grt f)(P)= 0$である。したがって
  $y \in \wt{Z}(\fraka \cap T^h) = Z(\fraka)$である。これで$\rho_d(\P^n) \subset Z(\fraka)$がいえた。

  逆を示そう。いま$m = (m_0, \cdots , m_N) \in Z(\fraka)$が与えられたとする。$0 \leq i \leq n$について
  \[
   M_{\grs(i)}=x_i^d
  \]
  と定めよう。このとき任意の$M_{\grd} = \prod_{k=0}^n x_k^{p(\grd, k)}$ $(0 \leq \delta \leq N)$に対して
  \begin{align*}
    (M_{\grd})^d &= \prod_{k=0}^n x_k^{p(\grd, k)d } \\
    &= \prod_{k=0}^n (M_{\grs(k)})^{p(\grd, k)} \\
    &= M_{\grd}(M_{\grs(0)}, \cdots , M_{\grs(n)} )
  \end{align*}
  が成り立つ。したがって、
  \[
  y_{\grd}^d - M_{\grd}(y_{\grs(0)}, \cdots , y_{\grs(n)}) \in \fraka \cap T^h
  \]
  であるから、任意の$0 \leq \grd \leq N$について
  \[
    m_{\grd}^d = M_{\grd}(m_{\grs(0)}, \cdots , m_{\grs(n)})
  \]
  が成り立つことがいえる。$m \in \P^N$より、ある$m_k$は$0$でないから、$0$でない$m_{\grs(i)}$が存在することが判る。ここで$0 \leq i \leq n$, $0 \leq j \leq n$について
  \[
M_{\grs(i,j)}=x_i^{d-1}x_j
  \]
  とおく。すると
  \begin{align*}
    (M_{\grs(i)})^{d-1}M_{\grd} &= x_i^{(d-1)d} \prod_{k=0}^n x_k^{p(\grd, k)} \\
    &= \prod_{k=0}^n x_i^{(d-1)p(\grd,k)} x_k^{p(\grd, k)} \\
    &= \prod_{k=0}^n (x_i^{d-1}x_k)^{p(\grd, k)} \\
    &= \prod_{k=0}^n (M_{\grs(i,k)})^{p(\grd, k)} \\
    &= M_{\grd}(M_{\grs(i,0)}, \cdots , M_{\grs(i,n)})
  \end{align*}
  が成り立つ。これは
  \[
  y_{\grs(i)}^{d-1}y_{\grd} - M_{\grd}(y_{\grs(i,0)} , \cdots , y_{\grs(i,n)}) \in \fraka \cap T^h
  \]
  を意味するので、これにより
  \[
  m_{\grs(i)}^{d-1}m_{\grd} = M_{\grd}(m_{\grs(i,0)} , \cdots , m_{\grs(i,n)})
  \]
  である。とくに$m_{\grs(i)} \neq 0$なる$i$をとって固定すると
  \[
\f{m_{\grd}}{m_{\grs(i)}} = M_{\grd}\left( \f{m_{\grs(i,0)}}{m_{\grs(i)}}, \cdots , \f{ m_{ \grs(i,n) }}{  m_{\grs(i)}  }   \right)
  \]
  である。ゆえに
  \begin{align*}
    m &= (m_0, \cdots , m_N) \\
    &= \left( \f{m_0}{m_{\grs(i)}} , \cdots , \f{m_N}{m_{\grs(i)}} \right) \\
    &= \rho_d \left( \f{m_{\grs(i,0)}}{m_{\grs(i)}}, \cdots , \f{ m_{ \grs(i,n) }}{ m_{\grs(i)} } \right) \\
    &= \rho_d ( m_{\grs(i,0) }, \cdots ,  m_{\grs(i,n)} )
  \end{align*}
  が成り立つ。すなわち、$m \in \rho_d(\P^n)$がいえた。
 \item[(c)] 全射であること: (b)によりあきらか。

 単射であること: $P,Q \in \P^n$について$\rho_d(P)=\rho_d(Q)$であるとする。このとき任意の$0 \leq i \leq n$について$p_i^d = q_i^d$である。そこで$p_i \neq 0$かつ$q_i \neq 0$となる$i$をとって固定することができる。すると任意の$0 \leq j \leq n$について$p_i^{d-1} p_j= q_i^{d-1} q_j$であるから、両辺を割って$p_j/p_i=q_j/q_i$
 を得る。よって$P=Q$であり、単射性が言えた。

 連続であること: $Y \clsub Z(\fraka)$が与えられたとする。すると$Y = Z(J)$なる斉次イデアル$J \subset T$がある。このとき$P \in \P^n$について
 \begin{align*}
   P \in \rho_d^{-1}(Z(J)) &\iff \rho_d(P) \in Z(J) \\
   &\iff \forall f \in J \cap T^h \quad f(\rho_d(P))=(\grt f)(P)= 0 \\
   &\iff P \in {\wt Z}(\grt (J \cap T^h)) \\
   &\iff P \in Z(\grt (J) S)
 \end{align*}
 が成り立つので、$\rho_d^{-1}(Z(J)) = Z(\grt (J))$である。よって$\rho_d$は連続。

 閉写像であること: $Z(I) \clsub \P^n$とする。$\rho_d(Z(I)) = Z(\grt^{-1}(I))$を示そう。$\rho_d(Z(I)) \subset  Z(\grt^{-1}(I))$は(b)の前半と同様なので、逆を示す。いま$Q \in Z(\grt^{-1}(I))$と$f \in I \cap S^h$が与えられたとする。(b)での結果により、$\rho_d$は全射なので$Q = \rho_d(P)$なる$P \in \P^n$がある。また、$f^d = \grt(g)$なる$g \in T$がある。すると
 $g \in \grt^{-1}(I) \cap T^h$であって、
 \begin{align*}
   f^d(P) &= (\grt g)(P) \\
   &= g(\rho_d(P)) \\
   &= g(Q) \\
   &= 0
 \end{align*}
 である。ゆえに$P \in {\wt Z}(I \cap S^h) =Z(I)$である。よって$Q \in \rho_d(Z(I))$がいえたので、$\rho_d$は閉写像、したがって同相であることが結論される。
 \item[(d)] $M_0 = x_0^3$, $M_1 = x_0^2x_1$, $M_2 = x_0x_1^2$, $M_0 = x_1^3$とする。このとき
 \[
 \rho_3(U_0 \cap \P^1) = \setmid{(1,t,t^2,t^3) \in \P^3}{t \in k} = X
 \]
 であるから
 \[
 \rho_3(\P^1) =  \rho_3(\ol{U_0 \cap \P^1}) = \ol{\rho_3(U_0 \cap \P^1)} = \ol{X}
 \]
 が判る。
 \end{description}



 \bfsubsection{演習問題 2.14}
 \barquo{
 $\psi$の像が$\P^N$の部分多様体であることを示せ。
 }
 \begin{rem}
   部分多様体(subvariety)という語は演習問題3.10で定義されている模様。
 \end{rem}




 \bfsubsection{演習問題 2.14}
 $\psi$がwell-definedであること: あきらか。

 $\psi$が単射であること: $\psi(a,b)=\psi(a',b')$とする。任意の$0 \leq i \leq r$, $0 \leq j \leq s$に対して$a_ib_j = a'_ib'_j$である。$b_j \neq 0$なる$j$を固定すると
 \[
 a_i = a'_i \cdot \f{b'_j}{b_j}
 \]
 である。$a_i \neq 0$なる$i$があるので、このとき$\f{b'_j}{b_j} \neq 0$である。よって$a = a'$がわかる。同様に$b = b'$である。

 $I(\Im \psi)$の構成: $\P^N$の座標環を$k[z_{ij}]$とする。準同形
 \[
 \grs \colon k[z_{ij}] \to k[x_0, \cdots , x_r, y_0, \cdots , y_s] \st \grs(z_{ij}) = x_i y_j
 \]
 を考え、$\fraka = \Ker \grs$とする。

 $Z(\fraka )=\Im \psi$であること: $Z(\fraka ) \supset \Im \psi$は、次の図式の可換性、つまり斉次元$f \in k[z_{ij}]$について$f \circ \psi = \grs f$が成り立つことによる。
 \[
 \xymatrix{
 \P^N \ar[r]^f & \{ 0, 1 \} \\
 \P^r \tm \P^s \ar[u]^{\psi} \ar[ur]_{\grs f} & {}
 }
 \]
 ここで$\grs f \in k[x_0, \cdots , x_r, y_0 , \cdots , y_s]$は、$x$と$y$のそれぞれについて斉次元であるため、$\P^r \tm \P^s$上の関数としてwell-definedであることに注意する。

 逆を示そう。$P = (P_{ij}) \in Z(\fraka)$が与えられたとする。このとき任意の$0 \leq i_0, i_1 \leq r$, $0 \leq j_0, j_1 \leq s$について
 \[
 z_{i_0 j_0} z_{i_1 j_1} -  z_{i_1 j_0} z_{i_0 j_1} \; \in \fraka
 \]
 が成り立つことから、
 \[
 P_{i_0 j_0} P_{i_1 j_1} =  P_{i_1 j_0} P_{i_0 j_1}
 \]
 である。さてここで$P_{lm} \neq 0$なる$l$, $m$が存在するので、ひとつ取って固定する。$a = (P_{0m}, \cdots , P_{rm})$, $b = (P_{l0}, \cdots , P_{ls})$とおく。このとき
 \[
 a_i b_j = P_{im} P_{lj} = P_{ij} P_{lm}
 \]
 が成り立つので、$\psi(a,b)=P$である。ゆえに$Z(\fraka ) \subset \Im \psi$がいえた。

 $\Im \psi \subset \P^N$が部分多様体であること: $\fraka$は整域への準同形の核だから素イデアル、したがって$\Im \psi$は既約集合であり、かつ$\P^N$の閉部分集合。ゆえに部分多様体。


\newpage

\bfsection{1.3 射}

\bfsubsection{補題3.1}
\barquo{
これは局所的に確かめることができる :位相空間$Y$の部分集合$Z$は、各$U$について$Z \cap U$が$U$において閉であるような開部分集合$U$たちで$Y$を覆うことができるとき、またそのときに限って閉である。
}
\begin{rem}
  次の補題の形で示す。
\end{rem}
 \lem{
  (閉集合の局所判定) \\
$Y$は位相空間で$\{U_i\}$はその開被覆であるとする。このとき部分空間$Z \subset Y$に対して次は同値。
\begin{enumerate}
  \item $Z \subset Y$は閉集合
  \item 任意の$i$について$Z \cap U_i \clsub U_i$が成り立つ。
\end{enumerate}
}
\begin{proof}
  1. $\To$ 2.はあきらかであるので逆を示す。
  \[
  Z^c = \bigcup (U_i \cap Z^c) = \bigcup (U_i \setminus (U_i \cap Z))
  \]
  と表すと、$U_i \opsub Y$なので$Z^c \opsub Y$がわかる。
\end{proof}



\bfsubsection{定義-局所環}
\barquo{
$P$が$Y$の点のとき、$Y$上の$P$の局所環$\calo_{P,Y}$(あるいは単に$\calo_P$)を、$P$の近くの$Y$上の正則函数の芽のなす環と定義する。
}
\begin{rem}
  定義から次がただちに従うことに注意する。
\end{rem}
\prop{
$Y$は多様体で$ \emptyset \subsetneq U \opsub Y$であるとする。このとき点$P \in U$に対して
$\calo_{P,Y} = \calo_{P,U}$が成り立つ。
}
\begin{proof}
  $\kakko{V,f} \in \calo_{P,Y}$ならば
  \[
  \kakko{V,f} = \kakko{V \cap U,f} \; \in \calo_{P,U}
  \]
  である。$P \in V \cap U$なのでこれは空でないことに注意。よって$\calo_{P,Y} \subset \calo_{P,U}$である。$U \opsub Y$により、$U$の開部分集合は$Y$の開部分集合でもあるので$\calo_{P,U} \subset \calo_{P,Y}$もいえる。
\end{proof}



\bfsubsection{定理3.2 直前}
\barquo{
$Y$を同型な多様体で置き換えるとき、対応する環たちは同型である。
}
\begin{rem}
  $\calo$は、多様体の圏から環の圏への反変関手である。実際、$f \colon X \to Y$が射であれば環準同型
  \[
  f^* \colon \calo (Y) \to \calo(X)  \st f^*(g) = g \circ f
  \]
  が誘導される。よって不変量であることが従う。また、$P \in X$に対しても$f \colon X \to Y$が射であれば
  \[
  f^* \colon \calo_{f(P),Y} \to \calo_{P,X} \st  f^*(g) = g \circ f
  \]
  も環準同型であることがわかるので、$\calo_P$も不変量。

  一方で、$K$はそうではない。$f \colon X \to Y$の像が空でない開集合を含む(したがって稠密)ような場合を除いては、$f$による空でない開集合の引き戻しは空でないとは限らないからである。もちろん$f$が同型なら問題はないが、完全に同様とはいかないので少しだけ注意が必要である。
\end{rem}



\bfsubsection{定理 3.2}
\barquo{
各$P$について自然な写像$A(Y)_{\frakm_P} \to \calo_P$がある。$\alpha$が単射だからこれも単射であり、正則関数の定義から全射である!
}
\begin{proof}
自然な写像の合成$A(Y) \to \calo(Y) \to \calo_P$による$A(Y) \setminus \frakm_P$の像は単元なので、局所化の普遍性により写像$A(Y)_{\frakm_P} \to \calo_P$が誘導される。これは単射により誘導される写像なので単射である。全射性は本文通り。
\end{proof}


\bfsubsection{定理 3.2}
\barquo{
(c)から$A(Y)$の商体はすべての$P$について$\calo_P$の商体に同型であるが、これは$K(Y)$に等しい。というのは、すべての有理函数は実際にはある$\calo_P$の中にあるからである。
}
\begin{rem}
商体をとる操作を$\Frac$で表すことにする。このとき
\begin{align*}
  \Frac (A(Y)) &= \bigcup_P A(Y)_{\frakm_P} \\
  &\cong \bigcup_P \calo_P \\
  &= K(Y)
\end{align*}
だから$A(Y)$の商体は$K(Y)$である。このことから$\calo_P$の商体が$K(Y)$であることがわかる。
\end{rem}



\bfsubsection{命題 3.3}
\barquo{
$U_i \subset \P^n$を方程式$x_i \neq 0$により定義される開集合とする。このとき前出の(2.2)の写像$\vp_i \colon U_i \to \A^n$は多様体の同型である。
}
\begin{proof}
簡単のため$i = 0$としてよい。また$\vp_0 = \vp$、$U_0 = U$と書くことにする。$\vp$と$\vp^{-1}$が射であることを示せば十分だろう。

$\vp$が射であること:
  \[
  \xymatrix{
  U \ar[r]^{\vp} & \A^n \\
  \vp^{-1}(V) \ar[u] \ar[r]^{\vp} \ar[dr]_{f \circ \vp} & V \ar[u] \ar[d]^f \\
  {} & k
  }
  \]
  正則関数$f \colon V \to k$と点$p \in \vp^{-1}(V)$が与えられたとする。$\vp(p) \in V$なので$f$の正則性により
  \[
  \exists V' \; \vp(p) \in V' \opsub V \; \exists h,g \in A  \st \; \forall x \in V' \; f(x) = \f{g(x)}{h(x)}
  \]
  が成り立つ。このとき
  \begin{align*}
    \forall y \in \vp^{-1}(V') \quad f \circ \vp(y) &= \f{g(\vp(y))}{h(\vp(y))} \\
    &= \f{g(y_1 / y_0, \cdots , y_n/y_0 )}{h(y_1 / y_0, \cdots , y_n/y_0 )} \\
    &= \f{\beta g(y_0 , \cdots , y_n)y_0^{- \deg g} }{\beta h(y_0 , \cdots , y_n )y_0^{- \deg h}} \\
    &= \f{\beta g(y) y_0^{ \deg h} }{\beta h(y) y_0^{\deg g}} \\
  \end{align*}
  であるから、$f \circ \vp$は$p$において正則である。$p \in \vp^{-1}(V)$は任意だったから、$f \circ \vp$は$\vp^{-1}(V)$上で正則であることがわかる。

$\vp^{-1}$が射であること:
  \[
  \xymatrix{
  U  & \A^n \ar[l]_{\vp^{-1}} \\
  V  \ar[u] \ar[d]_f   &  \vp(V) \ar[u] \ar[l]_{\vp^{-1}}  \ar[dl]^{f \circ \vp^{-1}} \\
  k &  {}
  }
  \]
  正則関数$f \colon V \to k$と$p \in \vp(V)$が与えられたとする。$\vp^{-1}(p) \in V$なので、$f$の正則性により
  \[
  \exists V' \; \vp^{-1}(p) \in V' \opsub V \; \exists h,g \in S^h  \; \deg h = \deg g \st \; \forall x \in V' \; f(x) = \f{g(x)}{h(x)}
  \]
  が成り立つ。このとき
  \begin{align*}
    \forall y \in \vp(V') \quad f \circ \vp^{-1}(y) &= f(1,y_1, \cdots , y_n) \\
    &= \f{\gra h(y)}{\gra g(y)}
  \end{align*}
  である。したがって、$f \circ \vp^{-1}$は$p$で正則であり、$p \in \vp(V)$は任意だったから$f \circ \vp^{-1}$は正則である。
\end{proof}



\bfsubsection{命題3.3 直後}
\barquo{
$S$が次数付き環で$\frakp$が$S$の斉次素イデアルのとき、$S$の斉次元のうち$\frakp$に含まれないもののなす乗法的部分集合$T$に関して$S$を局所化し、その中で0次の元のなす部分環を$S_{(\frakp)}$と書く。
}
\begin{rem}
  なぜ$\frakp$は斉次イデアルでなければならないのだろうか? 斉次イデアルでなくても素イデアルであれば乗法的になる。斉次元との共通部分をとっている理由は、局所化が再び次数付き環になるようにするためだろうと思われるが。次数環についての一般論を踏まえているのだろうか。
\end{rem}



\bfsubsection{命題3.3 直後}
\barquo{
 $T^{-1}S$には次のような自然な次数付けがあることに注意せよ: $S$の斉次元$f$と$g \in T$に対して$\deg (f/g) =\deg(f) - \deg (g)$.
}
\begin{rem}
  この定義は同値類の取り方によらず、well-definedである。$f/g=f'/g'$であったとしよう。このとき
  \[
  \exists h \in T \quad h(fg' - f'g) = 0
  \]
  である。$0$の次数はマイナス無限大であることに注意して変形していくと
  \begin{align*}
    hfg' &= hf'g \\
    \deg(hfg') &= \deg(hf'g) \\
    \deg(h) + \deg(f) + \deg(g') &= \deg(h) + \deg(f') + \deg(g) \\
    \deg(f) - \deg(g) &= \deg(f') - \deg(g')
  \end{align*}
  を得る。よってwell-defined性がいえた。
\end{rem}



\bfsubsection{命題3.3 直後}
\barquo{
$S_{(\frakp)}$は局所環で、極大イデアルは$(\frakp \cdot T^{-1}S) \cap S_{(\frakp)}$である。
}
\begin{proof}
  $x/t \in S_{(\frakp)} \setminus (\frakp \cdot T^{-1}S)$とする。このとき$x \notin \frakp$であり、かつ$x/t \in S_{(\frakp)}$より$x$は斉次元である。よって$x \in T$なので$x/t$は単元である。よって局所環であることがいえた。
\end{proof}



\bfsubsection{定理 3.4}
\barquo{$Y \subset \P^n$を射影多様体、その斉次座標環を$S(Y)$とする。このとき:
\begin{description}
\item[(a)] $\calo(Y) = k$.
\item[(b)] $P \in Y$について$\frakm_P \subset S(Y)$を$f(P)=0$となる斉次元$f \in S(Y)$の集合で生成されるイデアルとする。このとき$\calo_P = S(Y)_{(\frakm_P)}$.
\item[(c)] $K(Y) \cong S(Y)_{((0))}$.
\end{description}
}
\begin{proof}
  まず$U_i \subset \P^n$を開集合$x_i \neq 0$とし、$Y_i = \vp(Y \cap U_i)$とする。$Y_i$はアファイン多様体である。$A = k[x_1, \cdots , x_n]$と$S_{(x_i)}$の同型を
\[
\vp^*_i \colon A \to S_{(x_i)} \st \vp^*_i(f)(x_0, \cdots , x_n) = f(x_0/x_i, \cdots , \widehat{x_i/x_i} , \cdots , x_n/x_i)
\]
によって与える。($\vp$は既に使ったので本来は他の記号を使うべきだが、ここでは簡単のために同じ記号とした) このとき
\[
\vp_i^*(I(Y_i)) = (I(Y)\cdot S_{x_i} ) \cap S_{(x_i)}
\]
が成り立つ。なぜなら、$g \in S_{(x_i)}$に対して
\begin{align*}
  g \in \vp_i^*(I(Y_i)) &\iff \exists f \in I(Y_i) \st g = \vp^*_i f \\
  &\iff g(x_0, \cdots , x_n) =  f(x_0/x_i, \cdots , \widehat{x_i/x_i} , \cdots , x_n/x_i) \\
  &\iff \exists d \geq 0 \st x_i^d g(x) \in I(Y) \cap S^h \\
  &\iff g \in (I(Y)\cdot S_{x_i} ) \cap S_{(x_i)}
\end{align*}
が成り立つからである。

ゆえに商に移ると同型
\begin{align*}
  A(Y_i) &= A/I(Y_i) \\
  &\cong S_{(x_i)} / (I(Y)\cdot S_{x_i} ) \cap S_{(x_i)}
\end{align*}
を得る。これは次数付き環$S_{x_i}$の0次成分を、斉次イデアル$I(Y) \cdot S_{x_i}$の0次斉次成分で割ったものである。一般に次が成り立つ。
 \lem{
$S$が次数付き環$S = \bigoplus_{d \in \Z} S_d$で、$I \subset S$が斉次イデアルだとする。このとき
\[
(S/I)_0 = S_0/I \cap S_0
\]
が成り立つ。
}
\begin{proof}
  $S/I$は$S/I = \bigoplus_{d \in \Z} S_d/I \cap S_d$によって次数つき環だと見なされることから、あきらか。
\end{proof}
定理3.4の証明に戻る。したがって、次のように変形することができる。
\begin{align*}
  A(Y_i) &\cong (S_{x_i}/ I(Y)S_{x_i})_0 \\
  &\cong (S/I(Y) \otimes_S S_{x_i})_0 \\
  &\cong S(Y)_{(x_i)}
\end{align*}
同型は$\vp^*_i$によって誘導されていることに注意する。

このことを踏まえて、以下証明を行っていく。
\begin{description}
\item[(b)] 任意に点$P \in Y$が与えられたとする。$P \in U_i$なる$i$がある。
\[
\frakm'_P = \setmid{f \in A(Y_i)}{f(\vp_i(P))= 0}
\]
とおく。このとき
\[
\vp^*_i(\frakm'_P) = (\frakm_P \cdot S(Y)_{x_i}) \cap S(Y)_{(x_i)}
\]
が成り立つ。なぜなら、$g \in S(Y)_{(x_i)}$に対して
\begin{align*}
  \ol{g} \in \vp^*_i(\frakm'_P) &\iff \exists f \in \frakm'_P \st g = \vp^*_i f \\
  &\iff g(x_0, \cdots , x_n ) = f(x_0/x_i, \cdots , \widehat{x_i/x_i} , \cdots , x_n/x_i) \\
  &\iff  \exists d \geq 0 \st x_i^d g(x) \in I(\{ P \} ) \cap S^h = \wt{I}(\{ P \}) \\
  &\iff g \in (I(\{P\}) \cdot S_{x_i} ) \cap S_{(x_i)} \\
  &\iff \ol{g} \in (\frakm_P \cdot S(Y)_{x_i}) \cap S(Y)_{x_i}
\end{align*}
であるからである。したがって
\[
A(Y_i)_{\frakm'_P} \cong (S(Y)_{(x_i)})_{\vp^*_i(\frakm'_P)} \cong (S(Y)_{(x_i)})_{(\frakm_P \cdot S(Y)_{x_i}) \cap S(Y)_{(x_i)}}
\]
が成り立つ。

実は、右辺のややこしい式は次のように表せる。
\[
(S(Y)_{(x_i)})_{(\frakm_P \cdot S(Y)_{x_i}) \cap S(Y)_{(x_i)}} \cong S(Y)_{(\frakm_P)}
\]
局所化の普遍性によりこれを示そう。$x_i \notin \frakm_P$なので、自然な準同形$S(Y)_{(x_i)} \to S(Y)_{(\frakm_P)}$が誘導される。いま、環$M$と環準同形$f \colon S(Y)_{(x_i)} \to M$であって、$f((\frakm_P \cdot S(Y)_{x_i}) \cap S(Y)_{(x_i)}) \subset M^{\tm}$なるものが与えられたとする。
\[
\xymatrix{
S(Y)_{(x_i)} \ar[r] \ar[dr]_f & S(Y)_{(\frakm_P)} \ar[d]^{\wt{f}} \\
{} & M
}
\]
このとき$f$の延長$\wt{f}$が、
\begin{align*}
\wt{f}(x/s) &= \wt{f}(x/x_i^N \cdot x_i^N/s) \\
&= f(x/x_i^N) f(s/x_i^N)^{-1}
\end{align*}
により一意的に定まる。よって局所化の普遍性により同型
\[
(S(Y)_{(x_i)})_{(\frakm_P \cdot S(Y)_{x_i}) \cap S(Y)_{(x_i)}} \cong S(Y)_{(\frakm_P)}
\]
がいえる。

定理3.2により$\calo_{\vp_i(P)} \cong A(Y_i )_{\frakm'_P}$が成り立つことが判るので、したがって求める同型$S(Y)_{(\frakm_P)} \cong \calo_{\vp_i(P)} \cong \calo_P$がいえたことになる。

\item[(c)] まず次のことに注意する。
 \lem{
  $Y$を多様体とし、$\emptyset \subsetneq U \opsub Y$とする。このとき$K(U)=K(Y)$である。
}
\begin{proof}
  あきらかに$K(U) \subset K(Y)$が成り立つ。逆に、$\langle V, f \rangle \in K(Y)$だとする。$Y$は既約な位相空間なので、$V \cap U$は空でない。ゆえに、$y \in V \cap U$なる$y$が存在する。$f$は$y$において正則なので、ある$y$の近傍$W \opsub V$が存在して、$f$は$W$上では斉次多項式による有理式$h/g$に等しい。このとき$\langle V, f \rangle  = \langle W \cap U , h/g \rangle \in K(U)$であるから、$\langle V, f \rangle \in K(U)$
  がいえた。
\end{proof}

定理3.4(c)の証明に戻る。よって、$K(Y) = K(Y \cap U_i) \cong K(Y_i)$が成り立つことが判る。定理3.2により$K(Y_i)$は$A(Y_i)$の商体である。よって同型$\Frac A(Y_i) \cong \Frac S(Y)_{(x_i)} \cong S(Y)_{((0))}$が判る。したがって$K(Y) \cong S(Y)_{((0))}$である。

\item[(a)] ここまでくれば本文通りであるので省略する。

\end{description}
\end{proof}




\bfsubsection{命題 3.5}
\barquo{
よって$\psi$は$X$から$Y$への写像を定めるが、これは与えられた準同形$h$を引き起こすものである。
}
\begin{rem}
  本文で与えられている、$h \colon A(Y) \to \calo(X)$に対して$\psi \colon X \to Y$を対応させる写像を$\beta \colon \Hom(A(Y), \calo(X)) \to \Hom (X,Y)$と書くことにする。ここで示すべきことは$\alpha \beta = id$と$\beta \gra = id$である。

  $\alpha \beta = id$であること: $h \in \Hom(X,Y)$が与えられたとし、$\psi = \beta h$であるとする。$\xi_i$も本文で与えられたものそのままとする。このとき$\alpha \psi = h$であるとは、次の図式が可換であることを意味する。
  \[
  \xymatrix{
  \calo(Y) \ar[r]^{\psi^*} & \calo(X) \\
  A(Y) \ar[u]^i \ar[ur]_h & {}
  }
  \]
  いま$\ol{f} \in A(Y)$と$x \in X$が与えられたとする。このとき
  \begin{align*}
    \psi^* \circ i(\ol{f})(x) &= (f \circ \psi )(x) \\
    &= f(\xi_1(x), \cdots , \xi_n(x)) \\
    &= f(h(\ol{x_1})(x), \cdots , h(\ol{x_n})(x)  ) \\
    &= h(\ol{f})(x)
  \end{align*}
  が成り立つ。よって示すべきことがいえた。

  $\beta \gra = id$であること: 射$\vp \colon X \to Y$と$P \in X$が与えられたとする。このとき
  \begin{align*}
    \beta \gra \vp(P) &= (\gra \vp(\ol{x_1})(P), \cdots ,  \gra \vp(\ol{x_n})(P)) \\
    &= ((x_1 \circ \vp)(P), \cdots , (x_n \circ \vp)(P) ) \\
    &= \vp(P)
  \end{align*}
  である。よって$\beta \gra = id$がいえた。
\end{rem}


\bfsubsection{補題 3.6}
\barquo{
$X$を任意の多様体とし、$Y \subset \A^n$をアファイン多様体とする。
}
\begin{rem}
  同様の証明を適用することにより、\textblue{$Y \subset \A^n$が準アファイン多様体でも同じ結論がいえる。}
\end{rem}


\bfsubsection{補題 3.6}
\barquo{
$Y$の閉集合は多項式関数の零点集合として定義され、また正則函数は連続であるから、$\psi^{-1}$は閉集合を閉集合に写すことが分かり、よって$\psi$は連続である。
}
\begin{proof}
  $Z(I) \clsub Y$が与えられたとする。このとき
  \begin{align*}
    \psi^{-1}(Z(I)) &= \psi^{-1}(\bigcap_{g \in I} Z(g)) \\
    &= \bigcap_{g \in I} \psi^{-1}(Z(g)) \\
    &= \bigcap_{g \in I} \setmid{x \in X}{(g \circ \psi )(x) = 0}
  \end{align*}
  が成り立つ。$g \circ \psi$は正則なのでとくに連続であり、したがって$\psi$は連続であることが判る。
\end{proof}






\bfsubsection{系 3.7}
\barquo{
$X$, $Y$を2つのアファイン多様体とすると, $A(X)$と$A(Y)$が$k$代数として同型であるとき, またそのときに限って$X$と$Y$は同型である.
}
\begin{proof}
  命題3.5により、多様体の圏から$k$上の有限生成整域の圏への反変関手$Y \mapsto A(Y)$は忠実充満である。忠実充満関手については次の事実がある。
   \lem{
    $\calc$、$\cald$が圏であり、$F \colon \calc \to \cald$が忠実充満関手であるとする。$X, Y \in \calc$とするとき次は同値。
    \begin{enumerate}
      \item $X \cong Y$
      \item $FX \cong FY$
    \end{enumerate}
  }
  \begin{proof}
    1. $\Rightarrow$ 2. は$F$の関手性からあきらか。逆に、$FX \cong FY$であるとしよう。このとき次の図式を可換にする$\cald$での射$f,g$がある。
    \[
    \xymatrix{
    FX \ar[r]^f & FY \\
    FY \ar[u]^g \ar[ur]_{id} & {}
    }
    \]
    $F$が充満関手であることにより、ある射$f',g'$が存在して$Ff'=f$、$Fg'=g$を満たす。このとき$F(f' \circ g')= f \circ g = id_{FY} = F(id_Y)$が成り立つ。ゆえに$F$が忠実であることから、$f' \circ g' = id_Y$である。あとは同様に議論を続ければ$X \cong Y$がいえる。
  \end{proof}

この補題から、系が成り立つことがいえるのはあたりまえである。
\end{proof}



\bfsubsection{系 3.8}
\barquo{
関手$X \mapsto A(X)$は、$k$上のアファイン多様体の圏と$k$上の有限生成整域の圏の間に、矢印の向きを逆にする圏同値を引き起こす。
}
\begin{proof}
  強い選択公理を認めることにすると、関手が圏同値であることと、忠実充満かつ本質的全射であることは同値である。そこで$X \mapsto A(X)$が本質的全射であることを示せばよい。

  $R$を$k$上の有限生成整域であるとする。このとき多項式環からの全射$k[x_1, \cdots , x_n] \to R$が存在するので、あるイデアル$\frakp$が存在して$R \cong k[x_1, \cdots , x_n] / \frakp$と表せる。$R$は整域なので$\frakp$は素イデアルである。このとき$X = Z(\frakp) \subset \A^n$とすると$\frakp$が素イデアルなので$X$はアファイン多様体であり、$A(X) \cong R$を満たす。よって本質的全射であることがいえた。
\end{proof}


\bfsubsection{演習問題 3.2}
\begin{description}
  \item[(a)] $Y = Z(x^3-y^2)$とする。$\vp \colon k[x,y] \to k[t]$を$\vp(x)=t^2$、$\vp(y)=t^3$で定めると、$\Ker \vp = (x^3 - y^2)$であるので$(x^3 - y^2)$は素イデアルである。したがって$Y$はアファイン多様体である。

$(x,y) \neq 0$のとき$(x,y) \mapsto y/x$、そして$0$は$0$に写すという写像が逆写像を定めるので、あきらかに$\vp$は全単射。$\vp$は$\A^1$からの正則写像$t \mapsto t^2$と$t \mapsto t^3$の組なので、補題3.6により$\A^1$から$\A^2$への射を定める。$\A^1$の閉集合は全体もしくは有限集合であり、したがって$\vp$は閉写像でもあるので同相であると結論できる。

しかし$A(Y) \cong k[x,y]/(x^3-y^2) \cong k[t^2,t^3]$であり、$\vp^*$は全射でない。したがってとくに同型ではないので、系3.7により$Y$と$\A^1$は同型ではない。

\item[(b)] $k$は代数閉体なので$\vp$は全射。また$k$の標数は$p$なので$\vp$は単射。また多様体の射であるから連続性が従う。$\A^1$の閉集合は全体でなければ有限集合なので閉写像でもあり、同相であることがわかる。

しかし座標環の間に誘導される準同形$\vp^*$は全射ではないので、$\vp$は同型でない。
\end{description}



\bfsubsection{演習問題 3.4}
$\rho_d$が像への同相写像であることは既に示したので、$\rho_d$とその逆写像が射になっていることを示せばよい。

$\rho_d$が射であること:
  \[
  \xymatrix{
  \P^n \ar[r]^{\rho_d} & Z(\fraka) \\
  \rho_d^{-1}(V) \ar[u] \ar[r]^{\rho_d} \ar[dr]_{f \circ \rho_d} & V \ar[u] \ar[d]^f \\
  {} & k
  }
  \]
  任意の$p \in \rho_d^{-1}(V)$に対する、$\rho_d(p) \in V$における$f$の正則性から判る。ただし、斉次式の各変数に$d$次単項式を代入してもやはり斉次式であることに注意する。

  $\rho_d^{-1}$が射であること:
  \[
  \xymatrix{
  \P^n  & Z(\fraka) \ar[l]_{\rho_d^{-1}} \\
  W  \ar[u] \ar[d]_f   &  \rho_d(W) \ar[u] \ar[l]_{\rho_d^{-1}}  \ar[dl]^{f \circ \rho_d^{-1}} \\
  k &  {}
  }
  \]
  $q \in \rho_d(W)$が与えられたとする。$q = \rho_d(p)$なる$p \in W$がある。$p$における$f$の正則性により
  \[
  \exists W' \; p \in W' \opsub W \; \exists h,g \in S^h \st \forall P \in W' \; f(P) = \f{h(P)}{g(P)}
  \]
  である。ここである$l \geq 0$であって、$\deg (x_i^lh ) = \deg (x_i^lg )$が$d$の倍数であるようなものを選ぶことができる。すると$x_i^lh = \grt (h')$, $x_i^lg = \grt (g')$なる$h', g' \in T^h$をとることができる。したがって任意の$Q \in \rho_d(W' \cap U_i)$について
  \begin{align*}
    (f \circ \rho_d^{-1})(Q) &= \f{ h(\rho_d^{-1}(Q)) }{ g(\rho_d^{-1}(Q)) } \\
    &=  \f{ (x_i^l h)(\rho_d^{-1}(Q)) }{ (x_i^l g)(\rho_d^{-1}(Q)) } \\
    &= \f{ (\grt h')(\rho_d^{-1}(Q)) }{ (\grt g')(\rho_d^{-1}(Q)) } \\
    &= \f{ h'(Q) }{ g'(Q) }
  \end{align*}
  だから正則性がいえた。





\bfsubsection{演習問題 3.9}
2次単項式の全体に
\[
M_0 = x^2 \quad M_1 = xy \quad M_2 = y^2
\]
として番号を振っておく。このとき
\[
Y = \rho_2(\P^1) = \setmid{[s^2 : st : t^2 ] }{[s : t] \in \P^1}
\]
である。


$I(Y) = (xz - y^2)$であること: $I(Y) \supset (xz - y^2)$はあきらかであるので、逆を示す。$f \in I(Y)$なる斉次元が与えられたとする。$y^2 - xz$で割って、
  \[
  f(x,y,z )=(y^2 - xz)g(x,y,z) + yf_1(x,z) + f_2(x,z)
  \]
  なる$g, f_1,f_2$をとることができる。$f \in I(Y)$より
  \[
  \forall s,t \in k \; \; 0 = stf_1(s^2,t^2) + f_2(s^2, t^2)
  \]
  である。$s$について偶数次か奇数次かで分けることにより、$f_1 = f_2 = 0$が結論できる。よって示せた。

$S(X) \cong S(Y)$でないこと: $S(X) =k[x,y]$, $S(Y) = k[x,y,z]/(y^2-xz)$である。$\frakm = (x,y,z)/(y^2-xz)$とする。このとき$(y^2-xz)$が2次式なので$\frakm^2 = ((x,y,z)^2 + (y^2-xz) )/(y^2-xz) = (x,y,z)^2 /(y^2-xz)$である。よって
  \[
  \frakm / \frakm^2 = kx \oplus xy \oplus kz
  \]
  となり、$k$ベクトル空間としての次元は$3$である。一方で、$k[x,y]$のどんな極大イデアル$\frakm$に対しても$\dim_k \frakm / \frakm^2$は$2$であるので、同型でないことがいえた。

  なお、別証明として
  \begin{align*}
      S(Y) &= k[x,y,z] / I(Y) \\
      &= k[x,y,z] / I(Z(\Ker \grt)) \\
      &= k[x,y,z] / \Ker \grt \\
      &\cong \Im \grt \\
      &= k[x^2,y^2,xy]
  \end{align*}
  を利用してもよい。







\bfsubsection{演習問題 3.10}
まず次の補題を示す。

  \lem{
    $X$は位相空間で$W \subset X$であるものとする。このとき次は同値。
    \begin{description}
      \item[(1)] $W \subset X$は局所閉、すなわち$W \opsub \ol{W}$が成り立つ。
      \item[(2)] $W = U \cap V$なる$U \opsub X$と$V \clsub X$がある。
    \end{description}
  }

\begin{proof}
  順に示す。
  \begin{description}
    \item[(1)$\To$(2)] あきらか。
    \item[(2)$\To$(1)] $\ol{W} \subset V $であるから、
    $W \subset \ol{W} \cap U \subset V \cap U = W$
    より$W = \ol{W} \cap U$が成立する。
  \end{description}
\end{proof}




  \lem{
    多様体$X,Y$があり、$X \subset Y$であるとし、$p \in X$とする。このとき包含写像$X \to Y$を$i$とすると
    \[
    i^* \colon \calo_{p,Y} \to \calo_{p,X}
    \]
    は全射。
  }

\begin{proof}
  $\kakko{U,f} \in \calo_{p,X}$が与えられたとする。このときある$U' \opsub U$と多項式$h,g$が存在して、任意の$x \in U' \subset U$について
  \[
  f(x) = \f{h(x)}{g(x)}
  \]
  が成り立つ。とくに分母の$g$は$U'$上$0$でない。ここで$V = \setmid{x \in Y}{g(x) \neq 0}$とおくと$i^*(\kakko{V, \f{h}{g} }) = \kakko{U,f}$が成り立つので全射であることがいえた。
\end{proof}

演習問題3.10の証明のつづき。
\[
\xymatrix{
X' \ar[r]^{\vp|_{X'}} \ar[d]_{i_{X'}} & Y' \ar[d]^{i_{Y'}} \\
X \ar[r]^{\vp} & Y
}
\]
$p \in X'$と$f \in \calo_{\vp(p), Y'}$が与えられたとする。このとき$i^*_{Y'} (\wt{f}) = f$なる$\wt{f} \in \calo_{\vp(p), Y}$がある。このとき$(\vp \circ i_{X'})^*(\wt{f}) \in \calo_{p,X'}$であって、
\begin{align*}
(\vp \circ i_{X'})^*(\wt{f}) &= \wt{f} \circ \vp \circ i_{X'} \\
&= \wt{f} \circ i_{Y'} \circ \vp|_{X'} \\
&= f \circ \vp|_{X'}
\end{align*}
であるから$\vp|_{X'}$は射である。


\bfsubsection{演習問題 3.12}
\begin{description}
  \item[Step 1] $X$がアファイン多様体の場合は命題3.2(c)によりよい。
  \item[Step 2] $X$が一般の多様体の場合に示そう。次の補題をまず示す。
  \lem{
  多様体$X$と、その点$P \in X$が与えられたとする。このとき、開アファイン集合$U \subset X$であって、$P$の開近傍であり、かつ$\dim U = \dim X$なるものが存在する。とくに$\trdeg_k K(X) = \dim X$である。
  }
  \begin{proof}
    命題4.3により、開アファイン部分集合$U_i \subset X$による$X$の被覆$X = \bigcup_i U_i$がある。$U_i$はそれぞれアファイン多様体$Z_i$と同型であるとする。このときすべての$i$に対して
    \begin{align*}
      \trdeg_k K(X) &= \trdeg_k K(U_i) \\
      &= \trdeg_k K(Z_i) \\
      &= \dim A(Z_i) \\
      &= \dim Z_i \\
      &= \dim U_i
    \end{align*}
    だから$\dim U_i$はすべて等しい。$U_i$は$X$の開被覆を与えているので、演習問題1.10(b)により、すべての$i$に対して
    \[
    \dim X = \dim U_i
    \]
    が成り立つ。そこで、$P \in U_j$なる$U_j$を選べば、それが求めるアファイン集合である。
  \end{proof}
  演習問題3.12の証明に戻る。補題により、$P$の近傍$U \opsub X$であって、アファイン多様体と同型でかつ$\dim X = \dim U$なるものをとれる。
  すると
  \begin{align*}
    \dim \calo_{P,X} &= \dim \calo_{P,U} \\
    &= \dim U &(\text{Step 1による}) \\
    &= \dim X
  \end{align*}
  であるから、示すべきことがいえたことになる。
\end{description}


\bfsubsection{演習問題 3.15}
\barquo{
$X \subset \A^n$および$Y \subset \A^m$をアファイン多様体とする。
}
\begin{rem}
  補題3.6と同様、これも\textblue{(a)と(c)はアファイン多様体でなく準アファイン多様体で十分である}。
\end{rem}




\bfsubsection{演習問題 3.15}
\begin{description}
  \item[(a)] 以下とくに断らない限り集合$X \tm Y$の位相として、相対位相$X \tm Y \subset \A^{n+m}$を考えることにする。まず次の補題を用意しておく。

  \lem{
  (既約空間の連続像は既約) \\
  $X,Y$は位相空間で、$U \subset X$は既約であるとし、連続写像$f \colon X \to Y$が与えられているとする。このとき$f(U) \subset Y$は既約。
  }
  \begin{proof}
    $f(U) \subset V_1 \cup V_2$なる$V_i \clsub Y$が与えられたとする。このとき$U \subset f^{-1}(V_1) \cup f^{-1}(V_2)$で、$f$の連続性から$f^{-1}(V_1) \clsub X$かつ$f^{-1}(V_2) \clsub X$であるから、$U \subset X$の既約性により$U \subset f^{-1}(V_i) $なる$i$がある。このとき$f(U) \subset V_i$
    である。したがって既約性がいえた。
  \end{proof}

  演習問題3.15(a)の証明に戻る。

  $X \tm Y \clsub \A^{n+m}$であることなど: 射影$p_1 \colon \A^{n+m} \to \A^n$, $p_2 \colon \A^{n+m} \to \A^m$は多項式で定められた写像なので射である。$X \tm Y = p_1^{-1}(X) \cap p_2^{-1}(Y)$であるから$X$と$Y$がアファイン多様体なら$X \tm Y \clsub \A^{n+m}$
  であるし、準アファイン多様体なら$X \tm Y \loc \A^{n+m}$が成り立つことがわかる。。

  $X \tm Y$が既約であること: 既約性を示す以下の議論では$X$と$Y$は単に準アファイン多様体と仮定して話を進める。$X \tm Y \subset Z_1 \cup Z_2$なる$Z_i \clsub \A^{n+m}$が与えられたとする。
  \[
  X_i = \setmid{x \in X}{\{ x \} \tm Y \subset Z_i}
  \]
  とおく。ここで$x \in X$とすると、$\{ x \} \tm Y \subset Z_1 \cup Z_2$である。$x \in X$に対して$S_x \colon Y \to \A^{n+m}$を$y \mapsto (x,y)$で定めると、これは多項式で表される写像なので射。したがって連続であり、ゆえに既約空間の連続像は既約なので$\{ x \} \tm Y$は既約。(注意: $\{ x \} \tm Y$に入っている位相は積位相ではなく相対位相なので、既約性は自明ではないと考えた) よって$x$ごとに$\{ x \} \tm Y \subset Z_j$なる$j$がある。したがって$x \in X$
  は任意だったから$X = X_1 \cup X_2$である。$y \in Y$に対して$S_y \colon X \to \A^{n+m}$を$S_y(x)=(x,y)$で定めると、これは多項式で表される写像なのでもちろん射。とくに連続であり、$X_i = \bigcap_{y \in Y} S_y^{-1}(Z_i)$より$X_i \clsub X$がいえる。よって、$X$の既約性により$X = X_k$なる$k$が存在する。このとき$X \tm Y \subset Z_k$となる。ゆえに$X \tm Y$は既約であり、アファイン空間の部分多様体とみなせる。
  \item[(b)] 射影$X \tm Y \to X$と$X \tm Y \to Y$は、多項式で表される写像なので射である。座標環をとる操作は関手なので、環準同型$A(X) \to A(X \tm Y)$, $A(Y) \to A(X \tm Y)$が誘導される。これはまた環準同形$\grt \colon A(X) \ts_k A(Y) \to A(X \tm Y)$を誘導する。$\grt$は$\grt(f \ts g)(P,Q) = f(P)g(Q)$なる写像である。$\grt$は座標関数$\ol{x_1}, \cdots , \ol{y_m}$
  を$X \tm Y$上の座標関数に送るので、全射である。

  $\grt$が単射であることを示そう。単射でなかったと仮定する。$0$でない$\Ker \grt$の元$\sum_{i=1}^l f_i \ts g_i$のうち、$l$が最小なものがとれる。$A(X \tm Y)$は整域なので、$l \geq 2$としてよい。$X$上の正則関数として$f_l \neq 0$なので、$f_l(P) \neq 0$なる$P \in X$がある。このとき任意の$Q \in Y$に対して$\sum f_i(P)g_i(Q) = 0$が成り立つ。したがって
  $g_l = - f_l(P)^{-1} \sum f_i(P)g_i(Q)$と表せる。この関係式を使うと
  \[
  \sum_{i=1}^l f_i \ts g_i = \sum_{i=1}^{l-1} (f_i - f_l(P)^{-1} f_i(P)f_l) \ts g_i
  \]
  と、より短く表せるので$l$の最小性に矛盾。ゆえに$\grt$は単射で、よって同型。
  \item[(c)] 多様体の圏における直積の図式
  \[
  \xymatrix{
  X & X \tm Y \ar[l]_{\pi_1} \ar[r]^{\pi_2} & Y \\
  {} & Z \ar[ul]^{\vp} \ar[ur]_{\psi} \ar[u]^{\grs} & {}
  }
  \]
  を満たすことを示そう。まず射影$\pi_1$, $\pi_2$は多項式関数で定義されているので、準アファインの場合の補題3.6により射である。任意の多様体$Z$と射$\vp$, $\psi$が与えられたとすると、$\grs(P) = (\vp(P), \psi(P))$とすれば上の図式を可換にする。$\vp$, $\psi$が射なので$\grs$も射である。一意性はあきらかなので、直積であることがいえた。
  \item[(d)] 座標環をとっても次元は不変なので、$\dim(A(X \tm Y)) = \dim A(X) + \dim A(Y)$を示せば十分である。Noetherの正規化定理により、
  ある$k[t_1, \cdots , t_u]$と$k[s_1, \cdots , s_v]$が存在して、$A(X)$と$A(Y)$はそれぞれ$k[t_1, \cdots , t_u]$, $k[s_1, \cdots , s_v]$上整である。このとき$a \in A(X)$に対して、$a^r + \gra_{r-1}a^{r-1} + \cdots + \gra_0 = 0$なる$\gra_i \in k[t_1, \cdots , t_u]$
  がとれる。よって$b \in A(Y)$について
  \[
  (a \ts b)^r + (\gra_{r-1} \ts b)(a \ts b)^{r-1} + \cdots + (\gra_0 \ts b^r) = 0
  \]
  が成り立つことになり、$a \ts b$は$k[t_1, \cdots , t_u] \ts A(Y)$上整である。すなわち$A(X) \ts A(Y)$は$k[t_1, \cdots , t_u] \ts A(Y)$上整である。同様にして$k[t_1, \cdots , t_u] \ts A(Y)$は$k[t_1, \cdots , t_u] \ts k[s_1, \cdots , s_v]$上整であり、推移性から結局$A(X) \ts A(Y)$が$k[t_1, \cdots , t_u] \ts k[s_1, \cdots , s_v]$
  上整であることがいえる。ところが$k[t_1, \cdots , t_u] \ts k[s_1, \cdots , s_v]$は$u +v$変数多項式環と同型であるため、$A(X) \ts A(Y)$の商体は$k$の純超越拡大$k(t_1,  \cdots , t_u , s_1, \cdots , s_v)$上代数的であることになる。したがってその$k$上の超越次数は$u+v$であり、ゆえに$\dim A(X) \ts A(Y) = \dim A(X) + \dim A(Y)$がいえた。
\end{description}







\bfsubsection{演習問題 3.16}
\begin{description}
    \item[(a)] まず補題を示す。
\lem{
(射影が射であること) \\
Segre埋め込み$\psi \colon \P^n \tm \P^m \to \P^N$による$\P^n \tm \P^m$の像を$W$とおく。$\pi_i \; (i=1,2)$を$W$から$\P^n$, $\P^m$への射影とする。つまり、次の図式が可換になるような写像とする。
\[
\xymatrix{
\P^n & W \ar[l]_-{\pi_1} \ar[r]^-{\pi_2} & \P^m \\
{} & \P^n \tm \P^m \ar[u]_-{\psi} \ar[ul]^-{\wt{\pi_1} } \ar[ur]_-{\wt{\pi_2 }} & {}
}
\]
ただし$\wt{\pi_i}$は積集合からの自然な射影である。$\pi_i$がwell-definedであることはあきらかだろう。

このとき、$\pi_1$, $\pi_2$は多様体の圏における射である。
}

\begin{proof}
  証明のアイディアは補題3.6の応用で、「局所的に多項式だから射」というだけである。が、初めてなので丁寧に示そう。同じことなので、$\pi_1$についてだけ証明する。

  $U_i \opsub \P^n$, $V_j\opsub \P^m$, $W_{ij} \opsub \P^N$を標準的な開部分集合とする。$\vp_i \colon U_i \to \A^n$を自然な同型とする。このとき
  \[
  \psi(U_i \tm V_j) = W_{ij} \cap W
  \]
  が成り立つ。よって
  \begin{align*}
    \pi_1(W_{ij} \cap W) &= \pi_1 \circ \psi(U_i \tm V_j) \\
    &= \wt{\pi_1} (U_i \tm V_j ) \\
    &= U_i
  \end{align*}
  である。

  ここで合成$\vp_i \circ \pi_1|_{W_{ij} \cap W}$は
  \begin{align*}
    \vp_i \circ \pi_1|_{W_{ij} \cap W} ((z_{ij})_{i,j}) &= \vp_i \circ \wt{\pi_1} \circ \psi^{-1} ((z_{ij})_{i,j}) \\
    &= \vp_i \circ \wt{\pi_1} ((z_{0j}, \cdots , z_{nj}), (z_{i0}, \cdots , z_{im})) \\
    &= \vp_i (z_{0j}, \cdots , z_{nj}) \\
    &= \left( \f{z_{0j}}{z_{ij}}, \cdots ,  \widehat{\f{z_{ij}}{z_{ij}} }, \cdots , \f{z_{nj}}{z_{ij}} \right)
  \end{align*}
  なる写像である。ここで、補題3.6を適用して、$\vp_i \circ \pi_1|_{W_{ij} \cap W}$は射。したがって同型を外して、$\pi_1|_{W_{ij} \cap W} \colon W_{ij} \cap W \to U_i $は射。包含写像は射なので、包含射との合成$\pi_1|_{W_{ij} \cap W} \colon W_{ij} \cap W \to \P^n$
  も射である。開被覆を適当にとればその上に制限したときに射になるということなので、$\pi_1 \colon W \to \P^n$そのものも射。
\end{proof}



\begin{proof}
  演習3.16(a)の証明に戻る。Segre埋め込み$\P^n \tm \P^m \to \P^N$を$\psi$で書き、$Z = \psi(\P^n \tm \P^m)$、$W = \psi(X \tm Y)$とする。

  $W \loc Z$であること: 補題により次の可換図式
  \[
  \xymatrix{
  \P^n & Z \ar[l]_-{\pi_1} \ar[r]^-{\pi_2} & \P^m \\
  {} & \P^n \tm \P^m \ar[lu]^-{\wt{\pi_1}} \ar[ru]_-{\wt{\pi_2}} \ar[u]_-{\psi} & {}
  }
  \]
  により定まる射影$\pi_i$は射である。このとき
  \begin{align*}
    z \in \pi_1^{-1}(X) \cap \pi_2^{-1}(Y) &\iff \pi_1(z) \in X \; \text{かつ} \; \pi_2(z) \in Y \\
    &\iff \psi^{-1}(z) \in X \tm Y \\
    &\iff z \in W
  \end{align*}
  により$W = \pi_1^{-1}(X) \cap \pi_2^{-1}(Y)$である。仮定により$X \loc \P^n$、$Y \loc \P^m$なので、$\pi_i$の連続性から$\pi_1^{-1}(X) \loc Z$かつ$\pi_2^{-1}(Y) \loc Z$であることがわかる。ゆえに$W \loc Z$である。

  $W$が既約であること: $W \subset C_1 \cup C_2$なる$C_i \clsub Z$が与えられたとする。$y \in Y$に対して$\vp_y \colon X \to Z$を$\vp_y(x)=\psi(x,y)$で定める。$X_y = \Im (\vp_y)$とおく。
  $\vp_y \colon X \to Z$は多項式で定義された写像なので、補題3.6を適用すれば射であることが示せる。したがって$X_y$は既約空間の連続像なので既約。このとき$y \in Y$を固定するごとに$X_y \subset W \subset C_1 \cup C_2$なので、$X_y \subset Z$の既約性から$X_y \subset C_{i(y)}$なる$i(y)$がある。すなわち、$Y_j = \setmid{y \in Y}{X_y \subset C_j}$とおけば$Y = Y_1 \cup Y_2$である。
  $x \in X$に対して$\vp_x \colon Y \to Z$を$\vp_x(y)=\psi(x,y)$で定めると$\vp_y$と同様$\vp_x$も射である。したがって
  \begin{align*}
    Y_j &= \setmid{y \in Y}{X_y \subset C_j} \\
    &=  \setmid{y \in Y}{\forall x \in X \; \;  \psi(x,y) \in C_j} \\
    &= \bigcap_{x \in X} \vp_x^{-1}(C_j)
  \end{align*}
により$Y_j \clsub Y$がわかる。仮定より$Y$は既約なので$Y = Y_i$なる$i$があり、したがってこの$i$について$W \subset C_i$が成り立つ、つまり$W \subset Z$は既約である。
\end{proof}
    \item[(b)] $W = \pi_1^{-1}(X) \cap \pi_2^{-1}(Y)$により、$X \clsub \P^n$かつ$Y \clsub \P^m$ならば$W \clsub Z$である。
    \item[(c)] $W$が多様体の圏における直積であることを示そう。

    (a)での議論により、$W \subset Z$は部分多様体であることがわかった。そこで射影$\pi_1 \colon Z \to \P^n$, $\pi_2 \colon Z \to \P^m$を$W$に制限して、射影$\pi_1 \colon W \to X$と$\pi_2 \colon W \to Y$を得る。これが射であることは演習問題3.10が保証する。

    多様体$M$と射$\grs \colon M \to X$, $\tau \colon M \to Y$が与えられたとする。
    \[
    \xymatrix{
    X & W \ar[l]_-{\pi_1} \ar[r]^-{\pi_2} & Y \\
    {} & M \ar[lu]^-{\grs} \ar[ru]_-{\tau}  \ar@{.>}[u]^{\rho} & {}
    }
    \]
    射$\rho \colon M \to W$を$\rho(P)=\psi(\grs(P), \tau(P))$により定める。これは上の図式を可換にする唯一の写像であり、したがってあとは$\rho$が射であることをいえば十分である。

    $U_i \opsub \P^n$, $V_j\opsub \P^m$, $W_{ij} \opsub \P^N$を標準的な被覆とする。$X_i = X \cap U_i$, $Y_j = Y \cap V_j$とおくと、$X_i, Y_j$は準アファイン多様体と同形である。$M_{ij} = \grs^{-1}(X_i) \cap \tau^{-1}(Y_j) $として部分多様体$M_{ij} \opsub M$を定める。$M_{ij}$は$M$の開被覆を与えることに注意する。
    標準的な同形$\vp_i \colon U_i \to \A^n$と$\vp_j \colon V_j \to \A^m$をとる。
    \[
    \xymatrix{
    \vp_i(X_i) & \vp_i(X_i) \tm \vp_j(Y_j) \ar[l]_-{p_1} \ar[r]^-{p_2} & \vp_j(Y_j)   \\
    X_i \ar[u]^{\vp_i} & M_{ij} \ar[l]^-{\grs|_{M_{ij}}} \ar[r]_-{\tau|_{M_{ij}}} \ar@{.>}[u]_-{\beta} & Y_j \ar[u]_{\vp_j}
    }
    \]
    このとき、上の図式にある$\grs$と$\tau$の制限はそれぞれ射なので、準アファイン多様体の直積の普遍性により、ある射$\beta \colon M_{ij} \to \vp_i(X_i) \tm \vp_j(Y_j)$が存在して、上の図式が可換になる。ここで$\psi \circ (\vp_i^{-1} \tm \vp_j^{-1}) \colon \vp_i(X_i) \tm \vp_j(Y_j) \to W \cap W_{ij}$を$\psi \circ (\vp_i^{-1} \tm \vp_j^{-1})(P,Q) = \psi(\vp_i^{-1}(P), \vp_j^{-1}(Q))$
    により定める。$W \cap W_{ij}$は準アファイン多様体と同型である。多項式によって表される写像なので、準アファイン多様体の場合の補題3.6により$\psi \circ (\vp_i^{-1} \tm \vp_j^{-1}) \colon \vp_i(X_i) \tm \vp_j(Y_j) \to W \cap W_{ij}$は射。$P \in M_{ij}$に対して
    \begin{align*}
      \psi \circ (\vp_i^{-1} \tm \vp_j^{-1}) \circ \beta(P) &= \psi \circ (\vp_i^{-1} \tm \vp_j^{-1}) (\vp_i \circ \grs(P), \vp_j \circ \tau(P)) \\
      &= \psi(\grs(P), \tau(P)) \\
      &= \rho (P)
    \end{align*}
    が成り立つので
    \[
    \rho|_{M_{ij}} = \psi \circ (\vp_i^{-1} \tm \vp_j^{-1}) \circ \beta
    \]
    であって、右辺は射なので$\rho|_{M_{ij}} \colon M_{ij} \to W \cap W_{ij}$も射。適当な開被覆をとれば、それぞれに制限したときに射なので$\rho \colon M \to W$も射。
  \end{description}



  \bfsubsection{演習問題 3.16}
  \barquo{
  (a) $X \tm Y$が準射影多様体であることを示せ。
  }
  \begin{rem}
    完全にオマケの話だが、初期の案では$\psi(X \tm Y) \loc \P^N $を次のように示した。$X \tm Y$に積位相を入れてSegre埋め込みを位相的な写像だと考えるといかに話が面倒になるかをよく表しているように思われるので、書いておく。なぜ面倒になるのか考えたが、おそらくSegre埋め込みの位相的な性質があまりよくないからだろう。なにしろ埋め込みなのに (一方が有限集合という自明な場合を除いて) 像への同相写像になっていないのだから…。
  \end{rem}

  \lem{
        $\psi \colon \P^n \tm \P^m \to \P^N$がSegre埋め込みであるとし、$X \clsub \P^n$かつ$Y \clsub \P^m$ならば$\psi(X \tm Y) \clsub \P^N$である。
  }

\begin{proof}
  $\psi $は単射であるので、$\psi $の像は共通部分と交換することに注意する。そうすると、$X = Z(I)$, $Y = Z(J)$なる斉次イデアル$I,J$をとってくれば
  \begin{align*}
    \psi (X \tm Y) &= \psi (Z(I) \tm Z(J)) \\
    &= \psi \left( \bigcap_{f \in I : \text{homo}} (Z(f) \cap \P^m)  \cap \bigcap_{g \in J : \text{homo}} (\P^n \cap Z(g)) \right) \\
    &= \bigcap \psi(Z(f) \tm \P^m) \cap \bigcap \psi(\P^n \tm Z(g))
  \end{align*}
  である。よって斉次元$f \in k[x_0, \cdots , x_n]$について
  \[
  \psi(Z(f) \tm \P^m) \clsub \P^N
  \]
  が示せれば十分。

  いま$\grs \colon k[z_{ij}]  \to k[x_0, \cdots , x_n, y_0, \cdots , y_m] \st \grs(z_{ij}) = x_iy_j$とする。$f$は斉次元としたので、各$0 \leq j \leq m$に対して$y_j^{\deg f} f \in \Im \grs$である。そこで
  \[
  \psi( (Z(f) \tm \P^m) )= \bigcap_{j = 0}^m Z(\grs^{-1}(y_j^{\deg f}  f ) )
  \]
  を示そう。

  $\psi( (Z(f) \tm \P^m) ) \subset  \bigcap_{j = 0}^m Z(\grs^{-1}(y_j^{\deg f}  f ) )$であること: $(a,b) \in Z(f) \tm \P^m$と$j$と$g \in \grs^{-1}(y_j^{\deg f}  f ) $とが任意に与えられたとき
  \begin{align*}
    g(\psi (a,b)) &= (\grs g)(a,b) \\
    &= b_j^{\deg f} f(a)  \\
    &= 0
  \end{align*}
  が成り立つ。よって結論が示せた。

  $\psi( (Z(f) \tm \P^m) ) \supset  \bigcap_{j = 0}^m Z(\grs^{-1}(y_j^{\deg f}  f ) )$であること: $P \in \bigcap_{j = 0}^m Z(\grs^{-1}(y_j^{\deg f} f ) )$が与えられたとする。このとき任意の$g \in \grs^{-1}(y_j^{\deg f} f )$について$g(P) = 0$なので、$P \in Z(\Ker \grs)$
  である。したがって、$\psi(\P^n \tm \P^m) = Z(\Ker \grs)$という演習問題2.14の結果から$P = \psi(a,b)$なる$(a,b) \in \P^n \tm \P^m$があることがわかる。このとき任意の$j$と$g$について
  \begin{align*}
    0 &= g(P) \\
    &= g(\psi(a,b)) \\
    &= (\grs g)(a,b) \\
    &= (y_j^{\deg f})(a,b) \\
    &= b_j^{\deg f}f(a)
  \end{align*}
  である。よって、$b_j \neq 0$なる$j$が存在することから$f(a) = 0$である。これで示すべきことが言えた。
\end{proof}

\begin{proof}
演習3.16(a)の証明に戻る。$\psi$は単射であるので、$Z,W \subset \P^n \tm \P^m$に対して
\[
\psi(Z \setminus W) = \psi(Z) \setminus \psi(W)
\]
が成り立つことに注意しておく。

$X$,$Y$が準射影多様体であるということより、$X,Y$は局所閉であって、
\[
X = \ol{X} \cap U \quad Y = \ol{Y} \cap V
\]
なる$U \opsub \P^n$, $V \opsub \P^m$が存在する。すると
\begin{align*}
  \psi(U \tm V) &= \psi((\P^n \tm \P^m) \setminus (U^c \tm \P^m \cup \P^n \tm V^c)) \\
  &= \psi(\P^n \tm \P^m) \setminus \psi(U^c \tm \P^m \cup \P^n \tm V^c) \\
  &= \psi(\P^n \tm \P^m) \setminus (\psi(U^c \tm \P^m ) \cup \psi(\P^n \tm V^c))
\end{align*}
であるので、
\begin{align*}
\psi(X \tm Y ) &= \psi(\ol{X} \tm \ol{Y} \cap U \tm V)  \\
&= \psi(\ol{X} \tm \ol{Y}) \cap \psi(U \tm V)  \\
&= \psi(\ol{X} \tm \ol{Y}) \setminus (\psi(U^c \tm \P^m ) \cup \psi(\P^n \tm V^c))
\end{align*}
だから$\psi(X \tm Y) \subset \P^N $は局所閉である。
\end{proof}


\newpage

\bfsection{1.4 有理写像}


\bfsubsection{補題 4.1}
\barquo{
射$\vp$および$\psi$は写像$\vp \tm \psi \colon X \to \P^n \tm \P^n$を定めるが、実際これは射である
}
\begin{rem}
  $\P^n \tm \P^n$は多様体の圏における直積なので
  \[
  \xymatrix{
  \P^n & X \ar[r]^{\psi} \ar[l]_{\vp} \ar[d]^{\vp \tm \psi} & \P^n \\
  {} & \P^n \tm \P^n \ar[ul]^{\pi_1} \ar[ur]_{\pi_2} & {}
  }
  \]
  を可換にするような射$\vp \tm \psi$が存在する。
\end{rem}



\bfsubsection{補題 4.1}
\barquo{
$\grD = \setmid{P \tm P}{P \in \P^n}$を$\P^n \tm \P^n$の対角部分集合とする。

これは方程式$\setmid{x_iy_j = x_j y_i}{i,j = 0, \cdots , n}$で定義されるので$\P^n \tm \P^n$の閉部分集合である。
}
\begin{rem}
  Segre埋め込みを$s \colon \P^n \tm \P^n \to \P^N$で表すことにする。(Segre埋め込みを明示したということは、単に$\P^n \tm \P^n$と書けばそれは多様体の圏における直積ではなく集合の圏における直積を意味すると理解していただきたい) 次の補題を示そう。
\end{rem}

\lem{
(対角集合は閉) \\
Segre埋め込みを$s \colon \P^n \tm \P^n \to \P^N$とする。$Y \subset \P^n$が射影多様体で、
\[
\wt{\grD}(Y) = \setmid{P \tm P \in Y \tm Y}{P \in Y}
\]
とするとき、対角集合$\grD(Y) = s(\wt{\grD}(Y))$は$s(Y \tm Y)$の閉部分集合。
}
\begin{proof}
$\grD(Y) = s(Y \tm Y) \cap Z(\setmid{z_{ij} - z_{ji} }{0 \leq i,j \leq n})$を示せばよい。

$\grD(Y) \subset s(Y \tm Y) \cap Z(\setmid{z_{ij} - z_{ji} }{0 \leq i,j \leq n})$はあきらかなので、逆を示そう。

$z=(z_{ij}) \in s(Y \tm Y) \cap Z(\setmid{z_{ij} - z_{ji} }{0 \leq i,j \leq n})$とする。$z = s(x,y)$なる$x,y \in Y$をとる。このとき$\forall i,j \; \; x_i y_j = x_j y_i$が成り立つ。ある$i$について$y_i \neq 0$であるが、その$i$は$0$であるとして一般性を失わない。このとき$\forall j \; \; x_0 y_j / y_0 = x_j$が成り立つ。ある
$j$について$x_j \neq 0$なので、$x_0 \neq 0$でなければならない。よって$y_j / y_0 = x_j / x_0$であり、$x= y$がわかる。よって$z \in \grD(Y)$であるから、逆がいえた。
\end{proof}




\bfsubsection{補題 4.1}
\barquo{
仮定により$\vp \tm \psi (U) \subset \grD$である。
}
\begin{rem}
  $s \colon \P^n \tm \P^n \to \P^N$をSegre埋め込みとする。このとき次の図式が可換。
  \[
  \xymatrix{
  \P^n & X \ar[r]^{\psi} \ar[l]_{\vp} \ar[d]^{\vp \tm \psi} & \P^n \\
  {} & s(\P^n \tm \P^n) \ar[ul]^{\pi_1} \ar[ur]_{\pi_2} \ar[d]^{s^{-1}} & {} \\
  {} & \P^n \tm \P^n \ar[uul]^{\wt{\pi_1}} \ar[uur]_{\wt{\pi_2}} & {}
  }
  \]
  よって、$x \in X$に対して
  \begin{align*}
    (\vp \tm \psi)(x) \in \grD &\iff s^{-1} (\vp \tm \psi)(x) \in \setmid{P \tm P \in \P^n \tm \P^n}{P \in \P^n} \\
    &\iff \wt{\pi_1}s^{-1}(\vp \tm \psi)(x) = \wt{\pi_2}s^{-1}(\vp \tm \psi)(x) \\
    &\iff \vp(x) = \psi(x)
  \end{align*}
  であるから、求める包含関係がいえた。
\end{rem}




\bfsubsection{補題 4.2 直前}
\barquo{
ある(したがってすべての)対$\kakko{U,\vp_U}$について$\vp_U$の像が$Y$において稠密であるとき、有理写像$\vp$は支配的(dominant)であるという。
}
\begin{proof}
dominantという性質が同値類の取り方によらないことを示そう。

  $\kakko{V,\psi} = \kakko{U,\vp}$であるとし、$\vp(U) \subset Y$は稠密であるとする。
  \begin{align*}
    \vp(U) &= \vp(\ol{U \cap V} \cap U) &(\text{$U \cap V$は$U$の稠密部分集合}) \\
    &\subset \ol{\vp(U \cap V)} &(\text{$\vp$の連続性})
  \end{align*}
  よって$\vp(U) \subset Y$の稠密性から$Y = \ol{\vp(U \cap V)}$がわかる。

  ここで
  \[
  \psi(V) \supset \psi(U \cap V) = \vp(U \cap V)
  \]
  であるから、したがって$\psi(V) \subset Y$は稠密である。
\end{proof}



\bfsubsection{補題 4.2 直前}
\barquo{
明らかに支配的有理写像は合成することができるので、多様体と支配的有理写像のなす圏を考えることができる。
}
\begin{proof}
  支配的有理写像の合成が再び支配的であることを示そう。
  \[
  \xymatrix{
  X \ar[r]^{\wt{\vp}} & Y \ar[r]^{\wt{\psi}} & Z
  }
  \]
  支配的有理写像$\wt{\vp}, \wt{\psi}$が上のようにあり、代表元が
  \[
\kakko{U,\vp} = \wt{\vp} \quad \kakko{V,\psi} = \wt{\psi}
  \]
  と与えられていたとする。このとき合成$\wt{\psi} \circ \wt{\vp}$を、$\wt{\psi} \circ \wt{\vp} = \kakko{\vp^{-1}(V),\psi \circ \vp}$によって定める。
  以下、確認すべき事を粛々と確認していく。

  定義域が空集合でないこと:
  \begin{align*}
      \ol{\vp(U) \cap V} \cap V &= \ol{\vp(U)} \cap V &(\text{閉包と開集合への制限の可換性}) \\
      &= V &(\text{$\vp(U)$の稠密性})
  \end{align*}
  よって
  \begin{align*}
    \psi(V) &= \psi(  \ol{\vp(U) \cap V} \cap V ) \\
    &\subset \ol{\psi(\vp(U) \cap V)} &(\text{$\psi$の連続性})
  \end{align*}
  であるから、$\psi(V) \subset Z$の稠密性により
  \[
  Z = \ol{\psi(\vp(U) \cap V)}
  \]
  が成り立つ。ゆえに$\vp(U) \cap V \neq \emptyset$であるので、したがって$\vp^{-1}(V) \neq \emptyset$である。

  合成が支配的であること:
  \begin{align*}
  \psi(V) &= \psi( \ol{\vp(U) \cap V} \cap V ) \\
  &\subset \ol{\psi(\vp(U) \cap V)} \\
  &\subset \ol{\psi \circ \vp (\vp^{-1}(V))}
\end{align*}
が成り立つので、$\psi(V) \subset Z$の稠密性により
\[
Z = \ol{\psi \circ \vp (\vp^{-1}(V))}
\]
がわかる。
\end{proof}


\bfsubsection{補題 4.2}
\barquo{
$Y$を方程式$f(x_1, \cdots, x_n)=0$で与えられる$\A^n$の超曲面とする。
}
\begin{rem}
  \textblue{$Y$は超曲面でなくてもよい。}$f$がどんな多項式だったとしても、結論はいえる。
\end{rem}


\bfsubsection{補題 4.2}
\barquo{
$Y$を方程式$f(x_1, \cdots, x_n)=0$で与えられる$\A^n$の超曲面とする。このとき$\A^n - Y$は$x_{n+1}f = 1$で与えられる$\A^{n+1}$の超曲面$H$に同型である。特に$\A^n - Y$はアファインであり、そのアファイン環は$k[x_1, \cdots , x_n]_f$である。
}
\begin{proof}
  $x_{n+1}f - 1  \in k[x_1, \cdots , x_{n+1}]$が素元であること:
  $A = k[x_1, \cdots , x_n]$とする。$x_{n+1}f - 1 = gh$なる$g,h \in A[x_{n+1}]$が与えられたとする。イデアル$(x_{n+1})$による商をとって、$\ol{g},\ol{h} \in A^{\tm} = k$がわかる。ゆえに$g = x_{n+1}g' + c$, $h = x_{n+1}h' + d$なる$c,d \in k$と$g' , h' \in A[x_{n+1}]$
  がある。ゆえに
  \begin{align*}
  x_{n+1}f - 1 &= gh \\
     &= (x_{n+1}g' + c)(x_{n+1}h' + d) \\
    &= x_{n+1}^2 g'h' + x_{n+1}(ch' + d g') + cd
  \end{align*}
  である。ここで$x_{n+1}$に関して最高次数の項は$x_{n+1}^2 g'h'$に含まれるので$g' h' = 0$でなくてはならない。よって$g$と$h$のどちらかは$A[x_{n+1}]$の単元である。$g$と$h$は任意だったから$x_{n+1}f - 1 \in A[x_{n+1}]$は既約元、$A[x_{n+1}]$はUFDなのでとくに素元である。このことから$H$が多様体になっていることが保証される。

  $\A^n \setminus Y$と$H$が同型であること:
  $\vp \colon H \to \A^n \setminus Y$を$\vp(a_1, \cdots , a_{n+1}) = (a_1, \cdots , a_{n})$として定め、$\psi \colon \A^n \setminus Y \to H$を$\psi(x_1, \cdots , x_n) = (x_1, \cdots , x_n, 1/f((x_1, \cdots , x_n)))$として定める。$\vp$, $\psi$は互いに逆写像であり、よって$\vp$は全単射。$\vp$, $\psi$
  は有理関数で表される写像なので多様体の圏の射であり、したがって$\A^n \setminus Y$と$H$は同型。

  $A(\A^n \setminus Y) \cong A[x_{n+1}]/(x_{n+1}f - 1)$であること:
多様体としての同型$\A^n \setminus Y \cong H$があるので、座標環に送って同型$A(\A^n \setminus Y) \cong A(H)$がいえる。さらに$A(H) = A[x_{n+1}] / I(H) = A[x_{n+1}] / IZ(x_{n+1}f - 1) = A[x_{n+1}] / (x_{n+1}f - 1)$であることから、求める同型がいえた。

$A[x_{n+1}]/(x_{n+1}f - 1) \cong A_f$であること:
$p \colon A \to A[x_{n+1}]/(x_{n+1}f - 1)$を自然な射の合成$A \to A[x_{n+1}] \to A[x_{n+1}]/(x_{n+1}f - 1)$として定める。このとき積閉集合$S = \setmid{f^n}{n \geq 0}$の$p$による像は単元なので、局所化の普遍性により$\wt{p} \colon A_f \to A[x_{n+1}]/(x_{n+1}f - 1)$が誘導される。
$p$は単射なので、$\wt{p}$も単射。また、あきらかに$\wt{p}$は全射なので$\wt{p}$は同型。
\end{proof}


\bfsubsection{命題 4.3}
\barquo{
$Y$は$\A^n$の中の準アファイン多様体と仮定してよい。
}
\begin{proof}
  $Y$が準アファイン多様体のときに示せたとする。準射影多様体$Z \loc \P^n$と$P \in Z$が与えられたとしよう。このとき$\vp_i \colon U_i \to \A^n$を標準的な同型とすると、$Z \cap U_i \loc U_i$なので$\vp_i(Z \cap U_i)$は$\A^n$の局所閉部分集合であり、かつ既約。したがって準アファイン多様体。ゆえに仮定により$\vp_i(P) \in V \opsub \vp_i(Z \cap U_i)$なるアファイン集合$V$がある。当然$P \in \vp_i^{-1}(V) \opsub Z$であるので、準射影多様体についても示すべきことがいえたことになる。
\end{proof}



\bfsubsection{命題 4.3}
\barquo{
このとき、$Z$は閉で$P \notin Z$だから$f(P) \neq 0$となるような多項式$f \in \fraka$を見つけることができる。$H$を$\A^n$の中の超曲面$f=0$とする。このとき$Z \subset H$であるが
}
\begin{rem}
  \textblue{ここは注意が必要である。}$H$が超曲面であることを保証するには、$f \in A$は既約元でなくてはならない。そのためには、$f(P) \neq 0$なる$f \in \fraka$を素元分解するくらいしかないが、そうして得られた$f$の素因子$p$はもはや$\fraka$の元である保証がない。そうすると$Z \subset H$が示せなくなり、$Y \setminus H \clsub \A^n \setminus H$が示せなくなり御破算になる。

  これを解決する方法はない。なぜなら、この部分の記述が間違っているからである。$H$は超曲面でなくてもよいのだ。補題4.2についての注意を参照のこと。
\end{rem}


\bfsubsection{定理 4.4 直前}
\barquo{
$\vp_U(U)$は$Y$で稠密だから$\vp_U^{-1}(V)$は$X$の空でない開部分集合であり、
}
\begin{proof}
  補題4.2の直前の「支配的有理写像の合成は支配的」の特別な場合である。
\end{proof}


\bfsubsection{定理 4.4 直前}
\barquo{
以上のようにして、$k$代数$K(Y)$から$K(X)$への準同形が定義された。
}
\begin{rem}
  函数体が有理写像のなす圏から$k$上有限生成体拡大の圏への反変関手であるということである。このことからただちに次の命題が従う。
\end{rem}
\prop{
多様体$X$, $Y$が双有理同値ならば、$\dim X = \dim Y$である。
}
\begin{proof}
このとき$K(X)$と$K(Y)$は同型であるから、当然$k$上の超越次数も等しい。ところが、函数体の$k$上の超越次数はもとの多様体の次元に等しいので、$\dim X = \dim Y$がいえる。
\end{proof}


\bfsubsection{定理 4.4}
\barquo{
$Y$はアファイン多様体により覆われるから、$Y$はアファイン多様体と仮定してよい。
}
\begin{rem}
  まず次の補題に気をつける。
\end{rem}

\lem{
$Y$が多様体、$\emptyset \subsetneq U \opsub Y$であるとき、$U$と$Y$は双有理同値。
}
\begin{proof}
  結論がわかっていれば確かめるのは容易い。
\end{proof}

\begin{proof}
  引用部の証明に戻る。$Y$がアファインのときに、対応$\grd \colon \Hom_k(K(Y), K(X)) \to \Hom_{\text{Rat}}(X,Y)$がつくれたとする。$Y$が一般の多様体であるとき、$U \opsub Y$なるアファイン集合$U$をとると$K(Y) = k(U)$かつ、$U$と$Y$は双有理同値なので、図式
  \[
  \xymatrix{
  \Hom_k(K(Y), K(X)) \ar@{=}[d] \ar[r]^-{\grd_U} &  \Hom_{\text{Rat}}(X,Y) \\
\Hom_k(K(U), K(X)) \ar[r]^-{\grd} &  \Hom_{\text{Rat}}(X,U) \ar[u]_{\text{iso}}
  }
  \]
  が可換になるように$\grd$を拡張することができる。これは$U$の取り方によらない。
\end{proof}




\bfsubsection{定理 4.4}
\barquo{
(3.5)によりこれは射$\vp \colon U \to Y$に対応し、この射は$X$から$Y$への支配的有理写像を与える。
}
\begin{rem}
  次の補題に気をつける。
\end{rem}

\lem{
多様体$X$とアファイン多様体$Y$、および射$\vp \colon X \to Y$が与えられているとき次は同値。
\begin{description}
  \item[(1)] $\vp^* \colon A(Y) \to \calo(X)$が単射
  \item[(2)] $\vp(X) \subset Y$は稠密
\end{description}
}
\begin{proof}
同値変形を行って示す。
\begin{align*}
  \vp^* \text{が単射} &\iff \Ker \vp^* = \{ 0 \} \\
&\iff \forall f \in A(Y) \; \; (f \circ \vp)(U) = 0 \; \text{ならば} \; f=0 \\
&\iff Z(f) \supset \vp(U) \; \text{ならば} \; f=0 \\
&\iff f \neq 0 \; \text{ならば} \; (Y \setminus Z(f)) \cap \vp(U) \neq \emptyset
\end{align*}
ここで$f \in A(Y)$であるが、同値類の代表元をどう取ろうと$Y$上の各点での値には関係がないので、$Z(f)$という記号を使った。
$\{Y \setminus Z(f) \}_{f \in A(Y)}$が$Y$の位相の開基であることから、示すべきことがいえる。
\end{proof}

\begin{proof}
  引用部の証明に戻る。$\grt |_{A(Y)} \colon A(Y) \to \calo(U)$はk代数の準同形だから、充満性によりある射$\vp \colon U \to Y$が存在して$\vp^* \colon A(Y) \to \calo(U)$が$\grt |_{A(Y)}$に等しい。このとき補題から$\vp \colon X \to Y$は支配的有理写像。
\end{proof}


\bfsubsection{定理 4.4}
\barquo{
これが集合(ii)から集合(i)への写像を与えること、またそれが上で定義した写像の逆であることは容易に分かる。
}
\begin{proof}
  $\grg \colon \Hom_{\text{Rat}} (X,Y) \to \Hom_k(K(Y), K(X))$とおく。$\grd \circ \grg = id$, $\grg \circ \grd = id$を示せばよい。$Y$ははじめからアファインとしてよい。

    $\grd \circ \grg = id$であること:
    次の図式の可換性を示せばよい。
    \[
    \xymatrix{
    \Hom_{\text{Rat}}(X,Y) \ar[r]^-{\grg} \ar[dd]_{id} & \Hom_k(K(Y),K(X)) \ar[d]^{\cdot |_{A(Y)}} \\
    {} & \Hom_k(A(Y), \calo(U)) \ar[d]^{\beta} \\
\Hom_{\text{Rat}}(X,Y) & \Hom_{\text{Var}}(U,Y) \ar[l]
    }
    \]
    いま$p \in U$と$\kakko{U,\vp} \in \Hom_{\text{Rat}}(X,Y)$について
    \begin{align*}
      \beta (\grg \vp |_{A(Y)})(p) &= (( \grg \vp |_{A(Y)}(y_1) )(p), \cdots , ( \grg \vp |_{A(Y)}(y_n) )(p)) \\
      &= (( \grg \vp (y_1) )(p), \cdots , ( \grg \vp (y_n) )(p)) \\
      &= (\grg \vp)(y_1, \cdots , y_n)(p) \\
      &= \vp(p)
    \end{align*}
    だから、示すべきことがいえた。

$\grg \circ \grd = id$であること:
次の図式の可換性を示せばよい。
\[
\xymatrix{
\Hom_{k}(K(Y),K(X))   & \Hom_k(K(Y),K(X)) \ar[d]^{\cdot |_{A(Y)}} \ar[l]_-{id} \\
{} & \Hom_k(A(Y), \calo(U)) \ar[d]^-{\beta} \\
\Hom_{k}(X,Y) \ar[uu]^-{\grg} & \Hom_{\text{Var}}(U,Y) \ar[l]
}
\]
いま$\grt \in \Hom_k(K(Y), K(X))$と$f \in K(Y)$について
\begin{align*}
  (\grg \circ \beta (\grt|_{A(Y)}) )(f) &= f(\beta(\grt|_{A(Y)})) \\
  &= f(\grt(y_1), \cdots , \grt(y_n)) \\
  &= \grt(f(y_1, \cdots , y_n)) \\
  &= \grt(f)
\end{align*}
だから、示すべきことがいえた。
\end{proof}






\bfsubsection{系 4.5}
\barquo{
$(1) \To (2)$. $\vp \colon X \to Y$および$\psi \colon Y \to X$を互いに逆の有理写像としよう。$\vp$は$\kakko{U,\vp}$で代表され、$\psi$は$\kakko{V,\psi}$で代表されるものとする。このとき$\psi \circ \vp$は$\kakko{\vp^{-1}(V),\psi \circ \vp}$
で代表され、
}
\begin{rem}
有理写像と代表元を同じ記号で書いていることに注意。あと、「多様体と支配的有理写像の圏」で考えているのであって、「多様体と有理写像の圏」ではないことにも注意。またこの系では演習問題3.10を実は使っている。
\end{rem}


\bfsubsection{命題 4.9}
\barquo{
任意の$r$次元多様体$X$は$\P^{r+1}$内のある超曲面$Y$に双有理である。
}
\begin{rem}
  \textblue{射影空間の超曲面でなくてもよい。}以下の証明からわかるように、$X$と双有理であるような$\A^{r+1}$の超曲面$H$が存在する。
\end{rem}



\bfsubsection{命題 4.9}
\barquo{
$X$の函数体$K$は$k$の有限生成拡大体である。
}
\begin{proof}
$X$がアファイン多様体なら、定理3.2(d)で示されている。$X$が準アファイン多様体なら、$X \loc \A^n$となる$n$がある。したがって$X$は$\ol{X}$の開部分集合だから$K(X) = K(\ol{X})$となり、アファインの場合に帰着される。$X$が準射影多様体なら、$X \cap U_i \neq \emptyset$なる$i$をとる。ただし$U_i \opsub \P^n$はいつもの開集合である。このとき$X \cap U_i \loc U_i$により$\vp_i(X \cap U_i) \loc \A^n$がわかる。ゆえに$K(X) = K(X \cap U_i) = K(\vp_i(X \cap U_i))$
より準アファイン多様体の場合に帰着できる。
\end{proof}




\bfsubsection{命題 4.9}
\barquo{
よって超越基$x_1 , \cdots , x_r \in K$であって$K$が$K(x_1 , \cdots , x_r)$の有限次分離拡大であるようなものを見付けることができる。
}
\begin{rem}
次元については、$r = \dim X = \trdeg_k K$という結果がすでにあるので、よい。
  体論をど忘れしていると、有限次拡大というところに一瞬詰まるかもしれない。次の命題を思い出そう。
\end{rem}

\prop{
$L/K$を体拡大とする。このとき次は同値。
\begin{description}
  \item[(1)] $[L : K] < \infty$
  \item[(2)] $L/K$は代数拡大かつ$L$は体として$K$上有限生成。
  \item[(3)] $L$は$K$代数として有限生成。
\end{description}
}
\begin{proof}
  $(1)\Leftrightarrow (2)$は生成元の個数についての帰納法による。$(1)\Leftrightarrow(3)$はZariskiの補題による。
\end{proof}



\bfsubsection{命題 4.9}
\barquo{
分母を払って$f(x_1, \cdots , x_r, y) = 0$を得る。これは函数体$K$を持つ$\A^{n+1}$の超曲面を定めるが、(4.5)によるとこれは$X$と双有理である。これの射影閉包(Ex. 2.9)が求める超曲面$Y \subset \P^{r+1}$である。
}
\begin{proof}
  $A = k[x_1, \cdots , x_r]$とする。紛らわしいので、函数体$K(X)$のことを単に$K$とは書かないことにする。本文にあるように、分母を払うことによりある$g \in A[x_{r+1}]$であって$g(y)=0$なるものが存在することがわかる。$A[x_{r+1}]$はUFDなので素元分解して、$f(y)=0$なる素元$f \in A[x_{r+1}]$の存在がわかる。以降、$H = Z(f) \subset \A^{r+1}$とする。

  このとき$K(H) \cong \Frac A(H) = \Frac A[x_{r+1}]/f = \Frac A[y] = K(X)$であるので、$X$と$H$の函数体は同形。ゆえに圏同値があることから求める双有理同値がいえる。

  また、$U_0 \opsub \P^{r+1}$と標準的な同形$\vp \colon U_0 \to \A^{r+1}$をとると、$\vp^{-1}(H) \clsub U_0$であるから、$\vp^{-1}(H) \loc \P^{r+1}$がいえる。したがって$\P^{r+1}$における閉包をオーバーラインで表すと$\vp^{-1}(H) \opsub \ol{\vp^{-1}(H)}$
  である。ゆえに、$K(\ol{\vp^{-1}(H)}) = K(\vp^{-1}(H)) \cong K(H) \cong K(X)$である。

  あとは、$\ol{\vp^{-1}(H)}$が本当に超曲面であること、つまりイデアルが単項イデアルであることを示せばよいが、演習問題2.9での考察によりそれはあきらか。
\end{proof}






\bfsubsection{ブローアップ}
\barquo{
ここで$\A^n$の点$O$におけるブローアップを、方程式
\[
\setmid{x_iy_j = x_j y_i}{i,j = 1, \cdots , n}
\]
によって定義される$\A^{n} \tm \P^{n-1}$の閉部分集合$X$として定義する。
}
\begin{rem}
  Segre埋め込み$s \colon \P^n \tm \P^{n-1} \to \P^N$と同形$\vp_0 \colon U_0 \to \A^n$を用いて明示的に書いてみると次のようになる。
\begin{align*}
  X &= \setmid{s(\vp_0^{-1}(x), y )}{\forall i,j \;  x_iy_j = x_j y_i} \\
  &= s(\vp_0^{-1}(\A^n) \tm \P^{n-1}) \cap \bigcap_{1 \leq i,j \leq n} Z(z_{ij} - z_{ji})
\end{align*}
ただし、ここで$j$の動く範囲は$1 \leq j \leq n$であり、$i$の動く範囲は$0 \leq i \leq n$であって、少しズラしていることに注意する。このズレは$\P^{n-1}$の斉次座標の取り方がズレていたことから来ている。

Segre埋め込みを明示してしまったので、紛れの無いよう多様体の圏における直積は$\A^n \vartm \P^{n-1}$などと書くことにしよう。またあまり明示的に書きすぎると却って煩わしい場面もあると思う。そこで、$s(\vp_0^{-1}(x), y)$のことを単に$x \ts^{s} y$と書くことにする。以上の記号法はここだけのものであるので他所で使われないようにされたい。
\end{rem}




\bfsubsection{ブローアップ}
\barquo{
$\A^n \tm \P^{n-1}$から第一因子への射影を制限することによって、自然な射$\vp \colon X \to \A^n$が得られる。
}
\begin{rem}
  \textblue{$X$の既約性はまだ示していない}はずなのに、いきなり射であるといわれても困る。以下、$\vp$はとりあえずただの写像だと思って話を進める。
\end{rem}



\bfsubsection{ブローアップ}
\barquo{
(1) $P \in \A^n$, $P \neq O$のとき、$\vp^{-1}(P)$はただひとつの点からなる。実は$\vp$は$X - \vp^{-1}(O)$から$\A^n - O$への同型を与える。
}
\begin{proof}
  まず順序を変えて$X- \vp^{-1}(O)$の既約性を示そう。次の補題に注意する。なお補題の名前は僕(@seasawher)が勝手につけた。
  \lem{
  (射影の普遍性)\\
  $X$, $Y$, $Z$を多様体とし、$\vp \colon Z \to X \tm Y$を写像とする。$\pi_1 \colon X \tm Y \to X$, $\pi_2 \colon X \tm Y \to Y$を射影とする。このとき次は同値。
\begin{description}
  \item[(1)] $\vp$は射。
  \item[(2)] $\pi_1 \circ \vp$と$\pi_2 \circ \vp$はともに射。
\end{description}
  }
  \begin{proof}
    $(1) \To (2)$はあきらか。$(2) \To (1)$を示そう。多様体の積の普遍性により、次の図式
    \[
    \xymatrix{
    X & X \tm Y \ar[l]_-{\pi_1} \ar[r]^-{\pi_2} & Y \\
    {} & Z \ar[ul]^-{\pi_1 \circ \vp} \ar[ur]_-{\pi_2 \circ \vp} \ar@{.>}[u]^-{\psi} & {}
    }
    \]
    を可換にする射$\psi$が存在する。このとき図式の可換性から、$\vp = \psi$でなくてはならないので$\vp$は射。
  \end{proof}

  $X - \vp^{-1}(O)$の既約性の証明に戻る。写像$\psi \colon \A^n - O \to \A^n \vartm \P^{n-1}$を$\psi(x) = x \ts^s q(x)$で定める。ただし$q \colon \A^n - O \to \P^{n-1}$は自然な商写像である。$q$は多項式で与えられる写像なので射。すると$\psi$は、射影をかませるごとに射を与えるので射影の普遍性から射であることがわかる。とくに$\psi$は連続。

  本文中にあるのと同様の証明で
  \[
  X - \vp^{-1}(O) = \setmid{x \ts^s q(x)}{x \in \A^n - O}
  \]
  がいえる。したがって$X - \vp^{-1}(O) = \psi(\A^n - O)$は既約空間の連続像なので既約。

次に$\vp \colon X - \vp^{-1}(O) \to \A^n - O$がHartshorneの言うとおり射であることを示す。いま$\pi_1 \colon \A^n \vartm \P^{n-1} \to \A^n$を自然な射影とする。このとき$X - \vp^{-1}(O) = X \setminus \pi_1^{-1}(O)$は$\A^n \vartm \P^{n-1}$の閉部分集合と開部分集合の共通部分なので局所閉部分集合であり、したがって部分多様体だと思える。ゆえに演習問題3.10により$\pi_1 \colon \A^n \vartm \P^{n-1} \to \A^n$
の制限として得られる写像$\vp \colon X - \vp^{-1}(O) \to \A^n - O$は射である。

これだけ示しておけばあとは本文通りにすれば同型$X - \vp^{-1}(O) \cong \A^n - O$がいえる。
\end{proof}





\bfsubsection{ブローアップ}
\barquo{
(2) $\vp^{-1}(O) \cong \P^{n-1}$.
}
\begin{proof}
  (1)と同様の議論で示せるので、詳しくは説明しない。$\vp^{-1}(O)$は、射$\P^{n-1} \to \A^n \vartm \P^{n-1} \st Q \mapsto O \ts^s Q$による既約空間$\P^{n-1}$の像だから既約。射影$\A^n \vartm \P^{n-1} \to \A^n$による一点集合$O$の引き戻しと集合として等しいので、$\vp^{-1}(O) \clsub \A^n \vartm \P^{n-1}$
  もいえる。よって$\vp^{-1}(O)$は$\A^n \vartm \P^{n-1}$の部分多様体。したがって、演習問題3.10の結果が使えて、望みの同型がいえる。
\end{proof}





\bfsubsection{ブローアップ}
\barquo{
(3) $\vp^{-1}(O)$の点たちは$O$を通る$\A^n$の中の直線の集合と一対一対応にある。
}
\begin{proof}
  $\A^n$内の直線$L$が与えられたとする。このときある$a \in \A^n - O$が存在して
  \[
  L = \setmid{ta}{t \in \A^1 }
  \]
  が成り立つ。

  $L \clsub \A^n$であること: 斉次式$f$を
  \[
  f(x) = \prod_{1 \leq i < j \leq n} (a_j x_i - a_i x_j)
  \]
  で定める。このとき$L \subset Z(f)$であることはあきらか。逆に$y \in Z(f)$ならば、$a \neq O$であることから、$y \in L$であることがわかる。よって$L = Z(f)$であり、示すべき事がいえた。

  $L'$, $\ol{L'}$の定義: $L' = \vp^{-1}(L - O)$とする。つまり
  \begin{align*}
  L' &= \setmid{ta \ts^s q(ta)}{t \in \A^1 - O} \\
  &= \setmid{ta \ts^s q(a)}{t \in \A^1 - O}
\end{align*}
  である。そうして、$X$における$L'$の閉包を$\ol{L'}$とする。

  $\ol{L'} = L' \cup \{ O \ts^s q(a) \}$であること: 斉次式$f$であって、$L'$を零にするもの、つまり
  \[
  \forall t \in \A^1 - O \quad f(ta \ts^s q(a)) = 0
  \]
  なるものが与えられたとする。$f$を、$z_{01}, \cdots , z_{0n}$を除いた、$z_{ij} \; (1 \leq i \leq n, 1 \leq j \leq n)$についての次数で分けて$f = \sum_{k=0 }^{m} f_k$と表すことにすると任意の$t \in \A^1$について
  \[
  f(ta \ts^s q(a)) = f_0(a \ts^s q(a)) + \sum_{k=1}^m t^k f_k(a \ts^s q(a))
  \]
  が成り立つ。左辺は$t \in \A^1 - O$のときには常に$0$である。よって$\A^1$は代数閉体、とくに無限体なので右辺の$t$の係数はすべて$0$である。とくに$f_0(a \ts^s q(a))=0$がわかる。したがって$t=0$を上の式に代入して$f(O \ts^s q(a)) = 0$である。以上の議論により、$L' \cup \{ O \ts^s q(a) \} \subset \ol{L'}$がわかる。逆を示そう。$\pi_1 \colon \A^n \vartm \P^{n-1} \to \A^n$, $\pi_2 \colon \A^n \vartm \P^{n-1} \to \P^{n-1}$
  を射影とする。このとき
  \begin{align*}
  L' \cup \{ O \ts^s q(a) \} &= \setmid{ta \ts^s q(a)}{t \in \A^1} \\
  &= \pi_1^{-1}(L) \cap \pi_2^{-1}(q(a))
\end{align*}
である。射影空間において一点は閉であり、かつ$L \clsub \A^n$であるので、射影の連続性から$L' \cup \{ O \ts^s q(a) \} \clsub X$がわかる。したがって$L' \cup \{ O \ts^s q(a) \} \supset \ol{L'}$である。よって示すべきことがいえた。

一対一対応があること: 商写像$q$が全射であることより、容易に従う。
\end{proof}


\bfsubsection{ブローアップ}
\barquo{
最初の部分は$\A^n$に同型であり、したがって既約である。
}
\begin{rem}
  \textblue{このりくつはおかしい。}理由は先述の通り。
\end{rem}





\bfsubsection{ブローアップの定義}
\barquo{
$\vp \colon X \to \A^n$を$\wt{Y}$に制限して得られる射も$\vp \colon \wt{Y} \to Y$と書く。
}
\begin{proof}
  $\wt{Y}$は$\vp^{-1}(Y - O)$の$X$または$\A^n \tm \P^{n-1}$における閉包であるから、
  \begin{align*}
    \vp(\wt{Y}) &= \vp(\ol{\vp^{-1}(Y-O)}) \\
    &\subset \ol{\vp(\vp^{-1}(Y-O))} \\
    &\subset \ol{Y-O} \\
    &\subset Y
  \end{align*}
  が成り立つ。

  $\wt{Y}$が$\A^n \tm \P^{n-1}$の部分多様体であることを示そう。$Y-O$は$Y$の(たぶん空でない)開部分集合なので既約。$\vp^{-1} \colon \A^n-O \to X - \vp^{-1}(O)$は同型だったから$\vp^{-1}(Y - O)$も既約。閉包をとって、$\wt{Y}$も既約。局所閉部分集合であることはあきらかなので、部分多様体。

  したがって、射$\vp \colon \wt{Y} \to Y$が誘導される。
\end{proof}



\bfsubsection{ブローアップの定義 直後}
\barquo{
$\vp$は$\wt{Y} -\vp^{-1}(O)$から$Y-O$への同型を引き起こし、
}
\begin{rem}
  あきらかな等式$\vp(\wt{Y} -\vp^{-1}(O)) = Y - O$から従う。
\end{rem}


\bfsubsection{例 4.9.1}
\barquo{
$Y$を方程式$y^2=x^2(x+1)$で与えられる平面三次曲線とする。
}
\begin{rem}
  このあとの議論で出てくる数式をすべて明示的に書くと次のようになる。例外曲線$E$は
  \[
  E = \setmid{O \tm (t,u) \in \A^2 \tm \P^1}{(t,u) \in \P^1}
  \]
  と表される。また$Y$の逆像は
  \[
  \vp^{-1}(Y) \cap \{ t \neq 0 \} = \{ O \tm (1,u) \}_{u \in \A^1} \cup \{(u^2-1,u(u^2-1)) \tm (1,u) \}_{u \in \A^1}
  \]
  である。$X$軸と$Y$軸の逆像は次のようになる。
  \begin{gather*}
\vp^{-1}(\{ y=0 \}) =E \cup \{(x,0) \tm (1,0)\}_{x \in \A^1} \\
  \vp^{-1}(\{ x=0 \}) =E \cup \{(0,y) \tm (0,1)\}_{y \in \A^1}
  \end{gather*}
\end{rem}

\begin{comment}
\bfsubsection{例 4.9.1}
\barquo{
ここで$\P^1$は開集合$t \neq 0$と$u \neq 0$で覆われるので、別々に考える。
}
\begin{rem}
このあと、開集合$u \neq 0$の場合が考察されることはついになかった…。いったいどういうつもりなのか。
\end{rem}
\end{comment}


\bfsubsection{演習問題 4.7}
\begin{description}
\item[Step 1] まず$X$, $Y$がアファイン多様体の場合に示そう。圏同値に帰着したい。$\tau \colon \calo_{P,X} \to \calo_{Q,Y}$を$k$-代数の同型であるとする。とくに$\tau$は単射なので、$\calo_{P,X} \to \calo_{Q,Y} \to \Frac \calo_{Q,Y} = K(Y)$は単射である。ゆえに、局所化の普遍性から$\wt{\tau} \colon K(X) \to K(Y)$なる準同形が誘導される。
  $\tau$が同型なので、$\wt{\tau}$も同型である。したがって、圏同値があることにより、ある支配的有理写像$\vp \colon Y \to X$が存在して、次を満たす。

  (1) 定理4.4における対応$( \cdot )^* \colon \Hom_{Rat}(Y,X) \to \Hom_k(K(X),K(Y))$について$\vp^* = \wt{\tau}$が成り立つ。

  (2) 定理4.4における対応$( \cdot )^{\dagger} \colon \Hom_{k}(K(X),K(Y)) \to \Hom_{Rat}(Y,X)$について$\vp = (\wt{\tau})^{\dagger} $が成り立つ。
\item[Step 2] (2)から従うことを見ていく。$x_1 , \cdots ,x_n \in A(X)$を$k$代数としての生成元とする。次の図式が可換であることに気をつける。
  \[
\xymatrix{
A(X) \ar[dr] \ar[r] & \calo_{P,X} \ar[d]^-{j_X} \ar[r]^-{\tau} & \calo_{Q,Y} \ar[d]^-{j_Y} \\
{} & K(X) \ar[r]^-{\wt{\tau}} & K(Y)
}
  \]
すると$\wt{\tau}(x_i) = j_Y \circ \tau(x_i) \in K(Y)$より、$\wt{\tau}(x_i) = \kakko{V_i, t_i}$なる$(V_i, t_i) \in \coprod_{Q \in V} \calo(V)$が存在する。このとき$V = \bigcap_{i=1}^n V_i$とすると$\wt{\tau}$の$A(X)$への制限は$\tau' \colon A(X) \to \calo(V)$とみなせる。
したがって命題3.5により、ある支配的写像$\vp_V \colon V \to X$が存在して、$\vp_V$が誘導する写像が$\tau'$に一致する。すなわち、
\[
\forall f \in A(X) \quad f \circ \vp_V = \tau'(f)
\]
が成り立つ。このとき$(\wt{\tau})^{\dagger} = \kakko{V,\vp_V }$であるので、(2)により$\vp = \kakko{V,\vp_V }$が成りたつ。とくに、$\vp$の代表元として$Q$の近傍で定義された射をとることができる。
\item[Step 3] $\tau$は局所環の間の同型なので、$\frakm_P$, $\frakm_Q$をそれぞれ$\calo_{P,X}$と$\calo_{Q,Y}$の極大イデアルとすると、$\tau(\frakm_P) = \frakm_Q$である。このことの帰結を調べる。まず、$\calo_{Q,Y} \to K(Y)$は単射なので、次は可換である。($\tau'$は$\wt{\tau}$の制限であったが、$\tau$の制限でもあることがこのことからわかる)
\[
\xymatrix{
K(X) \ar[r]^-{\wt{\tau}} & K(Y) \\
\calo_{P,X} \ar[u]^-{j_X} \ar[r]^-{\tau} & \calo_{Q,Y} \ar[u]_-{j_Y} \\
A(X) \ar[u]^-{i_X} \ar[r]^-{\tau'} & \calo(V) \ar[u]_-{i_Y}
}
\]
したがって、任意の$f \in A(Y)$について
\begin{align*}
  f \circ \vp_V &= \tau'(f) \\
  i_Y(f \circ \vp_V) &= (\tau \circ i_X)(f)
\end{align*}
である。とくに$f \in \frakm_P \cap A(X)$ならば$\tau(\frakm_P) = \frakm_Q$により
\[
  i_Y(f \circ \vp_V) \in \frakm_Q
\]
が判る。$  i_Y(f \circ \vp_V) = \kakko{ V, f \circ \vp_V}$だから、とくに$f \circ \vp_V(Q) = 0$である。$f \in \frakm_P \cap A(X)$は任意だったから、$\vp_V(Q) \in X$は$P$に等しいことがわかる。

\item[Step 4]さて$\tau^{-1} \colon \calo_{Q,Y} \to \calo_{P,X}$も同型であるので、同様の議論を$\tau^{-1}$についてもできる。そうすれば、支配的有理写像$\psi \colon X \to Y$であって、ある$(U, \psi_U) \in \coprod_{P \in U} \Hom(U,Y)$によって代表されており、$\psi_U(P) = Q$を満たすようなものの存在がわかる。
このとき(1)により、$\psi^* \colon K(Y) \to K(X)$は$\wt{\tau^{-1}}$に等しく、$\vp^* \colon K(X) \to K(Y)$は$\wt{\tau}$に等しい。したがって$\psi$と$\vp$は互いに逆な支配的有理写像である。あとは系4.5と同様にすれば、示すべきことがいえる。
\item[Step 5] $X,Y$が一般の多様体のときは、点$P,Q$の開近傍であってアファインなものをとればアファイン多様体のときに帰着できる。
\end{description}


\bfsubsection{演習問題 4.8}
まず(b)は(a)を認めれば次の補題から従う。
\lem{
(多様体はNoether) \\
任意の多様体$Y$は、位相空間としてNoetherである。
}
\begin{proof}
  例1.4.7により、アファイン空間$\A^n$はNoetherである。また演習問題2.3(e)より、射影空間$\P^n$もまたNoetherであることがアファイン空間の場合と同様に示せる。よって、Noether空間の部分空間はNoetherである (演習1.7(d)) ことから、すべての多様体$Y$はNoetherである。
\end{proof}
\lem{
(一次元多様体の位相は補有限位相) \\
$Y$を1次元多様体とする。このとき、$Y$の位相は補有限位相である。つまり、真部分集合$F \subset Y$が閉部分集合となるのは$F$が有限集合であるとき、またそのときに限る。
}
\begin{proof}
  $Y$において1点集合は閉なので、真部分集合$F \clsub Y$が与えられたとして$\# F < \infty$を示せばよい。$Y$の既約性により、演習問題1.10(d)から$\dim F < \dim Y$である。よって$\dim F = 0$となるしかない。$Y$はNoetherなので既約分解(命題1.5)ができて、$F = \bigcup_{i=1}^n X_i$なる有限個の既約集合$X_i \clsub Y$がある。各$i$ごとに$P_i \in  X_i$をとる。$\{ P_i \} \clsub X_i$であり$X_i$は既約なので、再び演習問題1.10(d)から$X_i = \{ P_i\}$がいえる。(つまり一般に、0次元多様体は1点である)
  したがって、$F$は有限集合である。
\end{proof}
残る(a)を証明したい。
\textblue{以下は証明のアウトラインである。}
\begin{description}
  \item[Step 1] $\A^n$および$\P^n$の場合は、代数閉体$k$が無限体であることを踏まえた上で松坂\cite{松坂} 第3章5節定理12を参照のこと。
  \item[Step 2] 与えられた多様体$Y$の次元が$1$のとき。命題4.9により、$Y$は$\P^2$内のある超曲面$H$と双有理同値である。ここで$\dim Y = \dim H$だから、$\dim H = 1$である。系4.5により、ある$U \opsub Y$と$V \opsub H$が存在して、$U$と$V$は同型である。ところが$Y$と$H$に入っている位相は補有限位相なので、結局$\# H \geq \# k$を示せば、$\# Y \geq \# k$がいえることになる。
  逆はStep 1からあきらかだから、それで$\# Y = \# k$が従う。

  そこで全射$H \to \P^1$が欲しい。射影$\P^2 \sm \{ (1,0,0)\} \to \P^1$を考える。$H$のこの射影による像を考えたい。当然ながら$H$が$Q = \{ (1,0,0)\}$を含んでいるとまずい。
  もしも$Q \in H$ならば、(一般線形群の作用は推移的なので) 適当に点$W \in \P^2 \sm H$をとり、$AW = Q$なる$A \in GL_{3}(k)$をとってくる。すると$AH \cap \{ Q \} = A(H \cap W) = \emptyset$なので、$H$を$AH$で取り替えることにより(超曲面かどうかは議論が必要としても) 帰着させることができる。よって$Q \notin H$としてよい。

  射影の$H$への制限$H \to \P^1$を考える。これは、点$x = (t_0,t_1,t_2) \in H$に対して点$Q$と$x$を通るような$\P^2$内の直線
\[
\setmid{(t+st_0, st_1,  st_2)}{(t,s ) \in \P^1}
\]
  を考え、この直線と$\P^1 = \setmid{(0,s, t)}{(s,t) \in \P^1}$の交点$(0, t_1, t_2)$を返すような写像である。ここで$\P^1$の元$P = (0,t_1, t_2)$が与えられたとする。このとき、$P$と点$Q$を通る直線と$H$は交点を持つ。(演習問題3.7による。詳細は省略。)よって、射影$H \to \P^1$は全射である。ゆえに示すべき事がいえた。
  \item[Step 3] $\dim Y \geq 2$のとき。$Y \loc \P^n$としてよい。1次元部分多様体$H \clsub Y$がある。(本当か?) したがって$\# Y \geq \# k$がわかる。逆はあきらかなので結論がしたがう。
\end{description}


\newpage

\bfsection{1.5 非特異多様体}

\begin{comment}
\bfsubsection{定義-非特異}
\barquo{
$Y \subset \A^n$をアファイン多様体、$f_1, \cdots , f_t \in A = k[x_1, \cdots , x_n]$を$Y$のイデアルの生成元集合とする。$r$を$Y$の次元として、行列$\norm{(\del f_i)/(\del x_j)(P) }$の階数が$n-r$となるとき$Y$は点$P$で非特異であるという。
}
\begin{rem}
  $n-r$という数字がどこから出てきたか説明する。まず$r=\dim Y$により、
  \begin{align*}
    r &= \dim Y \\
    &= \dim A(Y) \\
    &= \dim A/I(Y) \\
    &= \coht I(Y)
  \end{align*}
  である。ここで松村\cite{松村}演習問題5.1により体上の有限生成多項式環$A$とその素イデアル$\frakp$について、$\dim A = \height \frakp + \coht \frakp$が成り立つ。したがって$\height I(Y) = \dim A - \coht I(Y) = n-r$である。

  とくに、$A$はNoether環なので、Krullの標高定理 (松村\cite{松村}定理13.5) から$n-r = \height I(Y) \leq t$が従うことに注意する。
\end{rem}
\end{comment}


\bfsubsection{定義-非特異 直後}
\barquo{
非特異性の定義が$Y$のイデアルの生成元集合の選び方によらないことは容易に示すことができる。
}
\begin{rem}
  これは後続の定理5.1の証明のなかで示されることだが、次の補題によっても示される。
\end{rem}

\lem{
(アファイン空間のZariski接空間) \\
$Y \subset \A^n$が$r$次元アファイン多様体であるとする。$A=k[x_1, \cdots , x_n]$とおく。$I(Y) \subset A$の生成元$f_1, \cdots , f_t$が与えられているとする。ここで$F \colon \A^n \to \A^t$を
\[
F(x) = (f_1(x), \cdots , f_t(x))
\]
で定めると、点$p \in \A^n$でのヤコビアン$JF_p \colon \A^n \to \A^t$が、$t \tm n$行列として
\[
JF_p = \left( \left(\f{\del f_i}{\del x_j} \right) (p) \right)_{1 \leq i \leq t, 1 \leq j \leq n}
\]
により定まる。このとき、
\[
T_p Y = \setmid{v \in \A^n}{\forall g \in I(Y) \quad \sum_{j = 1}^n v_j \f{\del g}{\del x_j}(p) = 0 }
\]
とおくと任意の$p \in Y$について$\Ker JF_p = T_p Y$が成り立つ。とくに、$JF_p$の階数は$f_1, \cdots , f_t$の取り方によらない。
}

\begin{proof}
  この本では定義が違うが、$T_pY$でZariski接空間を定義する本もある。たとえばHarris\cite{Harris}Lecture14を参照のこと。証明であるが、$T_p Y \subset \Ker JF_p$はあきらかであるので逆を示そう。$v \in \Ker JF_p$と$g \in I(Y)$とが与えられたとする。$f_1, \cdots , f_t$は$I(Y)$を生成するので
  \[
  g= \sum_{i=1}^t a_i f_i
  \]
  なる$a_i \in A$がある。よって
  \begin{align*}
    \f{\del g}{\del x_j} &= \sum_{i=1}^t  \f{\del a_i}{\del x_j} f_i + \sum_{i=1}^t   a_i \f{\del f_i}{\del x_j} \\
\f{\del g}{\del x_j}(p) &= \sum_{i=1}^t   a_i(p) \f{\del f_i}{\del x_j}(p)  &(f_i \in I(Y)\text{による}) \\
\sum_{j = 1}^n v_j \f{\del g}{\del x_j}(p) &= \sum_{i=1}^t a_i(p) \sum_{j=1}^n v_j \f{\del f_i}{\del x_j}(p) \\
&= 0
  \end{align*}
  が成り立つ。$g \in I(Y)$は任意だったので$v \in T_p Y$がわかる。これで示すべきことがいえた。
\end{proof}


\bfsubsection{定理 5.1}
\barquo{
$Y \subset \A^n$をアファイン多様体とする。$P \in Y$を点としよう。このとき、$\calo_{P,Y}$が正則局所環である場合、またその場合に限って$Y$は$P$で非特異になる。
}
\begin{proof}
  定理3.2で、$\calo_{P,Y}$がNoether局所環であり$\dim Y = \dim \calo_{P,Y}$が成り立つということはすでに示されている。$\frakm$を$\calo_{P} = \calo_{P,Y}$の極大イデアルとする。$A=k[x_1, \cdots, x_n]$とおく。$I(Y) \subset A$を$\frakb = I(Y)$と書く。$\frakb$の生成元集合$f_1, \cdots , f_t$をひとつとっておき、点$P$におけるJacobianを$t \tm n$行列として
  \[
  J = \left( \left( \f{\del f_i}{\del x_j} \right) (P) \right)_{1 \leq i \leq t,1 \leq j \leq n}
  \]
  と定めておく。このとき、
  \[
  \dim_k \frakm / \frakm^2 = n - \rank J
  \]
  が成り立つことを示せば十分である。これさえ示せば、あとは定理3.2から望みの同値性がいえる。

 $\grt, \grt'$の構成: $P = (a_1, \cdots, a_n)$とおいて、$\fraka = (x_1 - a_1, \cdots , x_n - a_n) \subset A$を$P$に対応する極大イデアルとする。$k$線形写像$\grt \colon A \to k^n$を
  \[
  \grt(f) = \left( \f{\del f}{\del x_1} , \cdots, \f{\del f}{\del x_n} \right)
  \]
  で定める。すると$\grt(x_i - a_i)$が$k^n$の基底をなすので$\grt$は全射。$\grt (\fraka^2) = 0$を満たすので、全射$k$線形写像$\grt' \colon \fraka / \fraka^2 \to k^n$が誘導される。ここで
  \[
\fraka / \fraka^2 \cong \bigoplus_{i=1}^n (x_i - a_i)k
  \]
  だから、$\dim_k \fraka / \fraka^2 = n$であり、したがって$\grt'$は同型。


  $\dim_k \grt(\frakb) = \dim_k (\frakb + \fraka^2)/ \fraka^2$であること: 次の図式
\[
\xymatrix{
\fraka \ar[rd]^{\grt|_{\fraka}} \ar[d] & { } \\
\fraka / \fraka^2 \ar[r]^{\grt'} & k^n
}
\]
  は可換なので$\grt(\frakb) = \grt|_{\fraka}(\frakb) = \grt'( (\frakb + \fraka^2)/ \fraka^2 )$を得る。$\grt'$は同型なので$\grt(\frakb) \cong (\frakb + \fraka^2)/ \fraka^2 $がわかる。とくにベクトル空間としての次元も等しい。


$\rank J = \dim_k \grt(\frakb)$であること: $g \in \frakb$とする。$f_1, \cdots , f_t$は$\frakb$を生成するので
\[
g= \sum_{i=1}^t a_i f_i
\]
なる$a_i \in A$がある。ゆえに$e_1, \cdots , e_n$を$k^n$の標準的な基底とすると、
\begin{align*}
  \grt(g) \cdot e_j &= \grt \left( \sum_{i=1}^t a_i f_i \right) \cdot e_j \\
  &= \f{\del}{\del x_j} \left.\left( \sum_{i=1}^t a_i f_i \right)  \right|_{x=P} \\
  &= \sum_{i=1}^t a_i(P) \f{ \del f_i}{\del x_j}(P) \\
  \grt(g) &= \sum_{i=1}^t a_i(P) \grt(f_i)
\end{align*}
であることがわかる。ゆえに$k^n$の中で
\[
\grt(\frakb) = \sum_{i=1}^t k \cdot \grt(f_i) = \Im {}^tJ
\]
である。とくに$\rank J = \rank {}^tJ = \dim_k \grt(\frakb)$が結論できる。

$\dim_k \frakm / \frakm^2 = \dim_k (\fraka /  (\frakb + \fraka^2 ))$であること: $\calo_P \cong A(Y)_{\ol{\fraka}} \cong A_{\fraka}/ \frakb A_{\fraka}$である。よって$\frakm = \fraka A_{\fraka}/ \frakb A_{\fraka}$と同一視できる。ここで、自然な写像$\pi \colon A_{\fraka} \to A_{\fraka}/ \frakb A_{\fraka}$
は全射なので
\begin{align*}
  \frakm^2 &= \pi(\fraka A_{\fraka})^2 \\
  &= \pi(\fraka^2 A_{\fraka} ) \\
  &=   (\frakb + \fraka^2 ) A_{\fraka}/ \frakb A_{\fraka}
\end{align*}
がいえる。したがって
\begin{align*}
\frakm / \frakm^2 &= \fraka A_{\fraka}/ (\frakb + \fraka^2) A_{\fraka} \\
&= \Coker( (\frakb + \fraka^2) \ts_A A_{\fraka} \to \fraka \ts_A A_{\fraka} \to \fraka A_{\fraka} ) \\
&= \Coker( (\frakb + \fraka^2) \ts_A A_{\fraka} \to \fraka \ts_A A_{\fraka} ) &(\text{局所化の平坦性}) \\
&= \Coker ( (\frakb + \fraka^2)  \to \fraka ) \ts_A A_{\fraka} &(\text{テンソル積の右完全性}) \\
&= (\fraka / (\frakb + \fraka^2 )) \ts_A  A_{\fraka} \\
&=  (\fraka / (\frakb + \fraka^2 )) \ts_k k  \ts_A  A_{\fraka}
\end{align*}
である。ここで
\begin{align*}
  k \ts_A A_{\fraka} &= (A/ \fraka) \ts_A A_{\fraka} \\
  &= A_{\fraka} / \fraka A_{\fraka} \\
  &= \Frac A/\fraka \\
  &= k
\end{align*}
であるから、$\dim_k \frakm / \frakm^2 = \dim_k (\fraka /  (\frakb + \fraka^2 ))$がいえる。

結論: 以上の議論により
\begin{align*}
  \dim_k \frakm / \frakm^2 &= \dim_k (\fraka /  (\frakb + \fraka^2 ) ) \\
  &=  \dim_k ( (\fraka / \fraka^2) /  ((\frakb + \fraka^2) / \fraka^2 ) ) \\
  &= \dim_k ((\fraka / \fraka^2) ) - \dim_k ( (\frakb + \fraka^2) / \fraka^2 ) \\
  &= n - \rank J
\end{align*}
がわかる。

\end{proof}

\begin{rem}
  命題5.2Aまたは雪江\cite{雪江3}命題2.10.5により、ネーター局所環$(R,\frakm,k)$があるとき$\dim R \leq \dim_k \frakm / \frakm^2$が成り立つ。したがって
  \begin{align*}
    \rank J &= n - \dim_k \frakm / \frakm^2 \\
    &\leq n - \dim \calo_P \\
    &= n - \dim Y
    \end{align*}
    であることがわかった。ゆえに非特異性の定義は、Jacobianの階数が最大値をとることを意味する。
\end{rem}





\bfsubsection{定理 5.3}
\barquo{
$Y$を多様体とする。このとき$Y$の特異点の集合$\Sing Y$は$Y$の真部分閉集合である。
}
\begin{proof} ${}$
  \begin{description}
    \item[Step 1] $Y$がアファイン多様体のときに、$\Sing Y \clsub Y$であることを示そう。$Y \subset \A^n$とみなす。$r = \dim Y$とおく。$I(Y)$の生成元$f_1 , \cdots , f_t$をとり、$F \colon k^n \to k^t$を$F(x) = (f_1(x) , \cdots , f_t(x))$で定める。このとき
    \[
    \Sing Y = \setmid{y \in Y}{ \rank JF_y < n -r}
    \]
    である。ここで、次が成り立つことを思いだそう。
\prop{
$A$は$m \tm n$型の体係数の行列であるとする。このとき$A$の正方部分行列であって行列式が$0$でないものの最大の次数と、$A$の階数は一致する。
}
\begin{proof}
  齋藤\cite{齋藤}定理3.2.15を参照のこと。
\end{proof}
したがって、$\Sing Y$は$Y$の部分集合として、ヤコビアン$JF_y$の$(n-r) \tm (n-r)$型小行列式の零点として定義されるような代数的集合である。ゆえに$\Sing Y \clsub Y$がわかる。

\item[Step 2] $Y$が一般の多様体であるときに、$\Sing Y \clsub Y$であることを示そう。命題4.3により、開アファイン部分集合$Y_i$による$Y$の被覆
$Y = \bigcup_i Y_i$がある。各$i$についてアファイン多様体$Z_i$への同型$\vp_i \colon Y_i \to Z_i$をとることができる。このとき
\[
\vp_i (\Sing Y \cap Y_i) \subset Z_i
\]
である。先に進むために、次の補題を示そう。


\lem{
(開部分多様体の特異点) \\
$Y$は多様体、$U$は$Y$の空でない開部分集合とする。このとき$U$は$Y$の部分多様体で、
\[
\Sing Y \cap U = \Sing U
\]
が成り立つ。
}
\begin{proof}
  局所環は開集合に制限しても不変であることに注意する。$x \in Y$に対して、
  \begin{align*}
    x \in \Sing U &\iff \calo_{x,U} \text{は正則局所環でない} \\
    &\iff \calo_{x,Y} \text{は正則局所環ではなく、かつ} x \in U  \\
    &\iff x \in \Sing Y \cap U
  \end{align*}
  である。したがって、示すべきことがいえた。
\end{proof}

定理5.3の証明に戻る。$\vp_i (\Sing Y \cap Y_i) \subset Z_i$について考えていたのだった。ここで、補題を使って
\begin{align*}
  \vp_i (\Sing Y \cap Y_i) &= \vp_i ( \Sing Y_i) \\
  &= \Sing (\vp(Y_i)) \\
  &= \Sing (Z_i)
\end{align*}
である。アファイン多様体の場合の考察により、$\Sing Z_i \clsub Z_i$なので、$\vp_i (\Sing Y \cap Y_i) \clsub Z_i$がいえた。$\vp_i$は同型なので引き戻して$\Sing Y \cap Y_i \clsub Y_i$である。ゆえに、閉集合の局所判定補題から$\Sing Y \clsub Y$がわかった。
\item[Step 3] $Y$がアファイン空間に埋め込まれた超曲面であるときに、$\Sing Y \subsetneq Y$を示そう。$Y = Z(f) \subset \A^n$とする。$A =  k[x_1, \cdots , x_n]$とする。ここで$f \in A$は既約元である。
%$\A^n$は非特異多様体なので、$Y = \A^n$ならば$\Sing Y = \emptyset$であり示すべきことはない。したがって$\emptyset \subsetneq Y \subsetneq \A^n$の場合だけ考えればよく、
とくに$f$は定数でない。このとき命題1.13により$\dim Y = n-1$なので、$n - \dim Y = 1$
である。したがって$I(Y) = ( f )$であることから、
\[
\Sing Y  = \setmid{P \in Y}{\forall i \quad  \f{\del f}{\del x_i}(P) = 0 }
\]
が成り立つ。ここで$Y = \Sing Y$であると仮定しよう。すると$\f{\del f}{\del x_i} \in I(Y)$であることになる。$I(Y)$は$f$で生成される単項イデアルなので、次数を考えて$\f{\del f}{\del x_i} = 0$でなくてはならない。$f$は定数でないので、体$k$の標数は$0$ではない。そこで$k$の標数を$p > 0$とする。$ \f{\del f}{\del x_i} = 0$
であるから、すべての$i$に対して、$f$の$x_i$のベキは$p$の倍数である。したがってある多項式$g \in A$が存在して、$f(x_1 , \cdots , x_n) = h(x_1^p, \cdots , x_n^p)$を満たす。$k$は代数閉体なので、$h$の係数の$p$乗根を係数に持つような多項式$g \in A$がとれる。$p$乗写像は環準同形になっているので、このとき$f = g^p$が成立する。これは$f$が既約であるという仮定に矛盾。したがって$\Sing Y \neq Y$である。
\item[Step 4] $Y$が一般の多様体の場合に、$\Sing Y \subsetneq Y$を示そう。$r = \dim Y$とする。命題4.9の証明から、$Y$に双有理同値であるようなアファイン空間内の超曲面$H \subset \A^{r+1}$が存在する。系4.5により双有理なふたつの多様体は、同型な開部分多様体をもつ。したがってある$U \opsub Y$と$V \opsub H$が存在し、$U \cong V$である。$\psi \colon U \to V$を同型を与える射とする。このとき
\begin{align*}
  Y \setminus \Sing Y &\supset U \cap (  Y \setminus \Sing Y) \\
  &= U \setminus \Sing U \\
  &= \psi^{-1}(V \setminus \Sing V) \\
  &= \psi^{-1}((H \setminus \Sing H) \cap V)
\end{align*}
  ここで(3)により$H \setminus \Sing H \opsub H$は空でない。したがって、$H$の既約性により$(H \setminus \Sing H) \cap V \neq \emptyset$である。ゆえに$Y \setminus \Sing Y \neq \emptyset$がわかる。
 \end{description}
\end{proof}




\bfsubsection{定理 5.3}
\barquo{
$\Sing Y$が$Y$の真部分集合であることを示すためにまず(4.9)を適用すると、$Y$は$\P^n$の中の超曲面と双有理である。双有理な多様体は同型な開部分集合を持つから、超曲面の場合に帰着される。$Y$の任意の開アファイン部分集合について考えれば十分なので、$Y$は$\A^n$内で一つの既約多項式を用いて$f(x_1, \cdots , x_n)=0$と定義される超曲面であると仮定してよい。
}
\begin{rem}
  \textblue{この部分は意図不明。}なぜ、アファイン空間内の超曲面の場合に帰着するのに射影空間内の超曲面を経由しようとおもったのだろう。
\end{rem}


\bfsubsection{例 5.6.3}
\barquo{
(可約代数的集合上の点の局所環についてはまだ一般論を展開していないので、暫定的な定義として$\calo_{O,Y} = (k[x,y]/(xy))_{(x,y)}$を使う。したがって$\wh{\calo}_{O,Y} \cong k[[x,y]]/(xy)$である。) この例は、$X$が$O$の近くでは二つ直線の交わったもののように見える、という幾何学的事実に対応する。

この結果を証明するために完備化$\wh{\calo}_{O,X}$を考えるが、これは$k[[x,y]]/ (y^2 - x^2 - x^3)$と同型である。
}
\begin{proof} ${}$
  \begin{description}
    \item[Step 1]   $A = k[x,y]$, $J= xy A$, $\frakm = xA + yA$とする。以下、$A$加群$M$の$\frakm$進完備化をハットをつけて$\wh{M}$のように表す。$\wh{A} = k[[x,y]]$である。
      $(A/J)_{\ol{\frakm}} = A_{\frakm} / J A_{\frakm}$の極大イデアル$\frakm A_{\frakm}/ J A_{\frakm}$による完備化が$\wh{A}/J\wh{A}$と同型であることを示そう。

      愚直に計算していくと
      \begin{align*}
    \llim_{n} (A_{\frakm} / J A_{\frakm}) / ((\frakm^n + J) A_{\frakm}/ J A_{\frakm})
    &= \llim_{n} A_{\frakm} / (\frakm^n + J) A_{\frakm} \\
    &= \llim_{n} (A/(\frakm^n + J) \ts_A A_{\frakm})
      \end{align*}
      がわかる。ここで、$A/ (\frakm^n + J)$の素イデアルは、$\frakm^n + J$を含む$A$の素イデアル$P$と一対一に対応するが、$P$は素イデアルなので$P \supset \frakm$, したがって$P = \frakm$でなくてはならない。ゆえに$A/ (\frakm^n + J)$は$\frakm / (\frakm^n + J)$を極大イデアルとする局所環であり、ゆえに$A/(\frakm^n + J) \ts_A A_{\frakm} = A/(\frakm^n + J)$である。よって
      \begin{align*}
      \llim_{n} (A/(\frakm^n + J) \ts_A A_{\frakm}) &= \llim_{n} A/(\frakm^n + J) \\
      &= \llim_{n} (A/J) / (\frakm^n + J/ J )
      \end{align*}
      であるので、これは$A/J$の$\frakm$進完備化に等しい。$A$はNoether環で、$A/J$は有限生成$A$加群なので
      \begin{align*}
        \llim_{n} (A/J) / (\frakm^n + J/ J ) &= (A/J) \ts_A \wh{A} \\
        &= \wh{A} / J \wh{A}
      \end{align*}
      \item[Step 2] まず$X$がアファイン多様体であることをいいたい。それには$y^2 -x^2 - x^3$の既約性を示せばよい。次のよく知られたEisensteinの既約性判定法を使う。
      \lem{
      (Eisenstein判定法) \\
      $A$は環、$f \in A[x]$はモニック多項式で、
      \[
      f(x) = x^n + a_{n-1} x^{n-1} + \cdots + a_0
      \]
      と表されているとする。このとき素イデアル$\frakp \subset A$であって$a_{n-1}, \cdots , a_0 \in \frakp$かつ$a_0 \notin \frakp^2$なるものがあれば、$f \in A[x]$は既約である。
      }
      \begin{proof}
        松村\cite{松村}\S 29, 補題1 を参照のこと。
      \end{proof}
      $y^2 -x^2 - x^3$の既約性の話に戻る。$A = k[x]$とし、$\frakp = (x+1)$としてEisensteinの既約性判定法を使えば、$y^2 -x^2 - x^3 \in A[y]=k[x,y]$が既約であることがわかる。よって$\wh{\calo}_{O,X} \cong k[[x,y]]/ (y^2 - x^2 - x^3)$がStep 1と同様に示せる。
  \end{description}
\end{proof}




\bfsubsection{例 5.6.3}
\barquo{
したがって$\wh{\calo}_{O,X} = k[[x,y]]/(gh)$である。$g$と$h$は線形独立な線形項で始まるから、$k[[x,y]]$の自己同形であって$g$と$h$をそれぞれ$x$と$y$に送るものがある。これで求める$\wh{\calo}_{O,X} \cong k[[x,y]]/(xy)$が示された。
}
\begin{proof} ${}$
  \begin{description}
    \item[Step 1] $\wh{\calo}_{O,X} = k[[x,y]]/(gh)$であるような$g,h \in k[[x,y]]$があることを示そう。次の補題を使う。
    \prop{
    (Henselの補題) \\
    $A$は完備な離散付値環で、$\frakm$をその極大イデアル、$\ord$を$\frakm$から定まるオーダー(加法的付値)とする。$f(x) \in A[x]$であるものとする。$a \in A$があり、
    \[
    \ord(f(a)) > 2\ord (f'(a))
    \]
    が成りたつとする。このときある$b \in A$が存在して
    \[
    f(b)=0, \quad b \equiv a \mod \frakm^{\ord (f'(a)) + 1}
    \]
    を満たす。
    }
  \end{description}
  \begin{proof}
    雪江\cite{雪江3} 定理3.3.1を参照のこと。
  \end{proof}
$y^2 - x^2 - x^3$の既約性の話に戻る。$A = k[[x]]$とおく。$A$は完備であり、離散付値環の構造を持つ。よってHenselの補題を使うことができる。$f= z^2 - (x+1) \in A[z]$を考える。このとき
\[
f(1 ) = -x, \quad f'(1) = 2
\]
である。体$k$の標数は$2$でないという仮定があったので、$f$はHenselの補題の仮定を満たす。ゆえに、ある$d \in A$が存在して$d^2=x+1$, $d \equiv 1 \mod (x)$を満たす。このとき$y^2 - x^2 - x^3 = (y + xd)(y - xd)$と分解できる。よって$g= y + xd$, $h = y - xd$とおけば引用部の前半が示せたことになる。

\item[Step 2] $k[[x,y]]$の自己同形であって$g$と$h$をそれぞれ$x$と$y$に送るものがあることを示そう。次の命題による。
\prop{
(べき級数環での逆関数の定理) \\
$A$はネーター環、$A[[x]] = A[[x_1, \cdots, x_n]]$, $\frakm = (x_1, \cdots , x_n)$, $f_1(x), \cdots , f_n(x) \in \frakm$とする。$F \colon A[[x]] \to A[[x]]$を$F(x_i) = f_i$なる準同形とする。($f_i \in \frakm$なのでこれはwell-definedである) このとき
\[
\det JF_{O} \in A^{\tm}
\]
ならば、$F$は同型である。
}
\begin{proof}
  雪江\cite{雪江3} 定理3.4.4を参照のこと。
\end{proof}
引用部の証明に戻る。$F \colon k[[x,y]] \to k[[x,y]]$を
\[
F(x) = y + xd, \quad F(y) = y - xd
\]
とおく。$y + xd$, $y - xd \in (x,y)$よりこれはwell-definedで、$F$は環準同型を定める。$d \equiv 1 \mod (x)$に気をつけて計算すると
\begin{align*}
  JF_{(x,y)} &= \begin{pmatrix}
   d(x) + xd'(x) & 1 \\
  -d(x) - xd'(x) & 1
\end{pmatrix} \\
\det JF_{(0,0)} &= \det \begin{pmatrix}
1 & 1 \\
-1 & 1
\end{pmatrix} \\
&= 2
\end{align*}
を得る。したがって、$k$の標数は$2$ではないので、$F$は同型である。ゆえに$k[[x,y]] = k[[g,h]]$であり、$y^2 - x^2 - x^3 = gh$より$k[[x,y]] /(xy) \cong k[[g,h]]/ (gh) = k[[x,y]]/ (gh)$がわかる。
\end{proof}


\newpage

\bfsection{1.6 非特異曲線}

\bfsubsection{補題 6.4}
\barquo{
$Y$を準射影多様体、$P,Q \in Y$として、$K(Y)$の部分環として$\calo_Q \subset \calo_P$とする。このとき$P = Q$である。
}
\begin{proof} 準射影多様体と書いてあるが、すべての多様体$Y$について示すことにする。すべての多様体はある準射影多様体に同型なので、(確かめよ) これは本質的な変更ではないだろうと思う。
  \begin{description}
    \item[Step 1] $Y$がアファイン多様体の場合。座標環$A(Y)$を単に$A$と書く。極大イデアル$\frakm$, $\frakn \subset A$をそれぞれ$P,Q$に対応するものとすると、$\calo_P = A_{\frakm}$, $\calo_Q = A_{\frakn}$である。仮定により$A_{\frakn} \subset A_{\frakm}$である。したがって特に$x \mapsto x$は準同形$A_{\frakn} \to A_{\frakm}$を定める。(これが単射であることは証明に使わない)
    $x \in A \setminus \frakn$としよう。$x \in A_{\frakn}$は単元なので、$x \in A_{\frakm}$
    だと思っても単元である。よって$x \in A \setminus \frakm$であり、このことから$\frakm \subset \frakn$がわかる。ところが$\frakm$は極大イデアルなので$\frakm = \frakn$である。したがって命題3.2(b)より$P = Q$である。
    \item[Step 2] $Y$が射影多様体の場合。$Y \subset \P^n$とおく。$\P^n$に対する線形作用を考えよう。
    \lem{
    ($\P^n$への線形作用) \\
    $A \in GL_{n+1}(k)$とする。$P \in \P^n$に対し、代表元$Q \in \A^{n+1} \setminus \{0\}$をとり、$AP \in \P^n$を$AQ$により代表される同値類として定める。これはwell-definedであり、これにより$\P^n$に群$GL_{n+1}(k)$の作用が定まる。かつ、$A \colon \P^n \to \P^n$とみなすと$A$は多様体の圏の同型射である。
    }
    \begin{proof}
$A$は正則行列なので、$AQ \in  \A^{n+1} \setminus \{0\}$である。また、$A$は線形写像なので、$AQ \in \P^n$は代表元$Q$のとりかたによらない。したがってこの作用はwell-definedである。群作用になっていることはあきらか。$A \colon \P^n \to \P^n$は多項式によって定義される写像なので射であり、可逆なので同型射。
    \end{proof}
    \lem{
    $P,Q  \in \P^n$とする。$U_0 \subset \P^n$を$x_0 \neq 0$により定まる開部分集合とする。このとき
    \[
    AP \in U_0 , \quad AQ \in U_0
    \]
    となるような$A \in GL_{n+1}(k)$が存在する。
    }
    \begin{proof}
背理法による。すべての$A \in GL_{n+1}(k)$に対して$  AP \notin U_0$または$  AQ \notin U_0$が成り立つと仮定する。代表元をとり$P = (p_0, \cdots , p_n)$, $Q = (q_0, \cdots , q_n)$と表す。$I = \setmid{i}{p_i \neq 0}$, $J = \setmid{j}{q_j \neq 0}$とおく。

ベクトルの$i$行目と$j$行目を入れ替える操作を表す行列$D_{ij} \in GL_{n+1}(k)$を考えると、とくに
\[
\forall i \quad D_{i0}P \notin U_0  \; \text{または} \;  D_{i0}Q \notin U_0
\]
が成りたつ。したがってすべての$i$について$i \notin I$または$i \notin J$である。
そこでベクトルの$i$行目に$j$行目の$c \in k$倍を加える操作を表す行列を$E_{ij}(c)$とすると、$E_{ij}(c) \in GL_{n+1}(k)$であり、任意の$i \in I$と$j \in J$に対して
\begin{align*}
  &D_{j0} E_{ji}(1) P \in U_0 \\
  &D_{j0} E_{ji}(1) Q = D_{j0}Q \in U_0
\end{align*}
となるので背理法の仮定に矛盾。
    \end{proof}
    引用部の証明に戻る。$AP \in U_0 ,  AQ \in U_0$なる$A \in GL_{n+1}(k)$をとり、$V = A^{-1}U_0$とする。このとき$V \opsub \P^n$は$P,Q$の共通の開近傍であり、またアファイン多様体$\A^n$と同型である。
    $\calo_{P,Y} = \calo_{P,V \cap Y}$, $\calo_{Q,Y} = \calo_{Q,V \cap Y}$なので同型$V \to \A^n$をとればアファインの場合に帰着できて、$P = Q$がいえる。
    \item[Step 3] $Y$が準射影多様体または準アファイン多様体の場合。$Y \opsub \ol{Y}$であるから、$\calo_{P,Y} = \calo_{P,\ol{Y}}$, $\calo_{Q,Y} = \calo_{Q,\ol{Y}}$が成り立ち、既に示した場合に帰着される。
  \end{description}
\end{proof}


\bfsubsection{補題 6.5}
\barquo{
$\frakn = \frakm_R \cap B$としよう。すると$\frakn$は$B$の極大イデアルであり、
}
\begin{rem}
  次の補題をまず用意する。
  \lem{
  $A$は整域、$K$はその商体、$L$は$K$の代数拡大体とし、$B$は$L$における$A$の整閉包であるとする。このとき、$B$の商体は$L$である。
  }
  \begin{proof}
    あきらかだが、一応示す。$x \in L$とする。$L/K$は代数拡大なので、
    \[
    x^m + a_1 x^{m-1} + \cdots + a_m = 0
    \]
    なる$a_1 , \cdots , a_m \in K$がある。$A$の商体が$K$なので、ある$d \in A \sm \{0\}$であって、任意の$i$について$da_i \in A$となるようなものがとれる。このとき
    \[
    (dx)^m + a_1d (dx)^{m-1} + \cdots + a_md^m = 0
    \]
    だから、$dx$は$A$上整。よって、$B$の定義により$dx \in B$である。以上により、$B$の商体が$L$であることがいえた。
  \end{proof}
  引用部の証明に戻る。
$B$はDedekind環なので、$\frakn \neq (0)$を示せば十分である。背理法による。$\frakn = \frakm_R \cap B = 0$と仮定しよう。$x \in K$が与えられたとする。これは$B \sm \{0\} \subset R \sm \frakm_R$を意味する。上の補題により、$\Frac B = K$だから、$bx \in B$なる$b \in B \sm \{0\}$がある。
このとき、$b \in R^{\tm}$なので、$x \in R$である。ゆえに$R = K$であるが、これは$R$がDVRであるという仮定に反する。
\end{rem}

\bfsubsection{補題 6.5}
\barquo{
$B$は$R$に支配される。ところが$B_{\frakn}$も$K/k$の離散付値環だから、
}
\begin{proof}
  \textblue{誤植であろう。} $B$とあるのは、おそらく$B_{\frakn}$の間違い。証明は、
  \begin{align*}
    (\frakm_R \cap B_{\frakn}) \cap B &= \frakm_R \cap B = \frakn \\
    \frakn B_{\frakn} \cap B &= \frakn
  \end{align*}
  から、$\frakm_R \cap B_{\frakn} = \frakn B_{\frakn}$がいえる。

  $B$はDedekind環なので、$B_{\frakn}$はDVR、とくに付値環である。あと$\Frac B = K$を示す必要があるが、それは既に示した。
\end{proof}

\bfsubsection{系 6.6}
\barquo{
$K/k$の任意の離散付値環は、ある非特異アファイン曲線上のある点の局所環に同型である。
}
\begin{rem}
  $K/k$の離散付値環$R$が与えられたとする。このとき補題6.5の証明で構成した$B$と同型なアファイン座標環をもつ多様体$Y$がある。ここで、$\Frac B = K$なので、$Y$の函数体が$K$となっていることを注意しておく。
\end{rem}


\bfsubsection{系 6.6 直後}
\barquo{
$C_K$が無限集合であることに注意せよ。これは、$C_K$が$K$を函数体に持つ任意の非特異曲線の局所環をすべて含んでおり、またこれらの局所環はすべて異なっていて(6.4) かつ無限個ある(Ex.4.8)からである。
}
\begin{rem}
  $C_K$が無限集合であることを示すだけなら演習問題4.8を使う必要はない。つまり、1次元多様体$Y$が与えられたとき、$Y$には補有限位相が入っているので、$Y$の既約性より$Y$が有限集合ではないことがわかる。
\end{rem}


\bfsubsection{系 6.6 直後}
\barquo{
$U \subset C_K$が$C_K$の開部分集合のとき、$U$上の正則函数の環を$\calo(U) = \bigcap_{P \in U} R_P$と定める。元$f \in \calo(U)$は、$R_P$の極大イデアルを法とした$f$の剰余を$f(P)$とすることにより$U$から$k$への函数を定める。
}
\begin{rem}
  この定義から、抽象非特異曲線$X = C_K$の点$P \in X$での局所環は$R_P$そのものであることがわかる。また、$f$が$X$の開集合$U$上の正則関数であるとき、補題6.5により$f \colon U \to k$は ($k$にザリスキ位相を入れて) $U$上の連続関数でもある。
\end{rem}



\bfsubsection{命題 6.7}
\barquo{
したがって、(4.3)により$Y$はアファインであるとして良い。
}
\begin{proof}
  アファインの場合に示せたとして、任意の1次元非特異多様体$Y$について示そう。

  $\vp_Y \colon Y \to C_K$を$\vp_Y(P) = \calo_{P,Y}$なる写像とする。$U \subset C_K$を$\vp_Y$の像とする。
  命題4.3により、空でないアファイン開集合$V \opsub Y$がある。$V$での局所環と$Y$での局所環は同型なので$V$は非特異多様体であり、演習3.12から$\dim V = \dim Y = 1$である。したがって、$\vp_Y$の$V$への制限を$\vp_V \colon V \to C_K$とし、$W$をその像とすると$W \opsub C_K$である。したがって$W \subset U$より、$C_K$の位相の定義より$U \opsub C_K$である。
\end{proof}




\bfsubsection{命題 6.7}
\barquo{
これらの局所環はすべて離散付値環であるから、$U$は実際には$A$を含む$K/k$の離散付値環すべてからなる。
}
\begin{proof}
  $A$を含む$K/k$のDVR, $R$が与えられたとする。$R = \calo_{P,Y}$なる$P \in Y$があることを示そう。$\frakn = \frakm_R \cap A$とおく。もし$\frakn = 0$ならば、$A \sm \{0\} \subset R^{\tm}$である。$K$は$A$の商体なので、$R = K$となり矛盾。
  よって$\frakn \neq 0$であるので、$\dim A = 1$より$\frakn \subset A$は極大イデアル。$Y$は非特異多様体であり、$A$の極大イデアルは$Y$の点に対応することから$A_{\frakn}$は正則局所環。したがって定理6.2Aより$A_{\frakn}$はDVRである。あきらかに$A_{\frakn} \subset R$であり、$R$は$A_{\frakn}$を支配する。よって付値環の極大性より$A_{\frakn} = R$である。
\end{proof}



\bfsubsection{命題 6.7}
\barquo{
$\vp$が同型であることを示すには、任意の開集合上で正則函数が同じものであることを確かめればよい、ところがこれは、$U$上の正則函数の定義および、任意の開集合$V \subset Y$について$\calo(V) = \bigcap_{P \in V} \calo_{P,Y}$であるという事実から従う。
}
\begin{proof}
  ここでは、$Y$はアファイン多様体という仮定は失効していることに気をつける。$\vp \colon Y \to U$と$\vp^{-1} \colon U \to Y$がともに射であることを確かめればよい。
  \begin{description}
    \item[Step 1] まず$\vp \colon Y \to U$が連続写像であることを示したいが、これはあきらかである。なぜなら$U$の閉集合とは、全体または有限集合であり、任意の多様体$Y$において1点集合は閉だからだ。
    \item[Step 2] $\vp \colon Y \to U$が射であることを示そう。$W \opsub U$と$W$上の正則函数$f$が与えられたとする。
    \[
\xymatrix{
Y \ar[r]^-{\vp} & U \\
\vp^{-1}(W) \ar[dr]_-{f \circ \vp} \ar[u] \ar[r]^-{\vp} & W \ar[u] \ar[d]^-f \\
{} & k
}
    \]
    まず
    \begin{align*}
      \calo(W) &= \bigcap_{R \in W} R \\
      &= \bigcap_{P \in \vp^{-1}(W)} \calo_{P,Y} \\
      &= \calo (\vp^{-1}(W))
    \end{align*}
    である。したがって、$W$上の正則函数を$\vp^{-1}(W)$上の有理関数だとみなすだけの写像$\calo(W) \to \calo(\vp^{-1}(W)) \st f \mapsto f^{+}$は同型である。ここで
    \begin{align*}
      f \circ \vp (P) &= f(\calo_{P,Y}) \\
      &= f^{+} \mod \frakm_P \\
      &= f^{+} (P)
    \end{align*}
    だから、$\vp$は射であり、かつ$\vp^{*} \colon \calo(W) \to \calo(\vp^{-1}(W))$は同型である。
    \item[Step 3] $\dim Y = 1$より$Y$の位相は補有限位相なので、$\vp \colon Y \to U$が閉写像であることはあきらか。よって$\vp^{-1}$は連続である。$\vp^{-1}$が射であることも、$\vp^*$が同型であったことからただちに従う。
  \end{description}
\end{proof}



\bfsubsection{命題 6.8}
\barquo{
このとき$\vp$が$X$から$\P^n$への射に延長されることを示せば十分である。というのは、$\P^n$への射に延長されれば像は必然的に$Y$に含まれるからである。したがって$Y = \P^n$の場合に帰着される。
}
\begin{proof}
  $Y = \P^n$のとき延長の存在と一意性が示せたとする。一般の場合に、多様体$Y$と射$\vp \colon X \sm P \to Y$が与えられたとする。$Y \to \P^n$との合成は延長することができ、$\ol{\vp} \colon X \to \P^n$を得る。延長の存在をいうために問題なのは$\ol{\vp}(P) \in Y$が成り立つかどうかである。なお、延長の一意性は$Y \to \P^n$が単射であることからただちに従う。

  $C_K$は無限集合であり、$X \opsub C_K$より(空ではないので)$X$も無限集合。したがって$X \sm P$を含むような$X$の閉部分集合は$X$自身しかない。したがって$X \sm P$の$X$における閉包は$X$に一致する。そこで、$\P^n$または$X$における閉包をオーバーラインで表すことにすると
  \begin{align*}
    \ol{\vp}(X) &= \ol{\vp}( \ol{X \sm P}) \\
    &\subset \ol{  \ol{\vp}(X \sm P) } \\
    &= \ol{Y} \\
    &= Y
  \end{align*}
  である。よって$\ol{\vp} \colon X \to Y$とみなすことができる。
\end{proof}



\bfsubsection{命題 6.8}
\barquo{
一意性は構成によりあきらかである((4.1)からも従う)。
}
\begin{rem}
  補題4.1では多様体の間の射が空でない開集合上で一致していれば元々等しいことを主張していた。しかし、ここでは$X$は抽象非特異曲線であり、これが今までの意味での「多様体」と同じと見なせるかどうかはわからない。そこで補題4.1の証明を見てみると、射$\vp \colon X \to \P^n$と$\psi \colon X \to \P^n$があったとき、射$\vp \tm \psi \colon X \to \P^n \tm \P^n$が誘導されることを確かめればよいことがわかる。これは演習3.15および演習3.16と関連する。これらの演習で鍵になっていたのは「正則函数を束ねたものが射である」という補題3.6であったことを思い出そう。そこで補題3.6の類似を証明する。この補題から引用部を示す議論は省略する。
  \end{rem}
  \begin{comment}
  \lem{
  (閉部分集合とアファイン座標環のイデアルの対応) \\
  $Y \subset \A^n$をアファイン多様体とする。部分集合$F \subset Y$があるとする。このとき、次は同値。
  \begin{description}
    \item[(1)] $F$は$Y$の閉部分集合。
    \item[(2)] ある$A(Y)$のイデアル$D$が存在して、
    \[
F = Z(D) = \setmid{x \in Y}{\forall f \in D \; f(x) = 0}
    \]
    が成り立つ。
  \end{description}
  }
  \begin{proof} ${}$
    \begin{description}
      \item[(1)$\To$(2)] $Y \clsub \A^n$より、$k[x_1, \cdots , x_n]$のイデアル$J$であって、$F = Z(J)$なるものがある。このとき
      \begin{align*}
        Z(J) \subset Y &\iff Z(J) \subset Z(I(Y)) \\
        &\iff I(Y) \subset I(Z(J)) \\
        &\iff I(Y) \subset \sqrt{J}
      \end{align*}
      より、$I(Y) \subset \sqrt{J}$がわかる。そこで$D = \sqrt{J} / I(Y)$とおけば条件を満たすことはあきらか。
      \item[(2)$\To$(1)] $D \subset A(Y)$を$k[x_1, \cdots , x_n]$のイデアル$J$に持ち上げる。$I(Y) \subset J$である。
      このとき$F = Z(J) \subset Y$であることをみるのはやさしい。よって$F \clsub Y$である。
    \end{description}
  \end{proof}
\end{comment}
  \lem{
  (補題3.6の類似) \\
  $X$を抽象非特異曲線、$Y \subset \A^n$を準アファイン多様体とする。集合の間の写像$\psi \colon X \to Y$は、各$i$について$x_i \circ \psi $が$X$上の正則函数であるとき、またそのときに限って射である。ここで$x_1 , \cdots , x_n$は$\A^n$上の座標函数である。
  }
  \begin{proof}
    $\psi$が射であると仮定すれば$x_i \circ \psi $が$X$上の正則函数となるのはあきらか。逆を示そう。すべての$i$について$x_i \circ \psi $が$X$上の正則函数であるとする。このとき任意の多項式$f \in k[x_1, \cdots ,x_n] = k[x]$に対して、$f \circ \psi$は$X$上の正則函数である。この状況で$\psi$が連続かつ正則函数を引き戻しで保つことを示そう。

    まず連続性をいう。$Y$の閉部分集合$F$が与えられたとする。$F =Y \cap  Z(J)$なるイデアル$J \subset k[x]$がある。したがって
    \begin{align*}
      \psi^{-1}(F) &= \psi^{-1}\left( \bigcap_{f \in J} \setmid{x \in Y}{f(x) = 0} \right) \\
      &= \bigcap_{f \in J} \setmid{w \in X}{(f \circ \psi)(w) = 0}
    \end{align*}
    である。$f \circ \psi$は$X$上正則なので、とくに連続である。よって$\psi^{-1}(F) \clsub X$がわかる。

    次に$\psi$による正則函数の引き戻しが正則であることをみる。$Y$上の有理関数$(U,f) \in \coprod_{U \opsub Y} \calo(U)$が与えられたとする。$\psi^{-1}(U) = \emptyset$の場合は考える必要はない。$x \in \psi^{-1}(U)$が与えられたとして、$x $の周りで$f \circ \psi$が正則であることをみれば十分。いま$\psi(x) \in U$より、
    $f$は$U$上正則だから、ある$x$の開近傍$V \subset U$と多項式$g$と$V$上でゼロにならない多項式$h$とが存在して、
    \[
    \forall x \in V \quad f(x) = \f{g(x)}{h(x)}
    \]
    が成り立つ。したがって$y \in \psi^{-1}(V)$について
    \[
    f \circ \psi (y) = \f{g(\psi(y))}{h(\psi(y))}
    \]
    である。いま仮定により$g \circ \psi$と$h \circ \psi$はそれぞれ$X$上正則なので、正則函数の定義からそれは$\bigcap_{P \in X} R_P$の元である。$h$は$V$上でゼロにならないので、$h \circ \psi$は$\psi^{-1}(V)$上でゼロにならない。
    つまり、任意の$P \in \psi^{-1}(V)$に対して$h \circ \psi \notin \frakm_P$であるが、$R_P$は局所環なので$h \circ \psi \in R_P^{\tm}$である。
    したがって$f \circ \psi \in \bigcap_{P \in \psi^{-1}(V)} R_P$がわかる。これは、$f \circ \psi$が$\psi^{-1}(V)$上の正則函数であることを意味する。
  \end{proof}




  \bfsubsection{命題 6.8}
  \barquo{
ここで$U_k$はアファインであり、アファイン座標環は
\[
k[x_0/x_k, \cdots , x_n/x_k]
\]
に等しい。これらの函数の引き戻しは$f_{0k}, \cdots , f_{nk}$であり、構成により$P$において正則である。
  }
\begin{rem}
$x_i/x_k$の$\ol{\vp}$による引き戻しは$x_i/ x_k \circ \ol{\vp}$という形をしている。$\ol{\vp}(P)$の$k$成分は$1$なので、これは$f_{ik}$に等しく、$f_{ik} \in R_P$より$P$において正則である。    
\end{rem}


\newpage

\bfsection{2.1 層}

\bfsubsection{定義-前層 直後}
\barquo{
すると前層はちょうど圏$\Top(X)$からAbel群のなす圏$\Ab$への反変函手となる。
}
\begin{rem}
\textblue{それはおかしい。}直前の定義では前層$\scrf$に対して$\scrf(\emptyset) = 0$という条件を課していたが、これは一般の反変関手$\Top(X) \to \Ab$については成り立たない。成り立たない例は、たとえば次の可換図式
\[
\xymatrix{
U \ar[d]_i & \scrf U \ar@{=}[r] & \Z \\
V & \scrf V \ar[u]^{\scrf i} \ar@{=}[r] & \Z \ar[u]_{id}
}
\]
で定められる反変関手$\scrf \colon \Top(X) \to \Ab$で与えられる。

これではHartshorne自身どういう定義にしたいかがわからないので困ってしまう。他の文献も調べたが、Wedhorn\cite{Wedhorn}やG\"{o}rts Wedhorn\cite{GW}では単に反変関手であるとする定義を採用しているし、Liu\cite{Liu}では空集合を$0$に送るという定義を採用していて、両方の流儀があった。ただしネット上では単に反変関手とする定義が優勢なようだったが…。したがって、前層の定義をどうするかというのは非常に悩ましいところだが、ここでひとつ注意しておきたいことがある。『前層の定義としてどちらを採用しても、層の定義にはまったく影響しない』ということだ。なぜなら、層の条件として課される貼り合わせの一意性が、空集合の送り先が$0$であることを保証しているからだ。

具体的には、層$\scrf \colon \Top(X) \to \Ab$があるとき、空集合の開被覆として空集合により添え字づけられた集合族$\{V_i\}_{i \in \emptyset}$をとればよい。このときすべての$s \in \scrf(\emptyset)$について
\[
\forall i  \quad s|_{V_i} = 0
\]
が成立するので、貼り合わせの一意性より$s=0$となる。よって$\scrf(\emptyset)=0$が従う。

それで我々のとる立場であるが、このPDFでは『前層はただの反変関手』という定義を採用することにしたい。理由は、圏論的な文脈では前層とは『(小さな)圏から集合の圏への反変関手』と定義されることが多く、その定義と整合性をとりたいからである。
\end{rem}



\bfsubsection{定義-前層 直後}
\barquo{
用語として、$\scrf$を前層としたとき、$\scrf(U)$を開集合$U$上の前層$\scrf$の切断といい、
}
\begin{rem}
  原著では『As a matter of terminology, if $\scrf$ is a presheaf on $X$, we refer to $\scrf(U)$ as
the sections of the presheaf $\scrf$ over the open set $U$, 』と書いてあるところである。普通は$\scrf(U)$の元のことを切断というようだが、こういう使い方もするのかもしれない。
\end{rem}


\bfsubsection{定義-層}
\barquo{
位相空間$X$上の前層$\scrf$は、$\scrf$が前層の条件に加えて以下の条件を満たすとき層という。
}
\begin{rem}
この層の条件では、$U$や$V_i$といったものは空でも構わない。普段、$U$が空かどうかを常にチェックするのは面倒なので省略することが多いが、気をつけるべき時もある。以下、注意すべき例を挙げる。
\begin{description}
  \item[例-定数前層] たとえば、あるAbel群$A$をとり、任意の$U$について$\scrf(U) = A$であるとし、制限写像は恒等写像であるとすれば
  $\scrf$は$X$上の前層だが、$A \neq 0$ならば層ではない。$\scrf$は貼り合わせの存在条件を満たしているのだが、開被覆$\emptyset  = \bigcup_{i \in \emptyset} V_i$をとると貼り合わせの一意性をみたさないことがわかる。開被覆の添え字集合が空でなければ条件を満たすので、かなり惜しい例であるといえる。
  \item[例-空集合で修正した定数前層] それでは、
\[
\scrf(U) = \begin{cases}
A  &(U \neq \emptyset) \\
\scrf(\emptyset) = 0
\end{cases}
\]
  とし、制限写像を自然に定めたらどうだろうか。この$\scrf$は前層であり、貼り合わせの一意性を満たす。しかし、今度は一般に貼り合わせの存在条件を満たさなくなる。たとえば、$V_i \neq \emptyset$かつ$U = V_1 \cup V_2$かつ$V_1 \cap V_2 = \emptyset$であるような開被覆を考えるとよい。したがってやはり層にならない。
\end{description}
このように、空集合になりうることが問題になる局面もある。何かの前層が層になるかどうかを確認するとき、貼り合わせの一意性を確かめるのに、$U = \emptyset$は分けて考えたほうがより丁寧だろう。貼り合わせの存在条件は、前件にも後件にも添え字集合が現れるので、$U$が空かどうかを気にする必要はないと思われる。その代わり、各$V_i \cap V_j $が空かどうかは気にしなければならない。
\end{rem}

\begin{comment}
\begin{rem}
このあと役に立つかどうかはわからないが、次のような言い換えがある。$X$上の$\Ab$-前層$\scrf$が層であるとは、任意の開集合$U \subset X$とその開被覆$U = \bigcup_i U_i$に対して、次の図式
\[
\xymatrix{
0 \ar[r] & \scrf(U) \ar[r]^-{d} & \prod_{i} \scrf(U_i) \ar[r]^-{e} & \prod_{i,j} \scrf(U_{ij})
}
\]
が完全系列であることである。ただし$U_{ij}=U_i \cap U_j$であり、かつ$d$, $e$は
\[
d(s)=(s|_{U_i})_i \quad e((s_i)_i) = (s_i|_{U_{ij}} - s_j|_{U_{ij}})
\]
により定まる準同形である。
\end{rem}
\end{comment}




\bfsubsection{例 1.0.3}
\barquo{
すべての連結開集合$U$に対して$\scra (U) \cong A$であることに注意しよう。
}
\begin{rem}
  連続写像$f \in \scra (U)$による$U$の像は連結であり、したがって$A$は離散空間なので1点集合である。ゆえに$f \mapsto f(U)$という同型が作れる。
\end{rem}




\bfsubsection{例 1.0.3}
\barquo{
すべての連結成分が開集合であるような開集合$U$では(これは局所連結な位相空間では常に正しい)
}
\begin{rem}
次の命題が知られている。
  \prop{
  位相空間$X$について、次は同値である。
  \begin{description}
    \item[(1)] $X$は局所連結である。つまり各点で連結集合からなる基本近傍系をもつ。
    \item[(2)] 任意の$U \opsub X$の各連結成分は$U$の開部分集合である。
    \item[(3)] $X$は連結開集合からなる開基をもつ。
  \end{description}
  }
  \begin{proof}
    証明はやさしいが、内田\cite{内田}定理25.6を参照のこと。
  \end{proof}
\end{rem}




\bfsubsection{定義-茎}
\barquo{
$\scrf$の$P$における茎$\scrf_P$を、$P$を含むすべての開集合$U$に対する群$\scrf(U)$と制限写像$\rho$がなす順系に関する順極限と定義する。
}
\begin{rem}
  各点$P \in X$を決めるごとに、$P$の開近傍の全体が包含射に関してなす圏$\Top(X,P)$(ここだけの記号)の反対圏は有向集合とみなせる。したがって前層
  が誘導する関手$\scrf \colon \Top(X,P)\op \to \Ab$はAbel群の順系である。したがって順極限
  \[
  \scrf_P = \coprod_{P \in U } \scrf(U) / \sim
  \]
  がある。ここで、$\sim$は$ \coprod_{P \in U } \scrf(U)$上で
  \[
(U,s) \sim (V,t) \iff  \exists W \subset U \cap V \st s|_W = t|_W
  \]
  により定められる関係である。順系の性質からこれが同値関係になることは認める。
\end{rem}



\bfsubsection{定義-前層の射}
\barquo{
射$\vp \colon \scrf \to \scrg$とは各開集合$U$に対するAbel群の写像$\vp(U) \colon \scrf(U) \to \scrg(U)$からなるもので
}
\begin{rem}
  圏論の言葉が好みであれば、射$\vp \colon \scrf \to \scrg$とは$\Top(X)\op$から$\Ab$への関手のあいだの自然変換であるといえばよい。
\end{rem}




\bfsubsection{命題 1.1 直前}
\barquo{
$X$上の前層の射$\vp \colon \scrf \to \scrg$は$X$上の任意の点$P$に対して茎の射$\vp_P \colon \scrf_P \to \scrg_P$を導くことに注意する。
}
\begin{rem}
順極限の普遍性により、任意の$U \in \Top(X,P)$について次の図式
\[
\xymatrix{
\scrf(U) \ar[d] \ar[r]^-{\vp(U)} & \scrg(U) \ar[d] \\
\scrf_P \ar[r]^-{\vp_P} & \scrg_P
}
\]
が可換になるような$\vp_P$の存在と一意性がいえる。ただし$\scrf(U) \to \scrf_P$は自然な入射である。合成を保つことと恒等射を恒等射に送ることはあきらかなので、順極限は関手圏$[\Top(X,P)\op , \Ab]$から$\Ab$への関手であることがわかる。
\end{rem}


\begin{comment}
ついでに言うと、次の命題が成り立つことが知られている。
\end{rem}

\prop{
順極限が定める関手$\rlim \colon [\Top(X,P)\op , \Ab] \to \Ab$について、次が成り立つ。$[\Top(X,P)\op , \Ab]$における図式
\[
\xymatrix{
0 \ar[r] & A \ar[r]^{r} & B \ar[r]^s & C \ar[r] & 0
}
\]
があって
\[
\xymatrix{
0 \ar[r] & A(U) \ar[r]^{r(U)} & B(U) \ar[r]^{s(U)} & C(U) \ar[r] & 0
}
\]
がすべての$U \in \Top(X,P)$に対して完全ならば、次の図式
\[
\xymatrix{
0 \ar[r] & \rlim A \ar[r]^{\ra{r}} & \rlim B \ar[r]^{\ra{s}} & \rlim C \ar[r] & 0
}
\]
も完全である。
}
\begin{proof}
  Rotman\cite{Rotman}命題5.33を参照のこと。
\end{proof}
\end{comment}


\bfsubsection{命題 1.1}
\barquo{
$\vp \colon \scrf \to \scrg$を位相空間$X$の上の層の射とする。$\vp$が同型射であるのは茎に誘導された写像$\vp_P \colon \scrf_P \to \scrg_P$が$X$上のすべての点$P$に対して同型射であるとき、またそのときに限る。
}
\begin{rem}
G\"{o}rts Wedhorn\cite{GW}命題2.23を真似て、次の形で示そう。
\end{rem}

\prop{
$X$が位相空間、$\scrf$と$\scrg$はAbel群に値をとる$X$上の前層であるとする。$\vp \colon \scrf \to \scrg$は前層の射とする。このとき、次が成り立つ。
\begin{description}
  \item[(1)] $\scrf$が層であるとする。このとき次が成り立つ。
  \begin{gather*}
  \forall x \in X \quad \vp_x \colon \scrf_x \to \scrg_x \; \text{は単射} \\ \iff  \forall U \opsub X \quad \vp_U \colon \scrf(U) \to \scrg(U) \text{は単射}
\end{gather*}
  \item[(2)] $\scrf$も$\scrg$も層であるとする。このとき次が成り立つ。
  \begin{gather*}
  \forall x \in X \quad \vp_x \colon \scrf_x \to \scrg_x \; \text{は全単射} \\ \iff   \forall U \opsub X \quad \vp_U \colon \scrf(U) \to \scrg(U) \text{は全単射}
\end{gather*}
\end{description}
}

\begin{proof} ${}$
\begin{description}
  \item[(1)] ($\Rightarrow$)を示そう。任意の$U \opsub X$について図式
  \[
  \xymatrix{
\scrf(U) \ar[r] \ar[d]_-{\vp_U} & \prod_{x \in U} \scrf_x \ar[d]^-{\prod_x \vp_x} \\
\scrg(U) \ar[r] & \prod_{x \in U} \scrg_x
  }
  \]
  は可換なので、$\scrf$が層であるとき写像
  \[
  \scrf(U) \to \prod_{x \in U} \scrf_x, \quad s \mapsto (s_x)_{x \in U}
  \]
  が単射であることをみればよい。ただし$s_x$は$U$上の切断$s$の$x$における芽である。実際、もし任意の$x \in U$について$s_x = 0$ならば、ある$x$の開近傍$V_x \subset U$が$x$ごとに存在して$s |_{V_x} = 0$である。$U=\bigcup_x V_x$よりこれは開被覆となる。したがって、$\scrf$が層であるという仮定から$s=0$でなくてはならない。よってこれは単射。

  ($\Leftarrow$)を示そう。順極限が完全列を保つこと(Rotman\cite{Rotman}命題5.33)による。つまり、任意の$x \in U$について
  \[
  \xymatrix{
    0 \ar[r] & \scrf(U) \ar[r]^-{\vp_U}  & \scrg(U)
  }
  \]
  が完全なので、順極限をとって
  \[
  \xymatrix{
    0 \ar[r] & \scrf_x \ar[r]^-{\vp_x}  & \scrg_x
  }
  \]
  も完全。
  \item[(2)]
($\Rightarrow$ $\vp_U$が単射)と($\vp_x$が単射 $\Leftarrow$) は(1)で既に示している。($\vp_x$が全射 $\Leftarrow$)は順極限が完全列を保つことからあきらか。

  ($\Rightarrow$ $\vp_U$は全射) を示そう。任意に開集合$U$が与えられたとし、$t \in \scrg(U)$とする。$x \in U$に対して、$t$の$x$における芽$t_x \in \scrg_x$は仮定より$\vp_x$の像に入る。したがって$\vp_x(s_x) = t_x$なる茎の元$s_x \in \scrf_x$がある。芽$s_x$の代表元$\kakko{U^x,s^x}$であって、
  \[
  x \in U^x \opsub U, \quad s^x \in \scrf(U^x) , \quad
  \]
  なるものをとる。このとき
  \[
  \xymatrix{
\scrf(U^x) \ar[r] \ar[d]_-{\vp_{U^x}} &  \scrf_x \ar[d]^-{ \vp_x} \\
\scrg(U^x) \ar[r] &  \scrg_x
  }
  \]
  は可換なので、
  \begin{align*}
    t_x &= \vp_x(s_x) \\
    &= \vp_x(\kakko{U^x,s^x}_x) \\
    &= \kakko{U^x,\vp_{U^x}(s^x) }_x
  \end{align*}
  が成り立つ。よって$t \in \scrg(U)$と$\vp_{U^x}(s^x) \in \scrf(U^x)$は点$x$における芽が等しいので
  \[
  x \in V^x \opsub U_x, \quad \vp_{U^x}(s^x)|_{V^x} = t|_{V^x}
  \]
  なる$V_x$がある。$\vp$は前層の射なので$t|_{V^x} = \vp_{U^x}(s^x)|_{V^x} = \vp_{V^x}(s^x|_{V^x})$である。$V^x$は$U$の開被覆を与える。
  ここで任意の$x , y \in U$に対して
  \begin{align*}
    \vp_{V^x \cap V^y} (s^x|_{V^x \cap V^y}) &=   \vp_{U^x}(s^x)|_{V^x \cap V^y} \\
    &=  (\vp_{U^x}(s^x)|_{V^x} )|_{V^x \cap V^y} \\
    &= (t|_{V^x})|_{V^x \cap V^y} \\
    &= t|_{V^x \cap V^y} \\
    &= (t|_{V^y})|_{V^x \cap V^y} \\
    &= \cdots \\
    &= \vp_{V^x \cap V^y} (s^y|_{V^x \cap V^y})
  \end{align*}
  が成り立つ。(1)より、$\scrf$が層なので$\vp_{V^x \cap V^y}$は単射。したがって
  \begin{gather*}
s^x|_{V^x \cap V^y} = s^y|_{V^x \cap V^y} \\
(s^x|_{V^x}) |_{V^x \cap V^y} = (s^y|_{V^y})|_{V^x \cap V^y}
  \end{gather*}
  である。$\scrf$は層なので、$U$全体にわたって$s^x|_{V^x}$を貼り合わせることができて、
  \[
  s^x|_{V^x} = s|_{V^x}
  \]
  なる$s \in \scrf(U)$の存在がいえる。このとき任意の$x \in U$に対して
  \begin{align*}
    \kakko{U,\vp_U(s)}_x &= \vp_x(\kakko{U,s}_x) \\
    &= \vp_x(\kakko{V^x,s^x|_{V^x}}_x) \\
    &= t_x
  \end{align*}
  である。ゆえに$\vp_U(s) - t $は$\scrg(U) \to \prod_{x \in U} \scrg_x$の核に入るが、$\scrg$は層なのでこれは単射。したがって$\vp_U(s)=t$であり、$\vp_U$の全射性がいえた。
\end{description}
\end{proof}




\bfsubsection{定義-前層核}
\barquo{
$\vp \colon \scrf \to \scrg$を前層の射とする。$\vp$の前層核、$\vp$の前層余核、$\vp$の前層像をそれぞれ$U \mapsto \ker(\vp(U))$, $U \mapsto \coker(\vp(U))$, $U \mapsto \im (\vp(U))$で与えられる前層と定義する。
}
\begin{rem}
  これらが本当に前層であることを確認しよう。$\vp$が射であることにより任意の射$i \colon U \to V$に対して次が可換。
  \[
  \xymatrix{
  \scrf(U) \ar[r]^-{\vp_U} & \scrg(U) \\
  \scrf(V) \ar[r]^-{\vp_V} \ar[u]^-{\scrf i} & \scrg(V) \ar[u]_-{\scrg i}
  }
  \]
  したがって
  \begin{gather*}
    \scrf i (\Ker \vp_V) \subset \Ker \vp_U \\
        \scrg i (\Im \vp_V) \subset \Im \vp_U
  \end{gather*}
  である。さらに$\scrg i$は$\Coker \vp_V \to \Coker \vp_U$を誘導する。これらの射の対応に関して合成を保つこと、恒等射を保つことはあきらかなので、たしかに前層になる。また、定義の仕方により$\Ker \vp \to \scrf$、$\scrg \to \Coker \vp$、$\Im \vp \to \scrg$は前層の射である。

  さて、以上で前層であることを確認したが、これは本当に$\Ker$や$\Coker$で書くに値するものなのだろうか、という疑問に答えておく必要がある。ここでは前加法圏や加法圏といった言葉を出すのはまだ避けたいので、$\Ker$や$\Coker$の一般的な定義は行わずに前層の圏に限定して話をする。
\end{rem}

\prop{
($\Ker$の普遍性) \\
$\vp \colon \scrf \to \scrg$が$X$上の前層の射であるとする。任意に前層$\scrh$と射$\psi \colon \scrh \to \scrf$が与えられ、$\vp \circ \psi=0$を満たすとする。このとき次の図式
\[
\xymatrix{
\scrh \ar@{.>}[dr]_-{i} \ar[drr]^-{\psi} \ar[ddr]_-{\psi}  & {} & { } \\
{} & \Ker \vp \ar[r] \ar[d] & \scrf \ar[d]^-{0} \\
{} & \scrf \ar[r]^-{\vp} & \scrg
}
\]
を可換にするような射$i$が存在し、一意である。

ただし、$0$というのは任意の開集合$U$に対して$0$写像を返すようなものである。これが前層の射であることはあきらか。
}

\begin{proof}

開集合$U$が与えられるごとに、Abel群の圏における$\Ker$の普遍性により、次の図式
\[
\xymatrix{
\scrh(U) \ar@{.>}[dr]_-{i_U} \ar[drr]^-{\psi_U} \ar[ddr]_-{\psi_U}  & {} & { } \\
{} & \Ker \vp_U \ar[r] \ar[d] & \scrf \ar[d]^-{0} \\
{} & \scrf(U) \ar[r]^-{\vp_U} & \scrg(U)
}
\]
を可換にするような$i_U$が存在していることがわかる。あとは$i_U$が自然変換になっているか(前層が誘導する制限写像と交換するか)をみればよい。なぜなら、一意性はあきらかだからである。自然変換$\psi$の値域の制限なので,元をとって議論すればすぐにわかるのだが、それでは$\Coker$も同様、などと言えなくなってしまう。射の性質だけで議論を行おう。開集合の包含射$j \colon V \to U$が与えられたとする。まず$\ker \colon \Ker \vp \to \scrf$は射なので、次が可換。
\[
\xymatrix{
\Ker \vp_U \ar[r]^-{(\Ker \vp)j } \ar[d]_{\ker_U} & \Ker \vp_V \ar[d]^-{\ker_V} \\
\scrf(U) \ar[r]^-{\scrf j} & \scrg(V)
}
\]
加えて、$i$の定義と$\psi$の自然性により次は可換。
\[
\xymatrix{
{} & \scrh(U) \ar[dl]_-{i_U} \ar[r]^-{\scrh j} \ar[dd]^-{\psi_U} & \scrh(V) \ar[dr]^-{i_V} \ar[dd]^-{\psi_V} & {} \\
\Ker \vp_U \ar[dr]_-{\ker_U} & {} & {} & \Ker \vp_V \ar[dl]^-{\ker_V} \\
{} & \scrf(U) \ar[r]^-{\scrf j} & \scrg(V) & {}
}
\]
したがって、
\begin{align*}
  \ker_V \circ (\ker \vp)j \circ i_U &= \scrf i \circ \ker_U \circ i_U \\
  &= \ker_V \circ i_V \circ \scrh j
\end{align*}
である。ここで$\ker_V$はモノ射だから(元をとってもいいし、Abel群の圏での核の普遍性からもいえる)$(\ker \vp)j \circ i_U =i_V \circ \scrh j $である。したがって自然性がいえた。
\end{proof}


\prop{
($\Coker$の普遍性) \\
$\vp \colon \scrf \to \scrg$が$X$上の前層の射であるとする。任意に前層$\scrh$と射$\psi \colon \scrg \to \scrh$が与えられ、$\psi \circ \vp =0$を満たすとする。このとき次の図式
\[
\xymatrix{
{} & {} & \scrh \\
\scrg \ar[r] \ar[rru]^-{\psi} & \Coker \vp \ar@{.>}[ru]_-k & {} \\
\scrf \ar[u]^-0 \ar[r]^-{\vp} & \scrg \ar[u] \ar[uur]_-{\psi} & {}
}
\]
を可換にするような射$k$が存在し、一意である。
}
\begin{proof}
  $\Ker$のときと同様。各自試みよ。$\scrg(U) \to \Coker \vp_U$がエピ射であることが重要である。
\end{proof}

また$\Im$については、任意の開集合$U$について$\Im(\vp_U) = \Ker(\scrg \to \Coker \vp_U)$であることから容易に予想されるように、次が成り立つ。

\prop{
($\Im$は$\Ker$と$\Coker$で書ける) \\
$\vp \colon \scrf \to \scrg$が$X$上の前層の射であるとする。前層の射$\scrg \to \Coker \vp$を$\psi$と書くことにする。このとき、前層として
\[
\Im \vp = \Ker (\psi)
\]
が成り立つ。
}
\begin{proof}
  すべての$U \opsub X$について、Abel群として$(\Im \vp)_U =\Im (\vp_U) = \Ker (\psi_U) = (\Ker \psi)_U$であることはあきらか。加えて、単なる包含写像であるためAbel群の準同形として$\ker \psi_U = \im \vp_U$が成り立つ。よって$\ker \psi \colon \Ker \psi \to \scrg$と$\im \vp \colon \Im \vp \to \scrg$はまったく同一の自然変換である。

ここで、$\Ker$の普遍性により次の図式
\[
\xymatrix{
\Im \vp \ar@{.>}[dr]_-{\beta} \ar[drr]^-{\im \vp} \ar[ddr]_-{\im \vp}  & {} & { } \\
{} & \Ker \psi \ar[r]_-{\ker \psi} \ar[d]^-{\ker \psi} & \scrg \ar[d]^-{0} \\
{} & \scrg \ar[r]^-{\coker \vp} & \Coker \vp
}
\]
を可換にするような自然変換$\beta$がある。ところが自然変換として$\ker \psi = \im \vp$であるので、$(\ker \psi)_U$の単射性により、つねに$\beta_U = id$でなくてはならない。これは、恒等射が自然変換であることを意味する。



\end{proof}


\bfsubsection{定義-前層核 直後}
\barquo{
$\vp \colon \scrf \to \scrg$を層の射とするとき、$\vp$の前層核は層になるが、
}
\begin{proof}
開集合$U \subset X$とその開被覆$U = \bigcup_i U_i$が与えられたとしよう。

貼り合わせの一意性: $s \in \Ker \vp_U$であって、$s|_{U_i}=0$なるものが与えられたとせよ。このとき$\Ker \vp \to \scrf$は射なので
\[
\xymatrix{
\Ker \vp_U \ar[r] \ar[d] & \scrf(U) \ar[d] \\
\Ker \vp_{U_i} \ar[r] & \scrf(U_i)
}
\]
は可換である。よって
\begin{align*}
  0 &= (\ker \vp)_{U_i} (s|_{U_i}) \\
  &= (\ker \vp)_U (s)|_{U_i}
\end{align*}
である。したがって$\scrf$は層なので$(\ker \vp)_U (s) = 0 \in \scrf(U)$である。$(\ker \vp)_U$は単射だから$s=0$である。

貼り合わせの存在: $s_i \in \Ker \vp_{U_i}$であって
\[
s_i|_{U_i \cap U_j} = s_j|_{U_i \cap U_j}
\]
なるものが与えられたとする。このとき$\Ker \vp \to \scrf$は射なので
\begin{align*}
  (\ker \vp)_{U_i} (s_i)|_{U_i \cap U_j} &= (\ker \vp)_{U_i \cap U_j} (s_i |_{U_i \cap U_j}) \\
  &= (\ker \vp)_{U_i \cap U_j} (s_j |_{U_i \cap U_j}) \\
  &= (\ker \vp)_{U_j} (s_j)|_{U_i \cap U_j}
\end{align*}
が成り立つ。$\scrf$は層なので$(\ker \vp)_{U_i}(s_i) \in \scrf(U_i)$を貼り合わせることにより
\[
s|_{U_i} = (\ker \vp)_{U_i}(s_i)
\]
なる$s \in \scrf(U)$の存在がいえる。ここで$\vp \colon \scrf \to \scrg$は自然変換なので、
\[
\xymatrix{
\scrf(U) \ar[r]^-{\vp_U} \ar[d] & \scrg(U) \ar[d] \\
\scrf(U_i) \ar[r]^-{\vp_{U_j}} & \scrg(U_i)
}
\]
は可換。ゆえに
\begin{align*}
  \vp_U(s)|_{U_i} &= \vp_{U_i} (s|_{U_i}) \\
  &= \vp_{U_i} ( (\ker \vp)_{U_i} (s_i) ) \\
  &= 0
\end{align*}
である。$\scrg$は層なので、$\vp_U(s) = 0 \in \scrg(U)$である。よって、ある$\wt{s} \in \Ker \vp_U$が存在して$s = (\ker \vp)_U (\wt{s})$である。このとき
\begin{align*}
  (\ker \vp)_{U_i} (\wt{s}|_{U_i}) &= (\ker \vp)_U (\wt{s}) |_{U_i} \\
  &= s|_{U_i} \\
  &= (\ker \vp)_{U_i} (s_i)
\end{align*}
である。$(\ker \vp)_{U_i}$は単射なので$\wt{s}|_{U_i} = s_i$が成り立つ。
\end{proof}





\bfsubsection{命題-定義 1.2}
\barquo{
与えられた前層$\scrf$に対して、次の性質を持つような層$\scrf^+$および射$\grt \colon \scrf \to \scrf^+$が存在する: 任意の層$\scrg$および任意の射$\vp \colon \scrf \to \scrg$に対して、$\vp = \psi \circ \grt$であるような射$\psi \colon \scrf^+ \to \scrg$が一意的に存在する。さらに対$(\scrf^+, \grt)$は同型を除いて一意的である。$\scrf^+$を前層$\scrf$
に付随した層という。
}

\begin{rem}
圏論的にいうと、これは層の圏から前層の圏への忘却関手$U \colon \Sh \to \PSh$が左随伴をもつことをいっている。随伴の定義を確認しておこう。
\end{rem}

\prop{
\thispagestyle{empty}%このページのページ番号を消去
$\bfc$, $\bfd$は圏であり、関手$G \colon \bfc \to \bfd$があるとする。このとき次は同値。
\begin{description}
  \item[(1)] $G$は右随伴(左随伴をもつ関手)である。すなわち、ある関手$F \colon \bfd \to \bfc$と自然同型
  \[
\Phi \colon \Hom_C(F -, -) \to \Hom_D(-, G -)
  \]
  が存在する。これは、任意の$X \in \bfc$と$Y \in \bfd$に対して
  \[
  \Phi_{Y,X} \colon \Hom_C(F Y, X) \to \Hom_D(Y, G X)
  \]
  が全単射であり、かつ$X$についても$Y$についても自然性を満たすことを意味する。

  \item[(2)] ある関手$F \colon \bfd \to \bfc$と、余単位射といわれる自然変換$\ve \colon FG \to 1_{\bfc}$と、単位射といわれる自然変換$\eta \colon 1_{\bfd} \to GF$が存在して、三角恒等式を満たす。すなわち、次の図式が可換となる。
  \[
  \xymatrix{
  F \ar[r]^{F \eta} \ar[dr]_{id} & FGF \ar[d]^{\ve F} & G \ar[dr]_{id} \ar[r]^{\eta G} & GFG \ar[d]^{G \ve} \\
  {} & F & {} & G
  }
  \]

  \item[(3)] 任意の$\bfd$の対象$Y$に対して、$Y$から$G$への普遍射$(X, \vp \colon Y \to G(X))$が存在する。ここで$(X, \vp \colon Y \to G(X))$が普遍射であるというのは、任意の対象$A \in \bfc$と射$g \colon Y \to G(A)$に対して、次の図式
  \[
  \xymatrix{
  Y \ar[dr]_-{g} \ar[r]^-{\vp} & G(X) \ar@{.>}[d]^-{Gf} & X \ar@{.>}[d]^-{f} \\
  {} & G(A) & A
  }
  \]
  が可換になるような$f$が一意的に存在することを意味する。
\end{description}
}

\begin{proof}
たとえばS.Mac Lane\cite{MacLane}第4章1節 定理2を参照のこと。
\end{proof}


\begin{proof}
  命題1.2の証明をしよう。段階を踏んで示す。
\begin{description}
  \item[Step 1] 位相空間$X$上の前層$\scrf$が与えられたとし、開集合$U \subset X$に対して、集合$\scrf^{+}(U)$を構成しよう。記号を用意しておく。点$x$に対して、$x$の開近傍であるようなすべての$V \opsub U$についての$\scrf(V)$の集合としての直和$\coprod_{x \in V \subset U} \scrf(V)$を考える。元$(V,s) \in \coprod_{x \in V \subset U} \scrf(V)$に対して、
  芽の直和への写像$\ol{s} \colon V \to \coprod_{x \in V} \scrf_x$を
  \[
  \ol{s} (y) = s_y
  \]
  で定める。そうして、
  \[
  \scrf^{+}(U) = \setmid{
  f \colon U \to \coprod_{x \in U}\scrf_x
  }
  {
  \forall x \in U \; \exists (V,s) \in \coprod_{x \in V \subset U} \scrf(V) \st f|_{V} = \ol{s}
  }
  \]
  と定める。この定義が本文でのものと一致することは容易に確かめられるだろう。直感的には、$\scrf^{+}(U)$というのは、各点$x \in X$に対して$x$での芽を対応させるような写像の貼り合わせで得られるような写像の全体である。
\item[Step 2] $U \opsub X$に対して、
\[
\Phi(U) = \setmid{f \colon U \to \coprod_{x \in U} \scrf_x    }{ f(x) \in \scrf_x     }
\]
とする。このとき$\scrf_x$がそれぞれAbel群なので、$\Phi(U)$はAbel群である。また、通常の写像の制限写像に関して前層になっている。のみならず、$\Phi$は層でもある。そこで、$\scrf^{+}(U) \subset \Phi(U)$が部分Abel群であることを示そう。

%まず、$e \in \Phi(U)$を単位元とする。$e \colon U \to \coprod_{x \in U} \scrf_x$は、つねに$0$を返すような写像である。したがって、任意の$x \in U$に対して$(U, 0) \in \coprod_{x \in V \subset U} \scrf(V)$
%をとれば$e|_U = e = \ol{0}$が成り立つので、$e \in \scrf^+(U)$がいえた。したがって$\scrf^+(U)$は単位元を持つ。

$\scrf^{+}(U)$が空でないことは明らかなので、$\scrf^+(U)$が和と逆元で閉じていることを示せば十分。$f_i \in \scrf^+(U) \; (i=1,2)$とする。このとき、$x \in U$に対して、
\[
f_i |_{V_i} = \ol{s_i}
\]
なる$(V_i, s_i) \in \coprod_{x \in V \subset U} \scrf(V)$がある。すると$y \in V_1 \cap V_2$に対して
\begin{align*}
  (f_1 - f_2)(y) &= \ol{s_1}(y) - \ol{s_2}(y) \\
  &= {s_1}_y - {s_2}_y \\
  &= (s_1|_{V_1 \cap V_2})_y - (s_2|_{V_1 \cap V_2})_y &(\scrf_x \text{の定義から}) \\
  &= (s_1|_{V_1 \cap V_2} - s_2|_{V_1 \cap V_2})_y
\end{align*}
が成り立つので、$(V_1 \cap V_2 , s_1|_{V_1 \cap V_2} - s_2|_{V_1 \cap V_2} ) \in \coprod_{x \in V \subset U} \scrf(V)$に関して$(f_1 - f_2)|_{V_1 \cap V_2} = \ol{s_1|_{V_1 \cap V_2} - s_2|_{V_1 \cap V_2}}$である。
$x \in U$は任意だったので、$\scrf^+(U)$が和と逆元に関して閉じていることがいえた。
ゆえに、$\scrf^+(U)$は$\Phi(U)$の部分Abel群である。
\item[Step 3] 層$\Phi$の制限写像が、$\scrf^+$の制限写像を誘導することをみよう。$f \in \scrf^+(U)$と$V \opsub U$が与えられたとして、$g = f|_V \in \Phi(V)$が$g \in \scrf^+(V)$を満たすことをいえばよい。$x \in V$とする。$x \in U$なので、$f \in \scrf^+(U)$によりある$(W,s ) \in \coprod_{x \in W \subset U} \scrf(W)$
が存在して、$f|_W = \ol{s}$を満たす。このとき$(W \cap V, s|_{W \cap V}) \in \coprod_{x \in E \subset V} \scrf(E)$であって、
\[
g|_{V \cap W} = \ol{s|_{W \cap V}}
\]
である。なぜなら、$y \in W \cap V$に対して
\begin{align*}
  (g|_{W \cap V})(y) &= g(y) \\
  &= f(y) \\
  &= \ol{s}(y) \\
  &= s_y \\
  &= (s|_{W \cap V})_y \\
  &= \ol{s|_{W \cap V}}(y)
\end{align*}
が成り立つからである。したがって$g \in \scrf^+(V)$であり、層$\Phi$の制限写像から誘導される制限写像がある。したがって$\scrf^+$が前層であることがいえた。
\item[Step 4] $\scrf^+$が層であることを示そう。開集合$U \subset X$と開被覆$U = \bigcup_{i} U_i$が与えられたとする。

貼り合わせの一意性は、層$\Phi$への単射があることから従う。つまり、$s \in \scrf^+(U)$が$s|_{U_i} =0$を満たしたとすると、$\Phi$が層なので、$s \in \Phi(U)$と見なしたときに$s = 0$でなくてはならない。したがって、$s = 0 \in \scrf^+(U)$である。これで貼り合わせの一意性がいえた。

貼り合わせの存在を示そう。$f_i \in \scrf^+(U_i)$がすべての$i,j$に対して$f_i |_{U_i \cap U_j} = f_j |_{U_i \cap U_j}$を満たしたとする。$\Phi$は層なのである$f_i \in \Phi(U)$が存在して、
\[
\forall i \quad f|_{U_i} = f_i
\]
を満たす。$f \in \scrf^+(U)$を示したい。$x \in U$とする。$x \in U_k$なる$k$がある。$f_k \in \scrf^+(U_k)$より、ある$(V,s) \in \coprod_{x \in V \subset U_k} \scrf(V)$が存在して
\[
f_k |_V = \ol{s}
\]
を満たす。このとき
\begin{align*}
  f|_V &= (f|_{U_k})|_V \\
  &= f_k |_V \\
  &= \ol{s}
\end{align*}
なので、$V$は$U$の開集合とも思えるので$f \in \scrf^+(U)$である。$f$が$f_i \in \scrf^+(U_i)$の貼り合わせを与えていることはあきらか。

\item[Step 5] 射$\grt \colon \scrf \to \scrf^+$を構成しよう。$U \opsub X$が与えられたとする。$\grt_U \colon \scrf(U) \to \scrf^+(U) $を$\grt_U (s) = \ol{s}$で定義する。$\grt_U$がAbel群の準同形であることはあきらか。準同形の族$\{ \grt_U \}$が自然変換$\scrf \to \scrf^+$を定めることをみたい。それには次の図式
\[
\xymatrix{
\scrf(U) \ar[r]^-{\grt_U} \ar[d] & \scrf^+(U) \ar[d] \\
\scrf(V) \ar[r]^-{\grt_V} & \scrf^+(V)
}
\]
が可換であることを確かめればよい。いま$x \in V$と$s \in \scrf(U)$に対して
\begin{align*}
  (\grt_U s)|_V (x) &= \grt_U s (x) \\
  &= \ol{s}(x) \\
  &= s_x \\
  (\grt_V (s|_V))(x) &= \ol{s|_V}(x) \\
  &= (s|_V)(x) \\
  &= s_x
\end{align*}
だから、これは確かめられた。よって射$\grt \colon \scrf \to \scrf^+$が構成できた。
\item[Step 6] $\scrf^+$が求められる普遍性を満たしていることを示そう。任意に層$\scrg$と射$\vp \colon \scrf \to \scrg$が与えられたとする。
\[
\xymatrix{
\scrf(U) \ar[r]^-{\grt_U} \ar[rd]^-{\vp_U} & \scrf^+(U) \ar@{.>}[d]^-{\psi_U} \\
{} & \scrg(U)
}
\]
最初の段階として、$f \in \scrf^+(U)$に対して$\psi_U(f) \in \scrg(U)$を構成したい。
まず、これまでの議論により次がいえることに気をつける。
\lem{
(層の切断は芽を返す写像) \\
$X$は位相空間、$\scrf$は$X$上の層であるとする。$X$上の層$\Phi$を
\[
\Phi(U) = \setmid{f \colon U \to \coprod_{x \in U } \scrf_x}{ f(x) \in \scrf_x}
\]
と通常の制限写像によるものとして定めておく。このとき、$U \opsub X$に対して写像$p_U \colon \scrf(U) \to \Phi(U)$を$p_U (s) = \ol{s}$で定めると、$p_U$は自然変換であり、$p_U \colon \scrf(U) \to \Phi(U)$は単射。おおざっぱにいうと、層$\scrf$の切断$s \in \scrf(U)$は$x \in U$に対して芽$s_x$を返す写像だと思うことができる。とくに、芽を考えることは代入することと同じとみなせる。
}
\begin{proof}
  ここまでくるとほとんど当たり前である。各$p_U \colon \scrf(U) \to \Phi(U)$がAbel群の準同形であることはあきらか。$\scrf$は層なので$\scrf(U) \to \prod_{x \in U} \scrf_x \st s \mapsto (s_x)_{x \in U}$は単射であり、したがって$p_U$も単射。また、$p$は射$\grt \colon \scrf \to \scrf^+$と、包含写像の族が誘導する射$\scrf^+ \to \Phi$の合成なので射である。
\end{proof}

$\psi_U$の構成に戻る。$x \in U$をとると、$f \in \scrf^+(U)$によりある$(V^x, s^x) \in \coprod_{x \in W \subset V^x} \scrf(W)$が存在して
$f|_{V^x} = \ol{s^x}$を満たす。このとき$t^x \in \scrg(V^x)$を
\[
t^x = \vp_{V^x} (s^x)
\]
で定めることができる。$t^x$を貼り合わせたいので、任意の$x,y \in X$に対して$t^x|_{V^x \cap V^y} = t^y|_{V^x \cap V^y}$が成り立つことをみよう。$\scrg$は層と仮定したので、$\scrg$の切断は写像だと思える。よって、任意の$z \in V^x \cap V^y$に対して
\[
(t^x)_z = (t^y)_z
\]
であることを示せば十分である。任意の$z$の開近傍$W$に対して次の図式
\[
\xymatrix{
\scrf(W) \ar[r]^-{\vp_W} \ar[d] & \scrg(W) \ar[d] \\
\scrf_z \ar[r]^-{\vp_z} & \scrg_z
}
\]
が可換であることに注意して、実際に計算すると
\begin{align*}
  (t^x)_z &= \vp_{V^x} (s^x)_z \\
  &= \vp_z((s^x)_z )  \\
  &= \vp_z (\ol{s^x}(z)) \\
  &= \vp_z(f(z))
\end{align*}
である。$\vp_z(f(z))$は$x$に依らないので、同様に続けて$(t^x)_z = (t^y)_z$がいえる。ゆえに、$\scrg$は層なので$t^x \in  \scrg(V^x)$の貼り合わせ$t \in \scrg(U)$が存在して$t|_{V^x} = t^x$を満たす。そこで、$\psi_U(f) = t$と定める。
\item[Step 7] $\psi$の定義がwell-definedであることを示そう。$\scrg$は層なので、$\psi_U(f) = t$の各点$x \in U$での芽を計算して、それが$f$と$\vp$にしか依らないことを確認すればよい。実行してみると
\begin{align*}
  t_x &= (t|_{V^x})_x \\
  &= (t^x)_x \\
  &= \vp_{V^x}(s^x)_x \\
  &= \vp_x((s^x)_x) \\
  &= \vp_x(\ol{s^x}(x)) \\
  &= \vp_x(f(x))
\end{align*}
であるから、写像としてwell-definedであることがいえた。
\item[Step 8] 各$\psi_U \colon \scrf^+(U) \to\scrg(U)$がAbel群の準同形になっていることを示そう。任意に$f,g \in \scrf^+(U)$が与えられたとする。このとき$x \in U$での芽を考えると
\begin{align*}
  \ol{\psi_U(f) + \psi_U(g) - \psi_U(f + g)}(x) &= \psi_U(f)_x + \psi_U(g)_x - \psi_U(f + g)_x \\
  &= \vp_x(f(x)) + \vp_x(g(x)) - \vp_x((f+g)(x)) \\
  &= 0
\end{align*}
となる。ゆえに$\scrg$は層なので、各点での芽が一致することから$\psi_U(f) + \psi_U(g) = \psi_U(f + g)$が成り立つ。
\item[Step 9] $\psi$が自然変換であることを示そう。$V \subset U$とする。次の図式
\[
\xymatrix{
\scrf^+(U) \ar[r]^-{\psi_U} \ar[d] & \scrg(U) \ar[d] \\
\scrf^+(V) \ar[r]^-{\psi_V} & \scrg(V)
}
\]
が可換であることをいえばよい。与えられた$f \in \scrf(U)$と$x \in V$について
\begin{align*}
  (\psi_U(f)|_V)_x &= \psi_U(f)_x \\
  &= \vp_x(f(x)) \\
  \psi_V(f|_V)_x &= \vp_x (f|_V (x)) \\
  &= \vp_x(f(x))
\end{align*}
が成り立つ。$\scrg$は層なので、各点での芽が一致することをいえば$\scrg(V)$の元として一致していることがわかる。よって示すべきことがいえた。
\item[Step 10] $\psi$の一意性を示そう。射$\psi \colon \scrf^+ \to \scrg$であって$\vp = \psi \circ \grt$を満たすものが任意に与えられたとする。ただし$\scrg$は層で、$\vp \colon \scrf \to \scrg$は射であるという状況はいままでと同じである。$\scrg$は層なので、$f \in \scrf^+(U)$と$x \in U$が与えられたとして、$\psi_U(f)_x$が$\vp$によって自由度なしに決定されてしまうことをみればよい。

$\psi$は射だったので、各点$x \in X$に対し、$x$の開近傍$U$をとれば
\[
\xymatrix{
\scrf^+(U) \ar[r]^-{\psi_U} \ar[d] & \scrg(U) \ar[d] \\
\scrf^+_x \ar[r]^-{\psi_x} & \scrg_x
}
\]
は可換である。したがって
\[
\psi_U(f)_x = \psi_x(f_x)
\]
である。ここで$f \in \scrf^+(U)$により、ある$(V,s) \in \coprod_{x \in V \subset U} \scrf(V)$が存在して、$f|_V = \ol{s}$である。ここで$\ol{s} \in \scrf^+(V)$であり、芽をとれば$f_x = \ol{s}_x$である。さらに$\grt_V(s) = \ol{s}$だから、
\begin{align*}
  \psi_U(f)_x &= \psi_x(f_x) \\
  &= \psi_x((\ol{s})_x ) \\
  &= \psi_x (\grt_V(s)_x ) \\
  &= \psi_x (\grt_x (s_x) ) \\
  &= (\psi \circ \grt )_x (s_x) \\
  &= \vp_x (s_x) \\
  &= \vp_x ( \ol{s} (x) ) \\
  &= \vp_x (f|_V (x)) \\
  &= \vp_x (f(x))
\end{align*}
となる。よって$\psi$は図式の可換性から一意に定まる。
\end{description}

\end{proof}




\bfsubsection{命題-定義 1.2}
\barquo{
なお、任意の点$P$に対して$\scrf_P = \scrf^+_P$であることに注意せよ。また、$\scrf$自身が層のとき、$\scrf^+$は$\grt$を介して$\scrf$と同型であることにも注意せよ。
}
\begin{proof} ${}$
  \begin{description}
    \item[Step 1] 次のことに気をつける。
    \prop{
    Abel群$A$, 位相空間$X$, 点$P \in A$が与えられたとする。このとき
  \[
  \scrf(U) = \begin{cases}
  0 &(P \notin U) \\
  A &(P \in U)
\end{cases}
  \]
  とおいて、制限写像を恒等写像と零写像で自然に定めると、$\scrf$は$X$上の層になる。これには摩天楼層という名前がついている。
    }
    \begin{proof}
      $\scrf$が前層になることはあきらか。空集合に気をつけつつ、層になることを見る。
      \begin{description}
        \item[Step 1] $\scrf(\emptyset)=0$はあきらか。
        \item[Step 2] 貼り合わせの一意性をみる。$U \neq \emptyset$と開被覆$U = \bigcup_{i \in I} V_i$が与えられたとする。$s \in \scrf(U)$であって$s|_{V_i}=0$なるものがあったとする。$P \notin U$ならば示すことはないので、$P \in U$としてよい。このとき$P \in V_j$なる$j \in I$がある。この$j$について、$V_i$への制限写像は恒等写像である。よって$s=0$がいえる。
        \item[Step 3] 貼り合わせ可能性をみる。開集合$U$と開被覆$U = \bigcup_{i } V_i$が与えられたとし、$s_i \in \scrf(V_i)$であって$s_i |_{V_i \cap V_j} = s_j |_{V_i \cap V_j}$なるものがあったとする。すべての$i$について$P \in V_i$であると仮定しても一般性を失わない。すると、仮定により$A$の元としてつねに$s_i = s_j$なので、$s = s_i$となる元$s \in A$がある。
      \end{description}
      以上により層であることが確認できた。
    \end{proof}

    \prop{
    $\scrf$を$X$上の前層、$P \in X$とする。茎$\scrf_P$に値をとる点$P$上の摩天楼層を$\wt{\scrf_P}$と書くことにする。このとき、自然な射$i \colon \scrf \to \wt{\scrf_P}$があり、さらに前層の射$\vp \colon \scrf \to \scrg$は射$\wt{\vp} \colon \wt{\scrf_P} \to \wt{\scrg_P}$
    を誘導する。
    }
    \begin{proof}
      あきらか。暇なときにでも確かめて欲しい。
    \end{proof}

    引用部の証明に戻る。摩天楼層$\wt{\scrf_P}$は層なので、層化の普遍性により、次の図式
    \[
    \xymatrix{
    \scrf \ar[r]^-{\grt} \ar[rd]_i & \scrf^+ \ar[d]^-{\psi} \\
    {} & \wt{\scrf_P}
    }
    \]
    を可換にするような$\psi$がある。すると今度は茎$\scrf^+_P$の普遍性により、
    \[
    \xymatrix{
    \scrf^+ \ar[r]^{i^+} \ar[rd]_{\psi} & \wt{\scrf^+_P} \ar[d]^{\wt{\psi}} \\
    {} & \wt{\scrf_P}
    }
    \]
    を可換にするような$\wt{\psi}$が存在する。以上により可換図式
    \[
    \xymatrix{
    \wt{\scrf_P} \ar[rrr]^-{\wt{\grt}} \ar@{=}[dd] & {} & {} & \wt{\scrf^+_P} \ar[dd]^-{\wt{\psi}} \\
    {} & \scrf \ar[r]^-{\grt} \ar[lu]_-i \ar[dl]^-i & \scrf^+ \ar[ru]^-{i^+} \ar[rd]_-{\psi} & {} \\
    \wt{\scrf_P} \ar@{=}[rrr] & {} & { } & \wt{\scrf_P}
    }
    \]
    を得る。この図式をぐっとにらむと、$i = \wt{\psi} \circ \wt{\grt} \circ i$であることがわかる。$\wt{\psi} \circ \wt{\grt}$の行き先$\wt{\scrf_P}$は摩天楼層なので、茎の普遍性により$id = \wt{\psi} \circ \wt{\grt}$である。

    また、再び図式をぐっとにらむことにより$\wt{\grt} \circ \wt{\psi} \circ i^+ \circ \grt = i^+ \circ \grt$
    がわかる。$\wt{\grt} \circ \wt{\psi} \circ i^+$の行き先$\wt{\scrf^+_P}$は層なので、層化の普遍性により$\wt{\grt} \circ \wt{\psi} \circ i^+ = i^+$がわかる。さらに、$\wt{\scrf^+_P}$は摩天楼層なので、茎の普遍性から$id = \wt{\grt} \circ \wt{\psi}$である。

    したがって自然同型$\wt{\scrf_P} \cong \wt{\scrf^+_P}$がわかるので、とくに$\scrf_P \cong \scrf^+_P$である。
\begin{comment}
    前半を示す。(層化の普遍性を使って示したいところだが、その方法は採らない。準備が足りないからである) 射$\grt \colon \scrf \to \scrf^+$は、すべての$P$の開近傍$U$について次の図式
\[
\xymatrix{
\scrf(U) \ar[r]^-{\grt_U} \ar[d] & \scrf^+(U) \ar[d] \\
\scrf_P \ar[r]^-{\grt_P} & \scrf^+_P
}
\]
    が可換になるような準同形$\grt_P \colon \scrf_P \to \scrf^+_P$を誘導する。また、準同形$\psi_U \colon \scrf^+(U) \to \scrf_P$を、$\psi_U(f)=f(P)$で定める。このときすべての$P$の開近傍の列$V \subset U$について次の図式
    \[
    \xymatrix{
    \scrf^+(U) \ar[r]^-{\psi_U} \ar[d] & \scrf_P  \\
    \scrf^+(V) \ar[ur]_-{\psi_V} & {}
    }
    \]
    は可換である。したがって茎の普遍性により、ある準同形$\wt{\psi}$が存在して、すべての$P$の開近傍$U$に対して次の図式
\[
\xymatrix{
\scrf^+(U) \ar[r]^-{\psi_U} \ar[d] & \scrf_P  \\
\scrf^+_P \ar[ur]_-{\wt{\psi}} & {}
}
\]
が可換になる。

さて$\gra \in \scrf^+_P$とする。ある$(V,f) \in \coprod_{P \in V} \scrf^+(V)$があって、$\gra = f_P$と表せる。$f \in \scrf^+(V)$なので、さらに$f|_W = \ol{s}$となるような$(W,s) \in \coprod_{P \in W \subset V} \scrf(W)$がとれる。このとき
\begin{align*}
  \grt_P \circ \wt{\psi} (\gra) &= \grt_P \circ \wt{\psi} (f_P) \\
  &= \grt_P \circ \psi_V (f) \\
  &= \grt_P (f(P)) \\
  &= \grt_P (s_P) \\
  &= \grt_W (s)_P \\
  &= (\ol{s})_P \\
  &= (f|_W)_P \\
  &= f_P \\
  &= \gra
\end{align*}
だから、$\grt_P \circ \wt{\psi} = id$がいえた。

逆に$\beta \in \scrf_P$とする。$\beta = s_P$なる$(V,s) \in \coprod_{P \in V} \scrf(V)$がある。このとき
\begin{align*}
  \wt{\psi} \circ \grt_P(\beta ) & = \wt{\psi} \circ \grt_P(s_P) \\
  &= \wt{\psi} (\grt_V (s)_P ) \\
  &= \psi_V ( \grt_V (s)) \\
  &= \psi_V (\ol{s}) \\
  &= \ol{s}(P) \\
  &= s_P \\
  &= \beta
\end{align*}
であるから、$\wt{\psi} \circ \grt_P = id$である。これで同型がいえた。
\end{comment}
\item[Step 2] 後半を示す。一般には、忘却関手の左随伴(自由対象)は、このような性質を満たさない。例として、自由群を与える関手を考えてみればわかる。しかし層の圏から前層の圏への忘却関手は忠実充満なので、これが成り立つということに注意する。さて$\scrf$は層なので、層化の普遍性により次の図式
\[
\xymatrix{
\scrf \ar[r]^-{\grt} \ar[rd]_-{id} & \scrf^+ \ar@{.>}[d]^-{\psi} \\
{} & \scrf
}
\]
を可換にするような$\psi$が存在する。このとき定義により$\psi \circ \grt = id$である。さらに
\[
\xymatrix{
\scrf \ar[r]^-{\grt} \ar[rd]_-{\grt} & \scrf^+ \ar[d]^-{\grt \circ \psi} \\
{} & \scrf^+
}
\]
は可換なので、射の一意性から$\grt \circ \psi = id$も成り立つ。
  \end{description}
\end{proof}


\bfsubsection{定義-部分層}
\barquo{
任意の点$P$に対して$\scrf'_P$は$\scrf_P$の部分群となっていることが従う。
}
\begin{rem}
  これは順極限の完全性から従うので、$\scrf$が層であるという仮定は必要ない。前層で十分である。
\end{rem}


\bfsubsection{定義-層の像}
\barquo{
自然な射$\im \vp \to \scrg$がある。この射は実際単射(Ex. 1.4 を見よ)であり、したがって$\im \vp$は$\scrg$の部分層と同一視できる。
}
\begin{rem}
正しい順序で証明を行い、循環論法にならないようにするために叙述の順序をいままでの形式(本文$\to$演習)から変更する。
\end{rem}

\prop{
(層化と茎の合成と茎の自然同型) \\
茎をとる関手$(\cdot)_P \colon \PSh \to \Ab$と層化してから茎をとる関手$(\cdot^{+})_P \colon \PSh \to \Ab$は自然同型である。すなわち前層の射$\vp \colon \scrf \to \scrg$が与えられたとき
次の図式
\[
\xymatrix{
(\scrf^+)_P \ar[r]^-{(\vp^+)_P} & (\scrg^+)_P \\
\scrf_P \ar[r]^-{\vp_P} \ar[u]^-{iso} & \scrg_P \ar[u]_-{iso}
}
\]
は可換である。
}
\begin{proof}
面倒なので茎の上の摩天楼層と茎をとくに区別しないで書く。次の立方体の形をした図式
\[
\xymatrix{
{}  & (\scrf^+)_P \ar[rr]^-{(\vp^+)_P } & {} & (\scrg^+)_P \\
\scrf_P \ar[ru]^-{iso}  \ar[rr]^(.7){\vp_P} & { } & \scrg_P \ar[ru]^-{iso} & {} \\
{} & \scrf^+ \ar[uu] \ar[rr]^(.7){\vp^+} & { } & \scrg^+ \ar[uu] \\
\scrf \ar[uu]^-i \ar[rr]^-{\vp} \ar[ru] & {} & \scrg \ar[uu] \ar[ru]
}
\]
を考える。この図式において、上の面以外の$5$つの面の可換性はすでに示されている。したがって、上の面と$i$の合成を考えると、茎の普遍性から上の面の可換性がいえる。よって示すべきことがいえた。
\end{proof}


\prop{
(演習1.4 (a)) \\
$\vp \colon \scrf \to \scrg$を前層のあいだの射とする。$\vp$が単射ならば、層化が誘導する射$\vp^{+} \colon \scrf^{+} \to \scrg^{+}$も単射である。
}
\begin{proof}
命題1.1より、$(\vp^{+})_P \colon (\scrf^+ )_P \to (\scrg^+)_P$がすべて単射であることを示せば十分。前の命題から、次の図式
\[
\xymatrix{
(\scrf^+)_P \ar[r]^-{(\vp^+)_P} & (\scrg^+)_P \\
\scrf_P \ar[r]^-{\vp_P} \ar[u]^-{iso} & \scrg_P \ar[u]_-{iso}
}
\]
は可換である。したがって、順極限の完全性より下辺の射は単射だから、示すべきことがいえる。
\end{proof}


\bfsubsection{定義-商層}
\barquo{
商層$\scrf / \scrf'$を前層$U \to \scrf(U) / \scrf'(U)$に付随する層と定義する。任意の点$P$に対して、茎$(\scrf/ \scrf')_P$は茎の商$\scrf_P / \scrf'_P$となることが分かる。
}
\begin{rem}
  対応$U \to \scrf(U) / \scrf'(U)$が前層を定めることの証明は省略。後半の茎についての命題は、次の演習1.2を余核について適用すればわかる。
\end{rem}


\prop{
(演習1.2(a)の拡張) \\
前層の射$\vp \colon \scrf \to \scrg$と各点$P \in X$に対して、次が成り立つ。ただしシャープは前層としての像・余核をとっていることを表す。
\begin{description}
  \item[(1)] $\scrf_P$の部分群として$(\Ker \vp)_P = \Ker \vp_P$である。また$\scrg_P$の商群として$(\Coker\sh \vp)_P = \Coker \vp_P$である。さらに$\scrg_P$の部分群として$(\Im\sh \vp)_P = \Im \vp_P$である。
  \item[(2)] $\scrf$と$\scrg$が層ならば、$\scrg_P$の商群として$(\Coker \vp)_P = \Coker \vp_P$である。さらに$\scrg_P$の部分群として$(\Im \vp)_P = \Im \vp_P$である。
\end{description}
}
\begin{proof} ${}$
  \begin{description}
    \item[(1)]
    次の前層の圏における完全列がある。
      \[
      \xymatrix{
      0 \ar[r] & \Ker \vp \ar[r] & \scrf \ar[r]^-{\vp} & \scrg
      }
      \]
      順極限の完全性から、任意の$P \in X$について次も完全である。
      \[
      \xymatrix{
      0 \ar[r] & (\Ker \vp)_P \ar[r]  & \scrf_P \ar[r]^-{\vp_P}  & \scrg_P
      }
      \]
      したがって$\scrf_P$の部分群として$\Ker \vp_P = (\Ker \vp)_P$である。つまり、次の図式
    \[
    \xymatrix{
    {} & \scrf_P \\
    (\Ker \vp)_P \ar[ru] \ar[r]  & \Ker \vp_P \ar[u]
    }
    \]
    が可換になるような同型$(\Ker \vp )_P \to \Ker \vp_P$がある。
      同様にして$\scrg_P$の商群として$\Coker \vp_P = (\Coker\sh \vp)_P$であることがわかる。つまり次の図式
\[
\xymatrix{
\scrg_P \ar[r] \ar[d] & (\Coker\sh \vp)_P \\
\Coker \vp_P \ar[ru]  &
}
\]
が可換になるような同型$\Coker \vp_P \to (\Coker\sh \vp)_P$がある。
      最後に$\Im\sh$について考える。自然な写像$\scrg \to \Coker\sh \vp$を$\psi$とおく。$\psi_P \circ \vp_P = 0$より$\Im \vp_P \subset \Ker \psi_P$である。したがって、次の各行が完全であるような可換図式
      \[
      \xymatrix{
0 \ar[r] & \Ker \psi_P \ar[r] & \scrg_P \ar[r]^-{\psi_P} \ar@{=}[d] & (\Coker \vp)_P \\
0 \ar[r] & \Im \vp_P \ar[u] \ar[r]  & \scrg_P \ar[r] & \Coker \vp_P \ar[u]^-{iso}
      }
      \]
      が得られる。したがって5-lemmaにより、自然な写像$\Im \vp_P \to \Ker \psi_P$は同型である。よって
\[
\xymatrix{
(\Im\sh \vp)_P \ar[r]  \ar@{=}[d] & \scrg_P \ar@{=}[d] \\
(\Ker \psi)_P \ar[r]  \ar[d]_-{iso} & \scrg_P \ar@{=}[d] \\
\Ker \psi_P \ar[r]  \ar[d]_-{iso} & \scrg_P \ar@{=}[d] \\
\Im \vp_P \ar[r]   & \scrg_P
}
\]
  は可換。したがって、$\scrg_P$の部分群として$(\Im\sh \vp)_P = \Im \vp_P$である。
      \item[(2)] 層化の茎はもとの前層の茎と同型であることから、(1)により従う。
  \end{description}
\end{proof}


\bfsubsection{警告 1.2.1}
\barquo{
しかしながら、$\vp$が全射であるのはおのおのの$P$に対して茎の間の写像$\vp_P \colon \scrf_P \to \scrg_P$が全射であるとき、またそのときに限る、ということはいえる。さらに一般に、層と射の列が完全であるのは、茎に誘導される列が完全であるとき、またそのときに限る(Ex. 1.2) 以上のことは再び層が局所的なものであるという性質を示している。
}

\prop{
(演習 1.2 (b)) \\
前層の間の射$\vp \colon \scrf \to \scrg$があり、$\scrg$は層であるとする。このとき$\vp$が全射であることと、すべての$P$について$\vp_P$が全射であることは同値である。
}
\begin{proof}
  命題1.1を使う。
  \begin{align*}
    \vp \text{が全射} &\iff \Im \vp = \scrg \\
    &\iff \Im \vp \to \scrg \text{が同型} &(\text{演習1.4(a)より}) \\
    &\iff \forall P \in X \quad (\Im \vp \to \scrg)_P \text{が同型} &(\text{$\scrg$が層より}) \\
    &\iff \Im \vp_P \to \scrg_P \text{が同型} \\
    &\iff \vp_P \text{が全射}
  \end{align*}
  より、示すべきことがいえる。
\end{proof}





\prop{
(演習1.2 (c)) \\
$3$つの層とその間の射
\[
\xymatrix{
\scrf \ar[r]^{\vp} & \scrg \ar[r]^{\psi} & \scrh
}
\]
があったとし、すべての$P \in X$について
\[
\xymatrix{
\scrf_P \ar[r]^{\vp_P} & \scrg_P \ar[r]^{\psi_P} & \scrh_P
}
\]
が完全だとする。このとき$\psi \circ \vp = 0$であって、かつ
\[
\xymatrix{
\scrf \ar[r]^{\vp} & \scrg \ar[r]^{\psi} & \scrh
}
\]
は完全である。
}
\begin{proof}
$\Im \vp_P = \Ker \psi_P$であるから層$\Ker \psi$と$\Im \vp$はすべての茎が一致する。ゆえに同じ層である…といいたいところだが、その前に$\Ker \psi$と$\Im \vp$の間に射が存在するということを示さなくてはならない。
  %(\textblue{反例はあるだろうか?}つまり、すべての茎が同型だが、互いに自然同型ではないような二つの層の例はあるだろうか。あるいは、ゼロでないような前層$\scrf$で、層化がゼロになるものがあるか、と言ってもよい)
  状況から言って、$\Ker \psi$は$\Im \vp$を部分層として含んでいることが予想されるが、それはあきらかではなく、層であるということを用いて示す必要がある。

  \lem{
  (局所的にゼロならもともとゼロ) \\
  2つの層の間の射$\vp \colon  \scrf \to \scrg$があり、すべての点$P$について$\vp_P = 0 $が成り立つとする。このとき$\vp = 0$である。
  }
  \begin{proof}
    $\scrf_P$の部分群として
    \begin{align*}
      (\Ker \vp)_P &= \Ker \vp_P &(\text{演習1.2(a)による}) \\
      &= \scrf_P
    \end{align*}
    である。したがって自然な写像$\Ker \vp \to \scrf$を$j$とおくと、$j_P$はすべて同型である。ゆえに$\scrf$と$\scrg$が層であることにより$\Ker \vp$は層なので、命題1.1より$j$は同型である。よって$\Ker \vp = \scrf$であり、これは$\vp = 0$を意味する。
  \end{proof}




  \lem{
  前層の列
  \[
  \xymatrix{
  \scrf \ar[r]^{\vp} & \scrg \ar[r]^{\psi} & \scrh
  }
  \]
  が与えられたとし、うち$\scrg$と$\scrh$は層であるとする。このとき次は同値。
  \begin{description}
    \item[(1)] $\psi \circ \vp = 0$
    \item[(2)] $\Im \vp$は$\Ker \psi$の部分層。
  \end{description}
  }
  \begin{proof} ${}$
    \begin{description}
      \item[(2)$\To$(1)] 前層の像を$\sharp$をつけて表すことにする。そうすると、次の図式
      \[
      \xymatrix{
      \Im \vp \ar[r]  \ar[rd] & \Ker \psi \ar[r] \ar[d] & \scrg \ar[d]^0 \\
      \Im\sh \vp \ar[r] \ar[u] & \scrg \ar[r]^-{\psi} & \scrh
      }
      \]
      は可換。よって$\psi \circ \vp = 0$である。
      \item[(1)$\To$(2)] 前層核の普遍性から、次を可換にするような$i$がある。
      \[
      \xymatrix{
      \Im\sh \vp \ar[rrd] \ar[ddr] \ar[rd]^-{i} & & \\
       & \Ker \psi \ar[r] \ar[d] & \scrg \ar[d]^-{0} \\
       & \scrg \ar[r]^-{\psi} & \scrh
      }
      \]
      $i$は単射であることに気を付ける。ここで$\scrg$と$\scrh$は層という仮定により$\Ker \psi$は層。したがって層化の普遍性により次の図式
      \[
      \xymatrix{
      \Im\sh \vp \ar[r] \ar[rd]_-{i} & \Im \vp \ar[d]^-{j} \\
      & \Ker \psi
      }
      \]
      を可換にするような$j$があり、$j$は単射により誘導されているので演習1.4(a)により単射である。

      「部分層に同型」よりもっと強く「部分層」であるということを示したいので、まだ示すべきことがある。それは次の図式
      \[
      \xymatrix{
      & \scrg \\
      \Im \vp \ar[r]^-j \ar[ru] & \Ker \psi \ar[u]
      }
      \]
      の可換性である。これは次の二つの図式
      \[
      \xymatrix{
      \Im\sh \vp \ar[r] \ar[rd]_-i \ar[rdd] & \Im \vp \ar[d]^-j & \Im\sh \vp \ar[r] \ar[rdd] & \Im \vp \ar[dd] \\
      {} & \Ker \psi \ar[d] & {} & { } \\
      {} & \scrg & {} & \scrg
      }
      \]
      の可換性と、層化の普遍性から従う。
    \end{description}
  \end{proof}

  \lem{
  (完全性の局所判定) \\
  前層の列
  \[
  \xymatrix{
  \scrf \ar[r]^{\vp} & \scrg \ar[r]^{\psi} & \scrh
  }
  \]
  が与えられたとし、うち$\scrg$と$\scrh$は層であるとする。このとき次は同値。
  \begin{description}
    \item[(1)] 与えられた前層の列
    $
    \xymatrix{
    \scrf \ar[r]^{\vp} & \scrg \ar[r]^{\psi} & \scrh
    }
    $
    は完全。
    \item[(2)] $\psi \circ \vp = 0$かつ、任意の点$P \in X$に対して
    $
    \xymatrix{
    \scrf_P \ar[r]^{\vp_P} & \scrg_P \ar[r]^{\psi_P} & \scrh_P
    }
    $
    は完全。
  \end{description}
  }
  \begin{proof}
    (1)$\To$(2)はすでに示されている。(2)$\To$(1)を示そう。はじめ$(\scrf / \scrf')_P = \scrf_P / \scrf'_P$を使う方針で考えたが、それだと$\scrf / \scrf' = 0$から$(\scrf / \scrf')\sh = 0$がいえるか、という問題を考える必要がありそうである。したがってこの方針は採らなかった。

    まず仮定から、補題を使えば$\Im \vp$は$\Ker \psi$の部分層であることがわかる。したがって、包含射$j \colon \Im \vp \to \Ker \psi$がある。茎に誘導される写像$j_P \colon (\Im \vp)_P \to (\Ker \psi)_P$を考える。
  次の図式
  \[
  \xymatrix{
  {} & {} & \scrg_P & {} & { } \\
  (\Im \vp)_P \ar[urr] \ar[rrrr]  & {} & {} & {} & (\Ker \psi )_P \ar[llu] \\
  {} & {} & {} & {} & {} \\
  {} & \Im \vp_P \ar[luu]_-{iso} \ar@{=}[rr] \ar[uuur] &  { } & \Ker \psi_P \ar[luuu] \ar[uur]_-{iso}
   }
  \]
  の、$=$と二つの同型を含む逆さ台形以外のところは可換である。したがって$(\Ker \psi)_P \to \scrg_P$は単射なので、「$=$と二つの同型を含む逆さ台形」の可換性がいえる。
  よって$j_P$は同型なので$j$ははじめから同型である。ゆえに$\scrg_P$の部分群として$\Im \vp = \Ker \psi$が成り立つ。
\end{proof}

\end{proof}



\prop{
(層の条件が必要であること) \\
ゼロでない前層$\scrf$であって、すべての茎がゼロであるようなものがある。
}
\begin{proof}
  簡単のため$X = \R$としておく。$\scrf$の制限写像を$V \subsetneq U$に対して$\scrf(U) \to \scrf(V)$がゼロ写像になるように定める。そのとき、茎への自然な写像$\scrf \to \scrf_P$はゼロ写像である。(行先は摩天楼層だと考えよ) このとき次の2つの図式
  \[
  \xymatrix{
\scrf \ar[r] \ar[dr] & \scrf_P \ar[d]^0 & \scrf \ar[r] \ar[rd] & \scrf_P  \ar[d]^{id} \\
{} & \scrf_P & {} & \scrf_P
  }
  \]
  は可換なので、茎の普遍性により$id = 0$, つまり$\scrf_P = 0$である。一方で、$\scrf(U) = \Z$とでもすれば$\scrf$はゼロではない。
\end{proof}


\bfsubsection{定義-順像・逆像}
\barquo{
$X$上の任意の層$\scrf$に対し、$Y$上の層の順像$f_*\scrf$を$Y$の任意の開集合$V$に対して$(f_*\scrf)(V) = \scrf(f^{-1}(V))$と置くことにより定義する。
}
\begin{rem}
これが実際に前層となり、層の条件を満たしていることの証明は省略する。
\end{rem}


\bfsubsection{定義-順像・逆像}
\barquo{
$Y$上の任意の層$\scrg$に対して$X$上の層の逆像$f^{-1}\scrg$を前層$U \mapsto \lim_{V \supset f(U)} \scrg(V)$に付随する層と定義する。
}
\begin{rem}
$\lim$とあるのは順極限であることを注意しておく。$f\sh\scrf (U) =  \rlim_{V \supset f(U)} \scrg(V)$が実際に制限写像を誘導することを確認しておこう。$U_1 \subset U_2$とする。このとき$V \supset f(U_1)$についてなす射の族
\[
\xymatrix{
\scrg(V) \ar[r] & \rlim_{V \supset f(U_1)} \scrg(V)
}
\]
の自然性により、余極限の普遍性から$V \supset f(U_2)$について次の図式
\[
\xymatrix{
\scrg(V) \ar[r] \ar[d] & \rlim_{V \supset f(U_1)} \scrg(V) \\
\rlim_{V \supset f(U_2)} \scrg(V) \ar[ru] & {}
}
\]
を可換にするような射$f\sh \scrg(U_2) \to  f\sh\scrg(U_1)$が存在する。これが関手性を満たすことの証明は省略する。

なぜこのような定義になっているのか説明する。$\scrg(f(U))$で定義できるといいのだが、それはできない。$f(U)$は開集合とは限らないし、$f(U)$を含む最小の開集合も一般には存在しないためである。よってこのような定義をしているのだと思われる。
\end{rem}


\begin{comment}
\prop{
  (随伴はKan拡張) \\
  圏$\bfc$, $\bfd$と関手$F \colon \bfc \to \bfd$, $G \colon \bfd \to \bfc$があり、随伴$F \dashv G$があるとする。このとき随伴の単位射を$\eta \colon 1_{\bfc} \To GF$, 余単位射を$\ve \colon FG \To 1_{\bfd}$とすると次が成り立つ。
  \begin{description}
\item[(1)] $(G,\eta)$は$F$に沿った$1_{\bfc}$の左Kan拡張である。したがって$G \cong \Lan_F 1_{\bfc}$が成り立つ。
\[
\xymatrix{
\bfc \ar[rd]_-F \ar[rr]^-{id} & {} \ar@{}[d]|{\Downarrow \eta} & \bfc \\
{} &  \bfd \ar[ru]_{G} & {}
}
\]
\item[(2)] $(F, \ve)$は$G$に沿った$1_{\bfd}$の右Kan拡張である。したがって$F \cong \Ran_G 1_{\bfd}$が成り立つ。
\[
\xymatrix{
\bfd \ar[rd]_-G \ar[rr]^-{id} & {} \ar@{}[d]|{\Uparrow \ve} & \bfd \\
{} &  \bfc \ar[ru]_{F} & {}
}
\]
  \end{description}
}

\begin{proof}
  Riehl\cite{Riehl}命題6.5.2を参照のこと。
\end{proof}
\end{comment}



\bfsubsection{定義-順像・逆像}
\barquo{
$f_*$は$X$の上の層の圏$\Ab(X)$から$Y$の上の層の圏$\Ab(Y)$への関手であることに注意せよ。同様に$f^{-1}$は$\Ab(Y)$から$\Ab(X)$への関手である。
}
\begin{proof} ${}$
\begin{description}
\item[Step 1] まず順像の関手性について説明する。$X$上の前層の間の射$\vp \colon \scrf \to \scrg$が与えられたとする。このとき$Y$上の前層の間の射$f_* \vp \colon f_* \scrf \to f_* \scrg$は
$f_* \vp(V) = \vp_{f^{-1}(V)}$により定めればよい。これが関手性を満たすことの証明は省略する。
\item[Step 2] $Y$上の層の射$\vp \colon \scrf \to \scrg$が与えられたとする。$U \opsub X$としよう。$\vp$は自然変換なので、すべての$V \supset f(U)$に対して次の図式
\[
\xymatrix{
\rlim_{V \supset U} \scrf(V) \ar[r]^-{f\sh\vp_U } & \rlim_{V \supset U} \scrg(V) \\
\scrf(V) \ar[u] \ar[r]^-{\vp_V} & \scrg(V) \ar[u]
}
\]
が可換になるような$f\sh\vp_U$がある。

$f\sh\vp_U$が自然変換になっていることを示そう。$U_1 \subset U_2$とする。このときすべての$V \supset f(U_2)$に対して次の図式
\[
\xymatrix{
  \scrf(V) \ar[rrr]^-{\vp_V} \ar[dd] \ar[rd]  & {} & {} &   \scrg(V) \ar[dd] \ar[dl] \\
{} & f\sh \scrf(U_2) \ar[r]^{f\sh\vp_{U_2}} \ar[dl]  &  f\sh \scrg(U_2) \ar[rd] & {}  \\
f\sh \scrf(U_1) \ar[rrr]^{f\sh\vp_{U_1}}  & {} & {} & f\sh \scrg(U_1)
}
\]
の底部台形以外は可換。$\scrf(V) \to \rlim_{V \supset f(U_2)} \scrf(V)$との合成を考えて余極限の普遍性を用いることにより、底部の可換性がいえる。
関手性の証明の残りの部分(合成を保つ)の証明は省略する。したがって$\scrf$に前層$f\sh\scrf$を対応させることは関手を与える。あとはこれの層化を考えればよい。
\end{description}
\end{proof}




\bfsubsection{定義 層の制限}
\barquo{
$i \colon Z \to X$を包含写像とし、$\scrf$を$X$上の層とする。このとき、$i^{-1}\scrf$を$\scrf$の$Z$への制限と呼び、しばしば$\scrf|_Z$と書く。任意の点$P \in Z$における$\scrf|_Z$の茎は$\scrf_P$に他ならないことに注意する。
}
\begin{rem}
やや一般化して、次の命題の形で示す。
\end{rem}

\prop{
(層の逆像の茎)\\
$f \colon X \to Y$を連続写像とし、$\scrf$は$Y$上の層であるとする。このとき$f^{-1}\scrf$の点$P \in X$における茎$(f^{-1}\scrf)_P$は$\scrf_{f(P)}$に同型である。
}
\begin{proof}
  層化によって茎は変わらないので、はじめから$\scrf$は前層だとしてよいし、$f^{-1}\scrf$の代わりに層化を取る前の$f\sh\scrf$について示せばよい。
  $P \in U$が与えられたとする。$V \supset f(U)$だとする。$Q = f(P)$とおいておく。このとき次の図式
\[
\xymatrix{
\scrf(V) \ar[r] \ar[d] & \scrf_Q \ar@{=}[d] \\
f\sh\scrf(U) \ar[d] \ar[r] & \scrf_Q \ar@{=}[d] \\
(f\sh\scrf)_P \ar[r]^-{\vp} & \scrf_Q
}
\]
  が可換になるような準同型$\vp$がある。(この図式の2段目の$f\sh\scrf(U) \to \scrf_Q$がまず誘導され、次に3段目の$\vp$が誘導されるという風に追う。自然性の確認は省略)

  同じ状況$P \in U$, $V \supset f(U)$でさらに考える。このとき$Q \in V$が与えられており、$P \in U \subset f^{-1}(U)$となるように$U$をとったとも思える。このとき次の図式
  \[
  \xymatrix{
  \scrf(V) \ar[d] \ar[r] & \scrf_Q \ar[d]^-{\psi} \\
f\sh\scrf(U) \ar[r] & (f\sh\scrf)_P
  }
  \]
  が可換になるような$\psi$がある。まとめると、次の図式
  \[
  \xymatrix{
  {}  & \scrf(V) \ar[rr] \ar[ld] \ar[dd] & {} & \scrf_Q \ar@{=}[ld] \ar[dd]^-{\psi} \\
  f\sh\scrf(U) \ar[dd]  \ar[rr] & { } & \scrf_Q \ar@{=}[dd] & {} \\
  {} & f\sh\scrf(U)  \ar[rr] \ar[dl] & { } & (f\sh\scrf )_P \ar@{=}[dlll] \ar[dl]^-{\vp} \\
   (f\sh\scrf )_P  \ar[rr]^-{\vp}  & {} & \scrf_Q & {}
  }
  \]
  の右の面以外は可換ということになる。茎の普遍性より、右の面も可換であることがいえる。よって$\vp \circ \psi = id$である。逆の$\psi \circ \vp = id$も図式の追跡と余極限の普遍性から従う。
\end{proof}

\begin{ano}
より簡潔な証明がある。茎と逆像はともに順極限で定義されていたのだった。写像$f \colon X \to Y$が与えられているとする。$\scrf$を$X$上の前層とし、$i \colon \{ * \} \to X$を点$\frakp \in X$への写像とする。このとき定義から$\scrf_{\frakp} = i\sh\scrf(\{*\})$が成り立つ。よって任意の$\frakp \in X$と$Y$上の前層$\scrg$に対して
\begin{align*}
  (f\sh\scrg)_{\frakp} &= (i\sh (f\sh\scrg))(\{*\}) \\
  &= (f \circ i)\sh \scrg (\{*\}) \\
  &= \scrg_{f(\frakp)}
\end{align*}
である。

\end{ano}






\bfsubsection{演習問題 1.18}
以下$\scrf$は$X$上の前層、$\scrg$は$Y$上の前層、$U$は$X$の開集合で$V$は$Y$の開集合とする。層化の普遍性により、$\scrf$も$\scrg$も前層であるとして逆像を前層の範囲で考えても十分である。
\begin{description}
  \item[Step 1] 自然変換$\ve \colon f\sh f_* \to 1$を構成しよう。$f(U) \subset V$とする。このとき次の図式
  \[
  \xymatrix{
  \scrf(f^{-1}(V)) \ar[r] \ar[d] & \scrf(U) \\
  \rlim_{f(U) \subset V} \scrf(f^{-1}(V)) \ar[ru] &
  }
  \]
  が可換になるような$\ve_{U} \colon \rlim_{f(U) \subset V} \scrf(f^{-1}(V)) \to \scrf(U)$がある。これが自然性を満たすことはあきらか。なお、本来は$(\ve_{\scrf})_U$などと書くべきだが煩わしいので省略した。$\ve$は各元$[s] \in f\sh f_*(U)$を次のように写す。
  \[
  \ve_U([s]) = s|_U
  \]
  ただし$s \in \scrf(f^{-1}(V))$は代表元である。
  \item[Step 2] 自然変換$\eta \colon 1 \to f_* f\sh $を構成しよう。以下、本来なら$\eta_{\scrg}$などと書くべきところでも記号の乱用により$\eta$と書く。

  あきらかに$f(f^{-1}(V)) \subset V$が成り立つので、自然な写像
  \[
  \scrg(V) \to \rlim_{f(f^{-1}(V)) \subset V' } \scrg(V')
  \]
  がある。これを$\eta_V$とすればよい。$\eta_V$は$\eta_V(s) = [s] $という写像である。
  \item[Step 3] 対応$\flat \colon \Hom_X(f\sh\scrg , \scrf) \to \Hom_Y(\scrg , f_*\scrf)$を構成しよう。これは$\vp \colon f\sh\scrg \to \scrf$に対して合成$\vp\fl = f_* \vp \circ \eta$により定める。具体的には
  \[
  \vp\fl_V (s) = \vp_{f^{-1}(V)} ([s])
  \]
  と表せる。
  \item[Step 4] 対応$\sharp \colon \Hom_Y(\scrg , f_*\scrf) \to \Hom_X(f\sh\scrg , \scrf)$を構成しよう。これは$\psi \colon \scrg \to f_* \scrf$に対して合成$\psi\sh = \ve \circ f\sh \psi$により定める。具体的には
  \[
  \psi\sh_U([s]) = \psi_V(s) |_U
  \]
  と表せる。ただし$V$は$s \in \scrg(V)$なるものをとる。
  \item[Step 5] $(\vp\fl)\sh = \vp$を示そう。これは
  \begin{align*}
    (\vp\fl)\sh_U([s]) &= \vp\fl_V (s) |_U \\
    &= \vp_{f^{-1}(V)} ([s]) |_U \\
    &= \vp_U ( [s] |_U  ) \\
    &= \vp_U([s])
  \end{align*}
  よりわかる。
  \item[Step 6] $(\psi\sh)\fl = \psi$を示そう。これは
  \begin{align*}
    (\psi\sh)\fl_V(s) &= \psi\sh_{f^{-1}(V)}([s]) \\
    &= \psi_V(s)|_{f^{-1}(V)} \\
    &= \psi_V(s)
  \end{align*}
  よりわかる。
  \item[Step 7] この全単射を与える対応が自然性を満たすことをいおう。射$g \colon \scrf_1 \to \scrf_2$が与えられたとき次の図式
  \[
  \xymatrix{
  \Hom_X(f\sh\scrg, \scrf_1) \ar[r]^-{\flat} \ar[d]^-{g_*} & \Hom_Y(\scrg,f_*\scrf_1) \ar[d]^-{(f_*g)_*} \\
  \Hom_X(f\sh\scrg, \scrf_2) \ar[r]^-{\flat} & \Hom_Y(\scrg,f_*\scrf_2)
  }
  \]
  が可換であることを確認すればいい。$\vp \in \Hom_X(f\sh\scrg, \scrf_1)$, $s \in \scrg(V)$が与えられたとき
  \begin{align*}
    (f_* g)_V \circ \vp\fl_V(s) &= (f_*g)_V ( \vp_{f^{-1}(V)}([s]) ) \\
    &= g_{f^{-1}(V)} \circ \vp_{f^{-1}(V)} ([s]) \\
    &= (g \circ \vp)_{ f^{-1}(V)} ([s]) \\
    &= (g \circ \vp)\fl_V (s)
  \end{align*}
  より$(f_* g) \circ \vp\fl = (g \circ \vp)\fl$だから示すべきことがいえた。
    \item[Step 8] 次に$\scrg$に関する自然性を示そう。射$h \colon \scrg_1 \to \scrg_2$が与えられたとき、図式
    \[
    \xymatrix{
    \Hom_X(f\sh\scrg_2, \scrf) \ar[r]^-{\flat} \ar[d]^-{(f\sh h)^*} & \Hom_Y(\scrg_2,f_*\scrf_) \ar[d]^-{h^*} \\
    \Hom_X(f\sh\scrg_1, \scrf) \ar[r]^-{\flat} & \Hom_Y(\scrg_1,f_*\scrf)
    }
    \]
    が可換であることを確認すればよい。これは$\vp \in \Hom_X(f\sh\scrg_2, \scrf)$, $s \in \scrg(V)$が与えられたとき
    \begin{align*}
      (\vp\fl \circ h)_V (s) &= \vp\fl_V \circ h_V (s) \\
      &= \vp_{f^{-1}(V)} ( [h_V(s)]) \\
      &= \vp_{f^{-1}(V)} \circ f\sh h_{f^{-1}(V)} ([s]) \\
      &= (\vp \circ f\sh h)_{f^{-1}(V)} ([s]) \\
      &= (\vp \circ f\sh h )_V\fl (s)
    \end{align*}
    であることからわかる。
\end{description}

\begin{ano}
  Kan拡張の性質を利用した別証がある。
\end{ano}

  \begin{definition}
    $\bfc$, $\bfd$, $\bfe$は圏であるとする。関手$F \colon \bfc \to \bfe$と$K \colon \bfc \to \bfd$が与えられているとする。
  \begin{description}
    \item[(1)] このとき、$F$の$K$に沿った左Kan拡張とは、次の図式
    \[
    \xymatrix{
    \bfc \ar[rd]_K \ar[rr]^F & {} \ar@{}[d]|{\Downarrow \eta} & \bfe \\
    {} &  \bfd \ar[ru]_{\Lan_K F} &{}
    }
    \]
    に示されるような関手$\Lan_K F \colon \bfd \to \bfe$と自然変換$\eta \colon F \To \Lan_K F \cdot K$の組$(\Lan_K F, \eta)$であって、以下の普遍性を満たすものである。つまり、任意の関手$G \colon \bfd \to \bfe$
    と自然変換$\grg \colon F \To G K$の組$(G,\grg)$に対して、自然変換$\gra \colon \Lan_K F \To G$であって図式
    \[
  \xymatrix{
  F \ar[r]^-{\eta} \ar[rd]_-{\grg} & \Lan_K F \cdot K \ar@{.>}[d]^{\gra K} & \Lan_K F \ar@{.>}[d]^{\gra} \\
  {} & GK & G
  }
    \]
    を可換にするものが存在して、かつ一意である。
    \item[(2)] このとき、$F$の$K$に沿った右Kan拡張とは、次の図式
    \[
    \xymatrix{
    \bfc \ar[rd]_K \ar[rr]^F & {} \ar@{}[d]|{\Uparrow \ve} & \bfe \\
    {} &  \bfd \ar[ru]_{\Ran_K F} &{}
    }
    \]
    に示されるような関手$\Ran_K F \colon \bfd \to \bfe$と自然変換$\ve \colon \Ran_K F \cdot K \To F$の組$(\Ran_K F, \ve)$であって、次の普遍性を満たすものである。つまり、任意の関手$G \colon \bfd \to \bfe$
    と自然変換$\grd \colon G K \To F $の組$(G,\grd)$に対して、自然変換$\beta \colon G \To \Ran_K F $であって図式
    \[
    \xymatrix{
    F   & \Ran_K F \cdot K \ar[l]_-{\ve}  & \Ran_K F  \\
    {} & GK \ar@{.>}[u]^{\beta K} \ar[lu]^-{\grd} & G \ar@{.>}[u]_{\beta}
    }
    \]
    を可換にするものが存在して、かつ一意である。
  \end{description}

  \end{definition}



  \prop{
  (前層の逆像は左Kan拡張) \\
$\scrg \colon \Top(Y)\op \to \Ab$を前層とし$f \colon X \to Y$を写像とする。$f$が誘導する射を$f^* \colon \Top(Y)\op \to \Top(X)\op$とする。このとき、自然な射$\eta \colon \scrg \to f\sh\scrg \cdot f^*$が存在して、
$(f\sh\scrg, \eta)$は$\scrg$の$f^*$に沿う左Kan拡張である。
\[
\xymatrix{
\Top(Y)\op \ar[dr]_-{f^*} \ar[rr]^-{\scrg} & {} \ar@{}[d]|{\Downarrow \eta}  & \Ab \\
{} & \Top(X)\op \ar[ru]_-{f\sh \scrg} & {}
}
\]
  }
  \begin{proof}
    $V \in \Top(Y)\op$に対して
    \[
    f\sh\scrg \cdot f^*(V) =  f\sh\scrg (f^{-1}(V)) = \rlim_{f(f^{-1}(V)) \subset V'} \scrg(V')
    \]
    であるから、自然な射$\eta \colon \scrg \to f\sh\scrg \cdot f^*$がある。これが左Kan拡張を定めることを示そう。次の(可換とは限らない)図式
\[
\xymatrix{
\Top(Y)\op \ar[dr]_-{f^*} \ar[rr]^-{\scrg} & {} \ar@{}[d]|{\Downarrow \grg}  & \Ab \\
{} & \Top(X)\op \ar[ru]_-{\scrf} & {}
}
\]
に示されるような前層$\scrf$と自然変換$\grg$が与えられたとする。次の図式
\[
\xymatrix{
\scrg \ar[rd]_-{\grg} \ar[r]^-{\eta} & f\sh \scrg \cdot f^* \ar@{.>}[d]^-{\grd \cdot f^*} & f\sh \scrg \ar@{.>}[d]^-{\grd} \\
{} & \scrf \cdot f^* & \scrf
}
\]
が可換になるような$\grd$の存在と一意性を言わなくてはいけない。

$U \in \Top(X),V \in \Top(Y)  \st f(U) \subset V$とする。このときすべての$V$について
\[
\xymatrix{
\scrg(V) \ar[d]_{\grg_V} \ar[r] & f\sh \scrg (U) \ar@{.>}[d]^-{\grd_U} \\
\scrf(f^{-1}(V)) \ar[r] & \scrf(U)
}
\]
を可換にするような$\grd_U$がある。この$\grd_U$の族が自然性を満たすことの証明は省略する。このとき定義から$\grg_V = \grd_{f^{-1}(V)} \circ \eta_V$が成り立つことがわかる。よって$\grd$の存在がいえた。

一意性を示そう。$U \in \Top(X),V \in \Top(Y)  \st f(U) \subset V$とする。このときすべての$V$について、次の図式
\[
\xymatrix{
{} & \scrg(V) \ar[dl]_-{\eta_V} \ar[dr] \ar[ddl]^(.3){\grg_V} & {} \\
f\sh (f^{-1}(V) ) \ar[rr] \ar[d]_-{\grd_{ f^{-1}(V) }} & {} & f\sh \scrg(U) \ar[d]^-{\grd_U} \\
\scrf( f^{-1}(V) )  \ar[rr] & {} & \scrf(U)
}
\]
は可換である。したがって$\scrg(V) \to f\sh \scrg(U)$と$\grd_U$の合成は一意に定まる。ゆえに普遍性から、$\grd$は一意である。これで、$(f\sh \scrg, \eta)$が$\scrg$の$f^*$に沿う左Kan拡張であることが証明できた。
  \end{proof}

\begin{rem}
Kan拡張が自然変換を誘導することについて注意しておく。
関手$K \colon \bfc \to \bfd$と圏$\bfe$を固定する。関手$F,T \colon \bfc \to \bfe$と自然変換$\gra \colon F \To T$が与えられていたとする。このときKan拡張の普遍性により
\begin{description}
  \item[(1)] $F$と$T$の$K$に沿う左Kan拡張$(\Lan_K F, \eta)$と$(\Lan_K T, \ve)$があったとする。このときある自然変換$\pi \colon \Lan_K F \To \Lan_K T$が存在して、
$\pi_K \circ \eta = \ve \circ \gra$を満たす。
  \item[(2)] $F$と$T$の$K$に沿う右Kan拡張$(\Ran_K F, \eta)$と$(\Ran_K T, \ve)$があったとする。このときある自然変換$\pi \colon \Ran_K F \To \Ran_K T$が存在して、
$\ve \circ \pi_K  =  \gra \circ \eta$を満たす。
\end{description}
ということがわかる。
\end{rem}




\prop{
(表現としてのKan拡張) \\
関手$K \colon \bfc \to \bfd$と圏$\bfe$を固定する。関手$F \colon \bfc \to \bfe$が与えられたとする。このとき次が成り立つ。
\begin{description}
  \item[(1)] $F$の$K$に沿う左Kan拡張$(\Lan_K F,\eta)$が存在すると仮定しよう。
  このとき$\eta$が定める自然変換
  \[
  \eta \colon \bfe^{\bfd}(\Lan_K F, -) \To \bfe^{\bfc}(F, - \cdot K)
  \]
  は同型である。
  \item[(2)] $F$の$K$に沿う右Kan拡張$(\Ran_K F,\ve)$が存在すると仮定しよう。
  このとき$\ve$が定める自然変換
  \[
  \ve \colon \bfe^{\bfd}(-, \Ran_K F) \To \bfe^{\bfc}(- \cdot K , F)
  \]
  は同型である。
\end{description}

}
\begin{proof}
  Kan拡張の普遍性からただちに従う。
\end{proof}


\prop{
(左Kan拡張は合成の左随伴) \\
関手$K \colon \bfc \to \bfd$と圏$\bfe$を固定する。関手$F,T \colon \bfc \to \bfe$と自然変換$\gra \colon F \To T$が与えられていたとする。さらに$F,T$の$K$に沿う左Kan拡張$(\Lan_K, \eta)$, $(\Lan_K T,\ve)$が存在したとし、これにより誘導される自然変換$\Lan_K F \To \Lan_K T$を$\pi$とする。このとき次の図式
\[
\xymatrix{
\bfe^{\bfd}(\Lan_K F, G)  \ar[r]^-{\eta} & \bfe^{\bfc}(F, G \cdot K) \\
\bfe^{\bfd}(\Lan_K T, G) \ar[u]^-{\pi^*} \ar[r]^-{\ve} &  \bfe^{\bfc}(T, G \cdot K) \ar[u]_-{\gra^*}
}
\]
は可換である。
}
\begin{proof}
  $\pi$の定義からあきらか。
\end{proof}

\begin{rem}
  右Kan拡張についても同様のことがいえて、右Kan拡張は合成の右随伴となる。
\end{rem}


\newpage

\bfsection{2.2 スキーム}


\bfsubsection{定義-スペクトラム 直前}
\barquo{
定義の局所性から$\calo$が層であることも明らかである。
}
\begin{proof}
  $\calo$の定義と層化の定義を見比べてみると (茎を与える写像を代入だと思えば) 両者は完全に一致するということが見て取れる。そこで前層$\scrf$を
  \[
  \scrf(U) = \setmid{s \colon U \to \coprod_{\frakp \in U} A_{\frakp}  }{s(\frakp) \in A_{\frakp} \text{かつ} \exists a \in A, \; \exists f \in A \sm \bigcup_{\frakp \in U} \frakp \quad s(\frakp) = a/f }
  \]
  として定める。これは貼り合わせが存在しないため一般には層にならないと予想される。この$\scrf$の点$\frakp \in \Spec A$における茎が$A_{\frakp}$であることが、命題2.2とまったく同様に示せる。したがって$\calo$は$\scrf$の層化であり、層であることがわかる。
\end{proof}


\bfsubsection{命題 2.2}
\barquo{
既に示した$\psi$の単射性を$D(h_ih_j)$に適用することによって、$A_{h_ih_j}$で$a_i / h_i = a_j / h_j$であることを得る必要がある。
}
\begin{rem}
  原著では『Hence, according to the injectivity of $\psi$ proved above, applied to $D(h_ih_j)$ we must have $a_i / h_i = a_j / h_j$ in $A_{h_i h_j}$.』と書いてある部分である。\textblue{誤訳。}
\end{rem}



\bfsubsection{定義-環付き空間}
\barquo{
環付き空間$(X,\calo_X)$から$(Y,\calo_Y)$への射とは連続写像$f \colon X \to Y$と$Y$上の環の層の写像$f^{\#} \colon \calo_Y \to f_* \calo_X$の対$(f,f^{\#})$である。
}
\begin{rem}
  $f^{\#}$と書いてはいるが、$f^{\#}$は$f$だけによって決まるものではないことに注意する。環付き空間は圏をなすのだが、射の合成の定義が書かれていないので説明する。$(f,f^{\#}) \colon (X,\calo_X) \to (Y,\calo_Y)$と$(g,g^{\#}) \colon (Y,\calo_Y) \to (Z,\calo_Z)$の合成$(h,h^{\#})$は次のように定める。まず$h = g \circ f$とする。そして$h^{\#}$は、任意の$W \opsub Z$に対して
  \[
  h^{\#}_W = f^{\#}_{g^{-1}(W)} \circ g^{\#}_W
  \]
  により定める。
\end{rem}





\bfsubsection{命題 2.3 (b)}
\barquo{
$\calo$の定義から写像$f$と$\vp_{\frakp}$を合成することによって、任意の開集合$V \subset \Spec A$に対して環の準同型$f^{\#} \colon \calo_{\Spec A}(V) \to \calo_{\Spec B}(f^{-1}(V))$を得、
}
\begin{proof}
  すこし詳しく説明する。$\calo_{\Spec A}$などといちいち書くのは面倒なので、$\calo_{A}$などと略記する。$s \in \calo_A(V)$が与えられたとする。$f^{\#}_V (s) \in \calo_B(f^{-1}(V))$を記述しよう。任意にとった$\frakp \in f^{-1}(V)$を$f^{\#}_V (s)$がどこへ写すかをみればいい。

  ひとことでいえば、それはこの可換図式による。
  \[
  \xymatrix{
  f^{-1}(V) \ar[r]^-{f^{\#}_V (s)} \ar[d]_-{f'} & \coprod_{\frakp \in f^{-1}(V)} B_{\frakp} \\
  \coprod_{\frakp \in f^{-1}(V)} \{f(\frakp)\} \ar[r]^-s & \coprod_{\frakp \in f^{-1}(V)} A_{f(\frakp)} \ar[u]_-{\vp'}
  }
  \]
  ただし、それぞれの写像は次のように定義される。
  \begin{align*}
    f'(\frakp) &= (\frakp, f(\frakp)) \\
    s(\frakp, f(\frakp)) &= (\frakp, s(f(\frakp))) \\
    \vp'(\frakp, t) &= (\frakp, \vp_{\frakp}(t))
  \end{align*}
以上の議論によって、$s \in \prod_{\frakq \in V} A_{\frakq}$を$f^{\#}_V(s) \in \prod_{\frakp \in f^{-1}(V)} B_{\frakp}$に対応させることができた。$f^{\#}_V(s) \in \calo_B(f^{-1}(V))$となっていることを確認したい。それは、
$t \in A \sm \bigcup_{\frakq \in V} \frakq$について次の図式
\[
\xymatrix{
A_t \ar[r] \ar[d]_-{\vp} & \prod_{\frakq \in V} A_{\frakq} \ar[d]^-{f^{\#}_V} \\
B_{\vp(t)} \ar[r] & \prod_{\frakp \in f^{-1}(V)} B_{\frakp}
}
\]
が可換であることによる。
\end{proof}




\bfsubsection{命題 2.3 (b)}
\barquo{
$f^{\#}$から茎に誘導される写像は局所準同形$\vp_{\frakp}$に他ならず、$(f,f^{\#})$は局所環付き空間の射となる。
}
\begin{proof}
  $\frakp \in \Spec B$と$f(\frakp) \in \Spec A$の近傍$D(g) \; (g \in A)$について、次の図式
  \[
  \xymatrix{
  {} & \calo_A(D(g)) \ar[ld] \ar[rr]^{f^{\#}_{D(g)} } & {} & f_*\calo_B(D(g)) \ar[ld] \ar@{=}[r] & \calo_B(D(\vp(g))) \ar[ld] \\
\calo_{A,f(\frakp)} \ar[rr] \ar[dd] & {} & (f_*\calo_B)_{f(\frakp)} \ar[r] & \calo_{B,\frakp} \ar[dd] & {} \\
{} & A_g \ar[uu]_(.3){\psi_g} \ar[rrr]^-{\vp_g} \ar[ld] & { } & {} &  B_{\vp(g)} \ar[uu]_{\psi_{\vp(g)}} \ar[ld] \\
A_{f(\frakp) } \ar[rrr]^-{\vp_{\frakp}} & {} & {}  & B_{\frakp}
   }
  \]
  の前面以外は可換である。ただし$\psi$は命題2.2(b)で構成された同型であるとする。よって、$\{ D(g)\}$が基本近傍系であり$\psi_g$が同型であることから、前面も可換であることがいえる。したがって、$\vp_{\frakp}$と$f^{\#}$が茎に誘導する写像とは自然に同一視される。

\end{proof}


\bfsubsection{命題 2.3 (c)}
\barquo{
$f^{\#}$もまた$\vp$から誘導されることは直ちに分かるので、局所環付き空間の射$(f,f^{\#})$は確かに環の準同形$\vp$から来ていることになる。
}
\begin{proof}
  $\vp$から誘導される層の射$\calo_A \to f_*\calo_B$を$\vp^{\#}$と表すことにする。このとき、命題2.3(b)での直方体図式に$g=1$を代入したものを考えると、$f^{\#}_{\frakp} = \vp^{\#}_{\frakp}$がわかる。$(f_*\calo_B)_{f(\frakp)} \to \calo_{B,\frakp}$は単射なので、
  $(f^{\#})_{f(\frakp)} = (\vp^{\#})_{f(\frakp)} $である。よって、$\calo_A$, $f_*\calo_B$は層だったので、$f^{\#} = \vp^{\#}$がわかったことになる。(これは、$\vp$が整とは限らないのでウソ。直さないといけない。具体的には、順像と逆像の随伴を使う。)
\end{proof}


\newpage

\bfsection{2.4 分離射と固有射}

\bfsubsection{冒頭}
\barquo{
というのは、Zariskiトポロジーは決してHausdorffにはならず、またスキームの下部位相空間はスキームのすべての性質を正確には反映していないからである。
}
\begin{rem}
  \textblue{これは嘘。} G\"ortz Wedhorn\cite{GW} section 2 Exercise 2.23 で言われているように、環$A$がBooleanならば$\Spec A$は compact (したがってとくにHausdorff) になることが知られている。  
\end{rem}


\newpage

\bfsection{2.5 加群の層}

\bfsubsection{定義-$\calo_X$加群の逆像}
\barquo{
正確には、任意の$\calo_X$加群$\calf$および任意の$\calo_Y$加群$\scrg$に対して自然な群の同型
\[
\Hom_{\calo_X}(f^*\scrg, \scrf ) \cong \Hom_{\calo_Y}(\scrg, f_*\scrf )
\]
が存在する。
}
\begin{proof}
  Bosch\cite{Bosch} section 6.9 Proposition 3 を参照のこと。
\end{proof}


\bfsubsection{命題 5.1}
\barquo{
(e) 任意の$A$加群$M$に対して$f^*(\wt{M}) \cong (M \ts_A B)\wt{}$
}
\begin{proof}
  詳しい証明はBosch\cite{Bosch} Section6.9 Cor.4を参照のこと。
\end{proof}


\bfsubsection{定義-連接層}
\barquo{
$(X,\calo_X)$をスキームとする。$\calo_X$加群$\calf$が準連接であるとは、$X$が開アファイン部分集合$U_i = \Spec A_i$で覆うことができ、各$i$に対して$A_i$加群$M$が存在し、$\calf|_{U_i} \cong \wt{M_i}$となるときをいう。さらに$M_i$が有限生成$A_i$加群ととれるとき$\calf$を連接という。
}
\begin{rem}
  連接層の定義に疑問。$X$が局所Noehterでないときにこれは正しいのか?
\end{rem}


\bfsubsection{命題 5.8}
\barquo{
(a) $\calf$を$\calo_X$加群の準連接層とすると、$f^*\scrg$は$\calo_X$加群の準連接層である。
}
\begin{proof}
  なぜアファインの場合に帰着できるのかについてはLiu\cite{Liu} 命題1.14を参照のこと。
\end{proof}


\newpage

\bfsection{2.7 射影的射}

\bfsubsection{定理7.1 直前}
\barquo{
さて、$X$を$A$上の任意のスキームとし、$\vp \colon X \to \P^n_A$を$X$から$\P^n_A$への$A$射とする。このとき$\scrl = \vp^*(\calo(1))$は$X$上の可逆層で、大域切断$s_0, \cdots , s_n$ (ここで$s_i = \vp^*(x_i)$, $s_i \in \grG(X,\scrl)$) は層$\scrl$を生成する。
}
\begin{rem} ${}$
  \begin{description}
    \item[$\vp^*(x_i)$とは何だろうか?]  Daniel Murfet氏のPDF\cite{RisingSea}に従って補足する。$f \colon X \to Y$がスキーム間の射で$\scrf$が$\calo_Y$-加群であるとする。このとき随伴であることからcanonicalな射$\eta \colon \scrf \to f_* f^* \scrf$がある。これにより、$V \opsub Y$に対して環の射$\eta_V \colon \grG(V, \scrf) \to \grG(f^{-1}(V), f^*\scrf)$が誘導される。
    そこで$s \in  \grG(V, \scrf)$に対して$f^*(s) = \eta_V(s)$と定めるのである。
    \item[大域切断$s_0, \cdots , s_n$が層$\scrl$を生成するのは何故?]
    茎を考えるとわかる。
  \end{description}

\end{rem}


\newpage

\begin{thebibliography}{1}%参考文献の リスト
  \bibitem{松坂} 松坂和夫『集合・位相入門』(岩波書店, 1968)
  \bibitem{齋藤} 齋藤正彦『線型代数学』(東京図書, 2014)
  \bibitem{内田} 内田伏一『集合と位相』(裳華房, 1986)
  \bibitem{松村} 松村英之『可換環論』(共立出版, 1980)
  \bibitem{藤崎} 藤崎源二郎『体とガロア理論』(岩波書店, 1991)
  \bibitem{雪江3} 雪江明彦『代数学3 代数学のひろがり』(日本評論社, 2011)
  \bibitem{Harris} Joe Harris『Algebraic Geometry』(Springer, 1992)
  \bibitem{GW} Ulrich G\"{o}rtz, Torsten Wedhorn『Algebraic Geometry I : Schemes  with Examples and Exercises』(Springer, 2010)
  \bibitem{Liu} Qing Liu『Algebraic Geometry and Arithmetic Curves』(Oxford University Press, 2002)
  \bibitem{Wedhorn} Torsten Wedhorn『Manifolds, Sheaves, and Cohomology』(Springer, 2016)
  \bibitem{Bosch} Siegfried Bosch『Algebraic Geometry and Commutative Algebra』(Springer, 2013)
  \bibitem{Rotman} Joseph J.Rotman『An Introduction to Homological Algebra』(Springer, 2009)
  \bibitem{MacLane} Saunders Mac Lane『Categories for the Working Mathematician』(Springer, 1971)
  \bibitem{Riehl} Emily Riehl『Category Theory in Context』(Dover, 2016)
  \bibitem{Chris} Chris Eur (\url{https://math.berkeley.edu/~ceur/notes.html})
  \bibitem{Math256} Math 256A Algebraic Geometry Fall 2012 (\url{https://math.berkeley.edu/~reb/courses/256A/})
  \bibitem{Math683} Mathematics 683 Algebraic Geometry Fall 2013

  (\url{http://sierra.nmsu.edu/morandi/oldwebpages/Math683Fall2013/})
\end{thebibliography}




\end{document}
