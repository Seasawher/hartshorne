\bfsection{2.2 スキーム}



\bfsubsection{補題 2.1 直前}
\barquo{
$\fraka$を$A$の任意のイデアルとして、部分集合$V(\fraka) \subset \Spec A$を$\fraka$を含むすべての素イデアルの集合と定める。
}
\begin{rem}
  G\"{o}rtz Wedhorn\cite{GW}にならって補足。$\Spec A$の部分集合を$A$のイデアルに対応させるような写像$I$を
  \[
  I(Y) = \bigcap_{\frakp \in Y} \frakp
  \]
  により定める。このとき次が成り立つ。
\end{rem}

\prop{
$Y \subset \Spec A$は部分集合、$\fraka \subset A$はイデアルとする。このとき次が成り立つ。
\begin{description}
  \item[(1)] $\sqrt{I(Y)} = I(Y)$
  \item[(2)] $I(V(\fraka)) = \sqrt{\fraka}$
  \item[(3)] $V(I(Y)) = \ol{Y}$
\end{description}
}
\begin{proof}
  G\"{o}rtz Wedhorn\cite{GW} Proposition 2.3を参照のこと。
\end{proof}


\bfsubsection{定義-スペクトラム 直前}
\barquo{
定義の局所性から$\calo$が層であることも明らかである。
}
\begin{proof}
  $\calo$の定義と層化の定義を見比べてみると (茎を与える写像を代入だと思えば) 両者は完全に一致するということが見て取れる。そこで前層$\scrf$を
  \[
  \scrf(U) = \setmid{s \colon U \to \coprod_{\frakp \in U} A_{\frakp}  }{s(\frakp) \in A_{\frakp} \text{かつ} \exists a \in A, \; \exists f \in A \sm \bigcup_{\frakp \in U} \frakp \quad s(\frakp) = a/f }
  \]
  として定める。これは貼り合わせが存在しないため一般には層にならないと予想される。この$\scrf$の点$\frakp \in \Spec A$における茎が$A_{\frakp}$であることが、命題2.2とまったく同様に示せる。したがって$\calo$は$\scrf$の層化であり、層であることがわかる。
\end{proof}


\bfsubsection{命題 2.2}
\barquo{
既に示した$\psi$の単射性を$D(h_ih_j)$に適用することによって、$A_{h_ih_j}$で$a_i / h_i = a_j / h_j$であることを得る必要がある。
}
\begin{rem}
  原著では『Hence, according to the injectivity of $\psi$ proved above, applied to $D(h_ih_j)$ we must have $a_i / h_i = a_j / h_j$ in $A_{h_i h_j}$.』と書いてある部分である。\textblue{誤訳。}
\end{rem}



\bfsubsection{定義-環付き空間}
\barquo{
環付き空間$(X,\calo_X)$から$(Y,\calo_Y)$への射とは連続写像$f \colon X \to Y$と$Y$上の環の層の写像$f^{\#} \colon \calo_Y \to f_* \calo_X$の対$(f,f^{\#})$である。
}
\begin{rem}
  $f^{\#}$と書いてはいるが、$f^{\#}$は$f$だけによって決まるものではないことに注意する。環付き空間は圏をなすのだが、射の合成の定義が書かれていないので説明する。$(f,f^{\#}) \colon (X,\calo_X) \to (Y,\calo_Y)$と$(g,g^{\#}) \colon (Y,\calo_Y) \to (Z,\calo_Z)$の合成$(h,h^{\#})$は次のように定める。まず$h = g \circ f$とする。そして$h^{\#}$は、任意の$W \opsub Z$に対して
  \[
  h^{\#}_W = f^{\#}_{g^{-1}(W)} \circ g^{\#}_W
  \]
  により定める。
\end{rem}

\begin{rem}
  G\"{o}rtz Wedhorn\cite{GW}に従って補足をする。構造層というのは、開集合$U$に対して$U$上で定義された「良い性質を満たす」写像の集まりを返すようなものだというイメージがあるらしい。そして層の射$f^{\#}_V \colon \calo_Y(V) \to \calo_X(f^{-1}(V)) $は$f$を口の方につなぐような写像だというイメージだそうだ。このとき$f$を合成することで「良い性質を満たす」ことは変わってはいけないこともこのイメージには含まれている。自然性が成り立ち、$f^{\#}$が層の射でなくてはならないことはこのイメージから説明できそうだ。

  同書は「Since viewing sections of the structure sheaves as functions is only a heuristic, we cannot acturally compose sections with the map $f$.」と書いているが、環のスペクトラムの場合は本当に$f$をつなぐ写像だと思うことができる。環準同型$\vp \colon A \to B$が与えられており、$f \colon \Spec B \to \Spec A$と$f^{\#} \colon \calo_A \to f_* \calo_B$は$\vp$から誘導される射とする。このとき$g \in A$に対して$f^{\#}_{D(g)}$はどのような写像かという事を考える。すると次の図式
  \[
  \xymatrix{
  A_g \ar[d]_{f\fl} & \coprod_{\frakq \in D(\vp(g))} D(g) \ar[r]  &  \coprod_{\frakq \in D(\vp(g))} A_{f(\frakq)} \ar[d]^-{\vp} \\
B_{\vp(g)} & D(\vp(g)) \ar[r] \ar[u]^-f & \coprod_{\frakq \in D(\vp(g))} B_{\frakq}
  }
  \]
  により定められるような、「口に$f$をつないで$\vp$を尻に継ぎ足す」写像$f\fl$は$f^{\#}_{D(g)} $と一致することがわかる。
\end{rem}





\bfsubsection{命題 2.3 (b)}
\barquo{
$\calo$の定義から写像$f$と$\vp_{\frakp}$を合成することによって、任意の開集合$V \subset \Spec A$に対して環の準同型$f^{\#} \colon \calo_{\Spec A}(V) \to \calo_{\Spec B}(f^{-1}(V))$を得、
}
\begin{proof}
  すこし詳しく説明する。$\calo_{\Spec A}$などといちいち書くのは面倒なので、$\calo_{A}$などと略記する。$s \in \calo_A(V)$が与えられたとする。$f^{\#}_V (s) \in \calo_B(f^{-1}(V))$を記述しよう。任意にとった$\frakp \in f^{-1}(V)$を$f^{\#}_V (s)$がどこへ写すかをみればいい。

  ひとことでいえば、それはこの可換図式による。
  \[
  \xymatrix{
  f^{-1}(V) \ar[r]^-{f^{\#}_V (s)} \ar[d]_-{f'} & \coprod_{\frakp \in f^{-1}(V)} B_{\frakp} \\
  \coprod_{\frakp \in f^{-1}(V)} \{f(\frakp)\} \ar[r]^-s & \coprod_{\frakp \in f^{-1}(V)} A_{f(\frakp)} \ar[u]_-{\vp'}
  }
  \]
  ただし、それぞれの写像は次のように定義される。
  \begin{align*}
    f'(\frakp) &= (\frakp, f(\frakp)) \\
    s(\frakp, f(\frakp)) &= (\frakp, s(f(\frakp))) \\
    \vp'(\frakp, t) &= (\frakp, \vp_{\frakp}(t))
  \end{align*}
以上の議論によって、$s \in \prod_{\frakq \in V} A_{\frakq}$を$f^{\#}_V(s) \in \prod_{\frakp \in f^{-1}(V)} B_{\frakp}$に対応させることができた。$f^{\#}_V(s) \in \calo_B(f^{-1}(V))$となっていることを確認したい。それは、
$t \in A \sm \bigcup_{\frakq \in V} \frakq$について次の図式
\[
\xymatrix{
A_t \ar[r] \ar[d]_-{\vp} & \prod_{\frakq \in V} A_{\frakq} \ar[d]^-{f^{\#}_V} \\
B_{\vp(t)} \ar[r] & \prod_{\frakp \in f^{-1}(V)} B_{\frakp}
}
\]
が可換であることによる。
\end{proof}




\bfsubsection{命題 2.3 (b)}
\barquo{
$f^{\#}$から茎に誘導される写像は局所準同形$\vp_{\frakp}$に他ならず、$(f,f^{\#})$は局所環付き空間の射となる。
}
\begin{proof}
  $\frakp \in \Spec B$と$f(\frakp) \in \Spec A$の近傍$D(g) \; (g \in A)$について、次の図式
  \[
  \xymatrix{
  {} & \calo_A(D(g)) \ar[ld] \ar[rr]^{f^{\#}_{D(g)} } & {} & f_*\calo_B(D(g)) \ar[ld] \ar@{=}[r] & \calo_B(D(\vp(g))) \ar[ld] \\
\calo_{A,f(\frakp)} \ar[rr] \ar[dd] & {} & (f_*\calo_B)_{f(\frakp)} \ar[r] & \calo_{B,\frakp} \ar[dd] & {} \\
{} & A_g \ar[uu]_(.3){\psi_g} \ar[rrr]^-{\vp_g} \ar[ld] & { } & {} &  B_{\vp(g)} \ar[uu]_{\psi_{\vp(g)}} \ar[ld] \\
A_{f(\frakp) } \ar[rrr]^-{\vp_{\frakp}} & {} & {}  & B_{\frakp}
   }
  \]
  の前面以外は可換である。ただし$\psi$は命題2.2(b)で構成された同型であるとする。よって、$\{ D(g)\}$が基本近傍系であり$\psi_g$が同型であることから、前面も可換であることがいえる。したがって、$\vp_{\frakp}$と$f^{\#}$が茎に誘導する写像とは自然に同一視される。

\end{proof}


\bfsubsection{命題 2.3 (c)}
\barquo{
$f^{\#}$もまた$\vp$から誘導されることは直ちに分かるので、局所環付き空間の射$(f,f^{\#})$は確かに環の準同形$\vp$から来ていることになる。
}
\begin{proof}
  $\vp$から誘導される層の射$\calo_A \to f_*\calo_B$を$\vp^{\#}$と表すことにする。$f^{\#}$の自然性により、自然な同型を省略して、次の図式
  \[
  \xymatrix{
  A_g \ar[r]^-{ f^{\#}_{D(g)} } & B_{\vp(g)} \\
  A \ar[u] \ar[r]^-{\vp} & B \ar[u]
  }
  \]
  は可換である。したがって局所化の普遍性から$f^{\#}_{D(g)} = \vp_g$である。この性質は$\vp^{\#}$を特徴付けるものなので、$f^{\#} = \vp^{\#}$がわかる。
\end{proof}


\begin{que}
  局所環付き空間の射
  \[
  (f,f^{\#}) \colon (X,\calo_X) \to (Y,\calo_Y)
  \]
  が与えられているとする。$P \in X$だとして、次の2通りの構成を考えよう。
  \begin{description}
    \item[(1)] $f^{\#} \colon \calo_Y \to f_* \calo_X$は、逆像と合成の随伴により$g \colon f^{-1}\calo_Y \to \calo_X$と対応している。この$g$に対して$P$における茎を考えると$g_P \colon \calo_{Y,f(P)} \to \calo_{X,P}$が誘導される。
    \item[(2)] $f^{\#}$の$f(P)$における茎を考えると$(f^{\#})_{f(P)} \colon \calo_{Y,f(P)} \to (f_* \calo_X)_{f(P)}$が誘導される。この写像と$(f_* \calo_X)_{f(P)} \to \calo_{X,P}$を合成して$f^{\#}_P \colon \calo_{Y,f(P)} \to \calo_{X,P}$を得る。
  \end{description}
  このとき、$g_P$と$f^{\#}_P$は一致するか?
\end{que}
\begin{sol}
  実際に元をとってみればすぐにわかる。$t \in \calo_{Y,f(P)}$とし、$t$の代表元$s \in \coprod_{f(P) \in V} \calo_Y(V)$をとる。このとき
  \[
  f^{\#}_P (t) = g_P(t) = (f^{\#}_V(s))_P
  \]
  となるから一致する。では、元を取らないような別証があるだろうか?
\end{sol}






\bfsubsection{例 2.3.1}
\barquo{
一方$t_0$は閉点で、
}
\begin{rem}
  $X = \Spec A$とする。次の事実が成り立つことに注意する。証明は省略する。
  \begin{description}
    \item[(1)] $x \in X$が閉点、つまり$\{x \} \clsub X$であることは$x \subset A$が極大イデアルであることと同値
    \item[(2)] $x \in X$が生成点であること、つまり$\ol{\{x\}}=X$であることは$x \subset A$が唯一の極小素イデアルであることと同値
  \end{description}
\end{rem}
