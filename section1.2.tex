\bfsection{1.2 射影多様体}
\bfsubsection{命題2.1 直前}
\barquo{
$T$を$S$の斉次元からなる任意の集合として, $T$の零点集合を
\[
Z(T) = \setmid{P \in \P^n}{\text{すべての$f \in T$について$f(P)=0$}}
\]
と定義する. $\fraka$が$S$の斉次イデアルのとき, $T$を$\fraka$の中の斉次元すべての集合として$Z(\fraka)=Z(T)$と定める.
}
\begin{rem}
  $H(S)$を$S$の斉次イデアル全体とし、$S^h$を$S$の斉次元全体とする。冪集合を以下$P$で表すことにし、上付きスターで逆像を、下付きスターで像を表すものとする。

  斉次イデアルに対してその斉次元全体を対応させる写像$H(S) \to P(S^h)$を$s$とする。また斉次元の集合からそれが生成する斉次イデアルを対応させる写像$P(S^h) \to H(S)$を$g$で表す。

  零点集合を対応させる写像を、定義域の違いにより$\widetilde{Z} \colon P(S^h) \to P(\P^n)$と$Z \colon H(S) \to P(\P^n)$との二つの記号により区別する。このとき定義から、次の図式が可換。
  \[
  \xymatrix{
  H(S) \ar[r]^Z \ar[d]_s & P(\P^n) \\
  P(S^h) \ar[ur]_{\widetilde{Z}}
  }
  \]
\end{rem}

\lem{
  次の図式は可換である。
  \[
  \xymatrix{
  H(S) \ar[r]^s \ar[dr]_1 & P(S^h) \ar[d]^g \\
  & H(S)
  }
  \]
}

\begin{proof}
  斉次イデアル$\fraka$をとる。斉次イデアルは斉次元によって生成することができるので、$\fraka \cap S^h$は生成元を含む。よって$gs(\fraka) \supset \fraka$である。逆はあきらかだから$gs = 1$がいえた。
\end{proof}
\lem{
($Z'$と$\widetilde{Z}$の基本関係式) \\
  $\pi \colon \A^{n+1} \setminus \single \to \P^n$を自然な商写像とし、$Z'$をアファイン空間での零点集合を対応させる写像、$i$、$j$を包含写像とすると次は可換。
  \[
  \xymatrix{
  P(S^h) \ar[rr]^{\widetilde{Z}} \ar[d]_{j_*} & & P(\P^n) \ar[d]^{\pi^*} \\
  P(S) \ar[r]^{Z'} & P(\A^{n+1}) \ar[r]^{i^*} & P(\A^{n+1} \setminus \single)
  }
  \]
  すなわち$T \subset S^h$に対して
  \[
  Z'(T) \setminus \single = \pi^{-1} (\widetilde{Z}(T))
  \]
  が成り立つ。
}
\begin{proof}
  ぐっとにらめばわかる。
\end{proof}
\prop{
($Z$と${\widetilde Z}$の同一性) \\
次は可換。
\[
\xymatrix{
{} & H(S) \ar[r]^Z & P(\P^n) \\
H(S) \ar[ur]^1 \ar[r]^s & P(S^h) \ar[u]_g \ar[ur]_{\widetilde Z}&
}
\]
}

\begin{proof}
  未知なのは$Zg = {\widetilde Z}$の部分である。$T \subset S^h$のとき
  \begin{align*}
    \pi^{-1}({\widetilde Z}(T)) &=  Z'(T) \setminus \single\\
    &=  Z'(gT) \setminus \single &(\text{$Z'$はイデアルと生成元を区別しない}) \\
    &\subset  Z'(sg(T)) \setminus \single \\
    &= \pi^{-1} ({\widetilde Z}(sgT))
  \end{align*}
  が成り立つ。$\pi$は全射なので
  \[
  {\widetilde Z}(T) \subset {\widetilde Z}(sg (T))
  \]
  が結論される。一方で$T \subset sg (T)$により${\widetilde Z}(sg (T)) \subset {\widetilde Z}(T)$はあきらかなので
  \[
  {\widetilde Z}(T) = {\widetilde Z}(sg(T)) = Z(g(T))
  \]
  がわかる。
\end{proof}
\prop{
($Z'$と$Z$の基本関係式) \\
  次は可換。
  \[
  \xymatrix{
  H(S) \ar[rr]^{Z} \ar[d] & & P(\P^n) \ar[d]^{\pi^*} \\
  P(S) \ar[r]^{Z'} & P(\A^{n+1}) \ar[r]^{i^*} & P(\A^{n+1} \setminus \single)
  }
  \]
  すなわち、斉次イデアル$\fraka \subset S$に対して
  \[
  \pi^{-1}(Z(\fraka)) = Z'(\fraka) \setminus \single
  \]
  が成り立つ。
}

\begin{proof}
  $g(T)= \fraka$なる$T$をとると
  \begin{align*}
\pi^{-1}(Z(\fraka)) &= \pi^{-1}(Zg(T)) \\
&= \pi^{-1}({\widetilde Z}(T)) \\
&= Z'(T) \setminus \single \\
&= Z'(\fraka) \setminus \single
  \end{align*}
  であることがわかる。
\end{proof}


\bfsubsection{命題2.2 直前}
\barquo{
$Y$を$\P^n$の任意の部分集合として, $Y$の$S$における斉次イデアルを$\setmid{f \in S}{\text{$f$は斉次ですべての$P \in Y$について$f(P)=0$}}$と定め, $I(Y)$と書く.
}
\begin{rem}
  \textblue{誤訳かと思われる。}原著では

If $Y$ is any subset of $\P^n$, we define the homogeneous ideal of $Y$ in $S$,
denoted $I(Y)$, to be the ideal generated by $\{ f \in  S \mid f \; \text{is homogeneous and} \;
f(P) = 0 \; \text{for all} \; P \in Y \}$.

となっている。$I(Y)$は$\setmid{f \in S}{\text{$f$は斉次ですべての$P \in Y$について$f(P)=0$}}$により生成されるイデアルとするのが正しい。ここで、
\[
{\widetilde I}(Y) = \setmid{f \in S}{\text{$f$は斉次ですべての$P \in Y$について$f(P)=0$}}
\]
と表すことにする。定義により次は可換。
\[
\xymatrix{
P(\P^n) \ar[r]^I \ar[dr]_{\wt{I}}& H(S) \\
{} & P(S^h) \ar[u]_g
}
\]
\end{rem}
\lem{
($I'$と${\widetilde I}$の基本関係式) \\
  次は可換。
 \[
 \xymatrix{
 P(\P^n) \ar[d]_{\pi^*} \ar[rr]^{\wt{I}} & {} & P(S^h)  \\
P(\A^{n+1} \setminus \single) \ar[r]^{i_*} & P(\A^{n+1}) \ar[r]^{I'} & P(S) \ar[u]^{j^*}
 }
 \]
  すなわち$Y \subset \P^n$に対して次が成り立つ。
  \[
  I'(\pi^{-1}(Y))\cap S^h = {\widetilde I}(Y)
  \]
}

\begin{proof}
  ぐっとにらめばわかる。
\end{proof}
\lem{
  $Y \subset \P^n$について、$I'(\pi^{-1}(Y))$は斉次イデアルである。
}

\begin{proof}
  いま$f \in I'(\pi^{-1}(Y))$とし、$f= f_0 + \cdots + f_k$と斉次元$f_i \; (\deg f_i = i)$の和に分解されたとする。このとき$y=(y_o, \cdots , y_n) \in \pi^{-1}(Y)$ならば、
  \[
  \forall \grl \in k \setminus \single \quad f(\grl y_0, \cdots , \grl y_n) = 0
  \]
  である。したがって
  \[
  \forall \grl \in k \setminus \single \quad f_0 + \grl f_1(y) + \cdots + \grl^k f_k(y) = 0
  \]
  ということになる。$k$は無限体なので$f_0 = f_1(y) = \cdots = f_n(y) = 0$でなくてはならない。$y \in \pi^{-1}(Y)$は任意だったから$f_0, \cdots , f_k \in I'(\pi^{-1}(Y)) \cap S^h$であり、ゆえに示したいことがいえた。
\end{proof}
\prop{
($I'$と$I$の基本関係式) \\
  次は可換。
  \[
  \xymatrix{
  P(\P^n) \ar[d]_{\pi^*} \ar[rr]^I & {} & H(S) \ar[d] \\
P(\A^{n+1} \setminus \single) \ar[r]^{i_*} & P(\A^{n+1}) \ar[r]^{I'} & P(S)
  }
  \]
  すなわち$I(Y)=I'(\pi^{-1}(Y))$である。
}

\begin{proof}
  $I'\pi^*$の値域が$H(S)$に含まれることがわかっているので
  \begin{align*}
    I &= g {\widetilde I} \\
    &= gsI'\pi^* \\
    &= I'\pi^*
  \end{align*}
  という計算でわかる。
\end{proof}



\prop{
  ($I$と${\widetilde I}$の同一性) \\
  次は可換。
  \[
  \xymatrix{
    P(\P^n) \ar[r]^I \ar[dr]_{\widetilde I} & H(S) \ar[d]^s \ar[dr]^1  \\
    {} & P(S^h) \ar[r]^g & H(S)
    }
\]
}

\begin{proof}
  $I'$と$I$の基本関係式からあきらか。
\end{proof}

\bfsubsection{命題 2.2}
\barquo{
素直に調べれば$\vp(Y) = Z(T')$であることがわかる。
}
\begin{rem}
  次の補題をまず示す。
\end{rem}

\lem{
  (零点変換公式$\gra$) \\
  次は可換。
  \[
  \xymatrix{
  P(S^h) \ar[d]^{\gra_*} \ar[r]^{\widetilde Z} & P(\P^n) \ar[r]^{i^*} & P(U) \ar[d]^{\vp_*} \\
  P(A) \ar[rr]_{Z'} & {} & P(\A^n)
  }
  \]
  すなわち
  \[
  Z'(\gra (T)) = \vp ({\widetilde Z}(T) \cap U)
  \]
が成り立つ。
}

\begin{proof}
  $T \subset S^h$と$x \in \A^n$に対して
  \begin{align*}
    x \in Z'(\gra (T)) &\iff \forall f \in T \quad \gra f (x)=0 \\
    &\iff f(1,x) = 0 \\
    &\iff \pi(1,x) \in \wt{Z}(T) \cap U \\
    &\iff \vp^{-1}(x) \in \wt{Z}(T) \cap U \\
    &\iff x \in \vp^{-1}(\wt{Z}(T) \cap U)
  \end{align*}
  であるから示したいことがいえた。
\end{proof}
\begin{proof}
  零点変換公式から引用部を示そう。$Y \clsub U$より、
  \begin{align*}
    \vp(Y) &= \vp(\overline{Y} \cap U) \\
    &= \vp(\wt{Z}(T) \cap U) \\
    &= Z'(\gra (T)) \\
    &= Z'(T')
  \end{align*}
  だからいえた。
\end{proof}


\bfsubsection{命題 2.2}
\barquo{
$\vp^{-1}(W) = Z(\beta(T')) \cap U$であることは簡単に確かめられる.
}
\begin{rem}
  次の二つの補題を示せば十分である。
\end{rem}
\lem{
  $\gra \beta = 1$が成り立つ。
}

\begin{proof}
  どうみてもあきらか。
\end{proof}

\lem{
(零点変換公式$\beta$) \\
  次は可換。
  \[
  \xymatrix{
P(U) & P(\P^n) \ar[l]_{i^*} & P(S^h) \ar[l]_{\wt Z} \\
P(\A^n) \ar[u]^{\vp^*} & {} & P(A) \ar[ll]_{Z'} \ar[u]_{\beta_*}
  }
  \]
  すなわち$\vp^{-1}Z'(T') = Z(\beta(T')) \cap U$である。
}

\begin{proof}
単純に計算するだけで示せる。
  \begin{align*}
    \vp^* Z' &= \vp^* Z' \gra_* \beta_* \\
&= \vp^* (\vp_* i^* \wt{Z}) \beta_* \\
&= i^* {\wt Z} \beta_*
  \end{align*}
\end{proof}




\bfsubsection{演習問題 2.1}
斉次元$f \in S$について$\deg f > 0$は$f \in I'(0)$を意味することに注意する。斉次イデアル$\fraka$について
\begin{align*}
  I(Z(\fraka)) \cap I'(0) &= I'(\pi^{-1} Z(a)) \cap I'(0) \\
  &= I'(\pi^{-1} Z(a) \cup \single) \\
  &\subset I'(Z'(\fraka)) \\
  &= \sqrt{\fraka}
\end{align*}
が成り立つことから判る。



\bfsubsection{演習問題 2.2}
\begin{description}
  \item[(i) $\to$ (ii)] $Z(\fraka)=\emptyset$により$Z'(\fraka)\setminus \single = \pi^{-1}(Z(\fraka))=\emptyset$である。したがって$Z'(\fraka)$は$\single$または$\emptyset$なので、零点定理により$\sqrt{\fraka} = I'(Z'(\fraka)) = S_+ \text{ または } S$である。
  \item[(ii) $\to$ (iii)] 仮定より$S_1 \subset {\sqrt \fraka}$である。よってある$k \in \N$が存在してすべての$i$について$x_i^k \in \fraka$となる。このとき$d = (n+1)(k-1)+1$とすると$S_d \subset \fraka$が成立する。
  \item[(iii) $\to$ (i)] $Z(\fraka) \subset Z(S_d) = \emptyset$よりあきらか。
\end{description}



\bfsubsection{演習問題 2.3}
\begin{description}
  \item[(a)] あきらか
  \item[(b)] ${\wt I}(Y_1) \supset {\wt I}(Y_2)$より、$I(Y_1) \supset {\wt I}(Y_2)$である。$I(Y_1)$はイデアルなので$I(Y_1) \supset I(Y_2)$である。
  \item[(c)] 次のように計算すればわかる。
  \begin{align*}
    I(Y_1 \cup Y_2) &= I'(\pi^{-1}(Y_1 \cup Y_2)) \\
    &= I'(\pi^{-1}Y_1 \cup \pi^{-1}Y_2)) \\
    &= I'(\pi^{-1}Y_1) \cap I'(\pi^{-1}Y_2) \\
    &= I(Y_1) \cap I(Y_2)
  \end{align*}
\end{description}


\bfsubsection{演習問題 2.4}
\begin{description}
  \item[(a)]$Z(S_+) = Z(S)= \emptyset$なので$S_+$と$S$のどちらかを除く必要があるが、$I(\emptyset)= S$なので除かれるべきなのは$S_+$である。あとはアファイン空間のときと同様。
  \item[(b)]『斉次イデアル$\fraka$が素イデアルであることは、任意の斉次元$f$、$g$について$fg \in \fraka$ならば$f \in \fraka$または$g \in \fraka$であることと同値である』ことに注意すれば、例1.4と同様。
  \item[(c)]$I(\P^n)=0$よりあきらか。
\end{description}



\bfsubsection{演習問題 2.5}
\begin{description}
  \item[(a)]例1.4.7と同様。
  \item[(b)]命題1.5よりあきらか。
\end{description}






\bfsubsection{演習問題 2.6}
\begin{rem}
  まず次の二つの補題を用意しておく。この補題を使わなくても十分証明できるが、証明の見通しをよくするために用意した。
\end{rem}
\lem{
  (随伴性) \\
  次が成り立つ。
  \begin{description}
    \item[(1)] 写像$f \colon X \to Y$があるとき$A \subset X$、$B \subset Y$に対して
    \[
    f(A) \subset B \iff A \subset f^{-1}(B)
    \]
    \item[(2)] $Z \colon H(S) \to P(\P^n)$と$I \colon P(\P^n) \to H(S)$を考える。斉次イデアル$\fraka$と$Y \subset \P^n$が与えられたとき
    \[
Y \subset Z(\fraka) \iff \fraka \subset I(Y)
    \]
    同様のことが$Z'$、$I'$や${\wt Z}$、${\wt I}$についても成り立つ。
  \end{description}
}
\begin{proof}
  (2)の$Z$、$I$の場合だけが非自明である。$Y \subset Z(\fraka)$とすると
  \begin{align*}
    Y \subset Z(\fraka) &\Rightarrow Y \subset {\wt Z}s(a) \\
    &\Rightarrow s(\fraka) \subset {\wt I}(Y)  \\
    &\Rightarrow \fraka \subset g{\wt I}(Y) \\
    &\Rightarrow \fraka \subset I(Y)
  \end{align*}
  となる。逆に$\fraka \subset I(Y)$と仮定すると
  \begin{align*}
    \fraka \subset I(Y) &\Rightarrow s(\fraka) \subset sI(Y) \\
    &\Rightarrow s(a) \subset {\wt I}(Y) \\
    &\Rightarrow Y \subset {\wt Z}(s(\fraka)) \\
    &\Rightarrow Y \subset Z(\fraka)
   \end{align*}
\end{proof}

\lem{
  (イデアル変換公式$\beta$) \\
$U = U_0$、$\vp = \vp_0$とおく。このとき次が可換。
\[
\xymatrix{
P(U) \ar[d]_{\vp_*} \ar[r]_{i_*} & P(\P^n) \ar[r]^{\wt{I}} & P(S^h)  \ar[d]_{\beta^*} \\
P(\A^n) \ar[rr]^{I'}  & {} & P(A)
}
\]
すなわち$W \subset U$に対して
\[
\beta^{-1}({\wt I}(W)) = I'(\vp(W))
\]
が成り立つ。
}

\begin{proof}
  随伴性により零点変換公式$\beta$に帰着することができる。$W \subset U$、$W' \subset A$に対して、
  \begin{align*}
    W' \subset \beta^* {\wt I} i_*(W)
    &\iff \beta_*(W') \subset  {\wt I} i_*(W) \\
    &\iff i_*(W) \subset {\wt Z} \beta_*(W') \\
    &\iff W \subset i^*{\wt Z} \beta_*(W') \\
    &\iff W \subset \vp^* Z' (W') \\
    &\iff \vp_*(W) \subset Z'(W') \\
    &\iff W' \subset I'\vp_*(W)
  \end{align*}
  したがって示したいことがいえた。
\end{proof}


\begin{proof}
演習2.6の証明を続ける。$A_i = k[x_0, \cdots , \widehat{x_i} , \cdots , x_n]$とする。$I' \colon P(\A^n) \to P(A_i)$として$A(Y_i) = A_i/I'(Y_i)$と定義する。

以下、$Y \cap U_i \neq \emptyset $なる$i$についてだけ考える。$i=0$として一般性を失わない。そうして$A_0 = A$と書く。このとき$x_0 \notin I(Y)$なので$x_0 \neq 0 \; \text{in} \; S(Y)$であることに注意する。したがって局所化$S(Y)_{x_0}$を考えることができる。このとき
\[
S(Y)_{x_0} = S(Y) \otimes_S S_{x_0} = S_{x_0} / I(Y)S_{x_0}
\]
である。自然な写像$S_{x_0} \to S(Y)_{x_0}$を$\sigma$とおく。

$S_{x_0}$の次数$0$の元からなる部分環を$R$とおく。環準同型$\psi \colon A \to R$を
\[
\psi(f)(x_0, \cdots , x_n) = f\left( \f{x_1}{x_0}, \cdots , \f{x_n}{x_0} \right) = x_0^{- \deg f}\beta f
\]
により定める。$\psi$はあきらかに全射である。

イデアル変換公式$\beta$を$Y \cap U_0$に適用することにより$I'(Y_0) = \beta^{-1}(\wt{I}(Y \cap U_0))$が成り立つことが判る。したがって$f \in A$に対して
\begin{align*}
  f \in \Ker (\sigma \psi) &\iff \psi(f) \in I(Y)S_{x_0} \\
  &\iff x_0^{- \deg f} \beta f \in I(Y)S_{x_0} \\
  &\iff  \exists d \geq 0 \quad x_0^{d} \beta f \in I(Y) \cap S^{h} = {\wt I}(Y) \\
  &\iff \beta f \in \wt{I}(Y \cap U_0)  &(\text{$U_0$の外では$x_0$は$0$である}) \\
  &\iff f \in I'(Y_0)
\end{align*}
が成り立つことが判る。よって$A(Y_0) \cong \sigma(R)$である。よって$\psi$を$\psi(x_0) = x_0$で延長して、
\[
A(Y_0)[x_0,x_0^{-1}] \cong \sigma(R)[x_0,x_0^{-1}] \cong S(Y)_{x_0}
\]
である。

よって、整域の商体をとる操作を$\Frac$で表すことにすると
\begin{align*}
  \dim S(Y) &= \trdeg_k \Frac(S(Y)) &(\text{定理1.8Aより})\\
  &= \trdeg_k \Frac(S(Y)_{x_0}) \\
  &= \trdeg_k \Frac(A(Y_0)[x_0,x_0^{-1}]) \\
  &= \trdeg_k \Frac(A(Y_0)) + 1 \\
  &= \dim A(Y_0) + 1 \\
  &= \dim Y_0 + 1 &(\text{命題1.7より})
\end{align*}
が結論される。

ここまで$i = 0$としてきたが、同様のことが$Y \cap U_i \neq \emptyset$なる任意の$i$について成り立つ。よってそのような$i$について
\[
\dim Y_i = \dim S(Y) - 1
\]
である。ゆえに
\begin{align*}
  \dim Y &= \sup \dim (Y \cap U_i) &(\text{演習1.10より}) \\
  &= \sup_i \dim Y_i &(\text{$\vp_i$は同相写像}) \\
  &= \dim S(Y) - 1
\end{align*}
が結論される。
\end{proof}



\bfsubsection{演習問題 2.7}
\begin{rem}
  次の補題を用意しておく。
\end{rem}
\lem{
  位相空間$X$があり、$Y \subset X$、$U \opsub X$であったとする。このとき
  \[
  \ol{Y} \cap U \subset \ol{Y \cap U}
  \]
  が成り立つ。したがってとくに$\ol{Y} \cap U = \ol{Y \cap U} \cap U$である。この等式は、$X$での閉包をとってから$U$に制限することと、$U$へ制限してから$U$で閉包をとることが同じだと主張していることに注意。
}
\begin{proof}
『したがってとくに』以降はあきらか。前半を示す。

$x \in \ol{Y} \cap U$が与えられたとする。すると$x \in \ol{Y}$なので、ある有向点列$\{x_{\gra}\} \subset Y$が存在して$x_{\gra} \to x$である。$x \in U$より、$U$は$x$の開近傍であるので、ある$N$が存在して$\gra \geq N$ならば$x_{\gra} \in U$が成り立つ。よって$\{ x_{\gra} \}_{ \gra \geq N } \subset Y \cap U$は$x$
に収束する有向点列なので$x \in \ol{Y \cap U}$が結論される。
\end{proof}
\begin{proof}
  演習2.7の証明に戻る。
  \begin{description}
    \item[(a)] $\P^n$は射影多様体なので
    \[
    \dim \P^n = \dim S(\P^n) -1 = \dim S - 1 = n
    \]
    \item[(b)] 演習2.6の記法を用いて、適当な$i$を選べば
    \begin{align*}
      \dim \ol{Y} &= \dim \vp_i(\ol{Y} \cap U_i) &(\text{演習2.6より}) \\
      &= \dim \vp_i (\ol{Y \cap U} \cap U) &(\text{開集合への制限と閉包の可換性より}) \\
      &= \dim \ol{\vp_i (Y \cap U_i)} \\
      &= \dim \vp_i(Y \cap U_i) &(\text{命題1.10より}) \\
      &= \dim (Y \cap U_i) \\
      &\leq \dim Y
    \end{align*}
    逆は明らかなので$\dim \ol{Y} = \dim Y$である。
  \end{description}
\end{proof}

\bfsubsection{演習問題 2.9}
\barquo{
この例を用いて、$f_1, \cdots , f_r$が$I(Y)$を生成しても$\beta(f_1), \cdots , \beta(f_r)$が$I(\ol{Y})$を生成するとは限らないことを示せ。
}
\begin{proof}
  一般に、$A = k[x_1, \cdots , x_n]$, $B = A[w]$として、$f_1, \cdots , f_r \in A$とするとき
  \[
  \sum_{i=1}^r (\beta f_i) B \subset \beta \left( \sum_{i=1}^r f_i A \right) \cdot B
  \]
  は成り立つが、逆の包含は言えない。なぜかというと、$\gra \colon B \to A$は全射環準同型であるが、$\beta \colon A \to B$は環準同型ではないからだ。$\beta$は単射であり積を保つが、和を保たないのである。より正確にいうと$F = \sum_{i=1}^r f_i$のとき
  \[
  \beta \left(\sum_{i=1}^r f_i \right) = \sum_{i=1}^r w^{\deg F - \deg f_i} \beta f_i
  \]
  である。したがって、$g_1, \cdots , g_r \in A$があるとき$H = \sum_{i=1}^r f_i g_i$とすると
  \begin{align*}
\beta \left( \sum_{i=1}^r f_i g_i \right) = \sum_{i=1}^r w^{\deg H - \deg (f_i g_i)} \beta(f_i)\beta( g_i)
  \end{align*}
  である。このとき$\deg H - \deg (f_i)$は必ずしも$0$以上ではないので、
  \[
 w^{\deg H - \deg (f_i g_i)} \beta( g_i) \in B
  \]
  とは限らない。たとえば$r=2$, $f_1 = x^4y^3 + 1$, $f_2 = x^3y^4 + 1$, $g_1 = y$, $g_2 = -x$としてみればわかる。当たり前だが、$r=1$のときにはこのような問題は起こらない。
\end{proof}



\bfsubsection{演習問題 2.9}
\begin{description}
  \item[(a)] $X = \vp_0^{-1}(Y)$とする。$g \colon P(S^h) \to H(S)$を斉次元の集合に対してそれが生成する斉次イデアルを対応させる写像とし、$s \colon H(S) \to P(S^h)$を斉次イデアルに対してその斉次元の全体を対応させる写像とする。$\ol{X} = Z(I(X))$より、
  \begin{align*}
  I(\ol{X}) &= I(Z(I(X))) \\
  &= \sqrt{I(X)} &(Z(I(X)) = \ol{X} \neq \emptyset \text{による}) \\
  &= I(X) &(\text{斉次元をとって各自確認せよ})
  \end{align*}
  がわかる。だから$I(\ol{X})$でなく$I(X)$を考えればよい。

  $I(X) \supset g(\beta (I(Y)))$を示す: $g \in I(Y)$と$P \in X \subset U_0$が与えられたとする。このとき
  \begin{align*}
    \beta(g)(P) &= \left( x_0^e g\left( \f{x_1}{x_0}, \cdots , \f{x_n}{x_0} \right) \right) (P) \\
    &= p_0^e g(\vp_0(P)) \\
    &= 0
  \end{align*}
  であるから、示せた。

  $I(X) \subset g(\beta (I(Y)))$を示す: $f \in I(X) \cap S^h$と$Q \in Y$が与えられたとする。このとき
  \begin{align*}
    (\gra f)(Q) &= f(1, x_1, \cdots , x_n)(Q) \\
    &= f(\vp_0^{-1}(Q)) \\
    &= 0
  \end{align*}
  であるから$\gra f \in I(Y)$である。そうすると
  \begin{align*}
    f &= x_0^{\deg f} f\left( 1, \f{x_1}{x_0} , \cdots , \f{x_n}{x_0} \right) \\
    &= x_0^{\deg f - \deg \gra(f)} \beta(\gra (f))
  \end{align*}
  より$f \in g(\beta(I(Y)))$である。ゆえに$s(I(X)) \subset g(\beta(I(Y)))$であるが、これはすなわち$I(X) \subset g(\beta(I(Y)))$を意味する。
  \item[(b)] $A=k[x,y,z]$, $S=k[w,x,y,z]$としてこの上でイデアルを考えることにする。

  $I(Y)$の生成元: 演習問題1.2により、$I(Y)=(y-x^2,z-x^3)$である。

  $I(\ol{Y})$の生成元: $X = \vp_0^{-1}(Y) = \setmid{(1,t,t^2,t^3) \in \P^3}{t \in k}$とおく。このとき$I(\ol{Y}) = I(X)$である。実は$I(X)=(yw-x^2, zw-xy, y^2-xz)$が成り立つことを確認する。$(yw-x^2, zw-xy, y^2-xz) \subset I(X)$はあきらかであるので、逆を示せばよい。$J = (yw-x^2, zw-xy, y^2-xz)$とおき、割り算を実行する。まず$yw-x^2$で割ると
  \[
  S = (yw-x^2)S + xk[w,y,z] + k[w,y,z]
  \]
  を得る。次に$y^2-xz$で割ると
  \[
S = (yw-x^2, y^2-xz)S + xk[w,y] + k[w,y,z]
  \]
  を得る。さらに$zw-xy$で割ると
  \begin{align*}
  S &= J + zk[w,z] + xk[w] + k[w,y,z] \\
  &= J + xk[w] + k[w,y,z]
\end{align*}
  を得る。最後に$z^2w-y^3 = z(zw-xy) -y(y^2-xz) \in J$で割ると
  \[
  S = J + xk[w] + y^2k[w,z] + yk[w,z] + k[w,z]
  \]
  を得る。ゆえに$f \in I(X) \cap S^h$とすると
  \[
  f(w,x,y,z) + xf_0(w) + y^2f_1(w,z) + yf_2(w,z) + f_3(w,z) \; \in J
  \]
  なる$f_0, \cdots , f_3$がある。よって任意の$t \in k$と$s \in k \setminus \single$について
  \[
  tsf_0(s) + t^4sf_1(s,t^3s) + t^2sf_2(s,t^3s) + f_3(s,t^3s) = 0
  \]
  が成り立つことがいえる。この式で、$t$についての次数が$1$かそうでないかを見ることにより$tsf_0(s)=0$が言える。$k$は無限体なのでこれは$f_0$が多項式として$0$であることを意味する。また残った部分について$t$についての次数を$\bmod 3$で分けることができて$t^4sf_1(s,t^3s)=t^2sf_2(s,t^3s)=f_3(s,t^3s) = 0$がいえる。一つの変数を止めて$1$変数多項式だとみなすごとに零多項式なので、これは$f_1 = f_2 = f_3 =0$ということである。したがって$f \in J$がいえた。よって$I(X) \cap S^h \subset J$であり、ゆえに$I(X) \subset J$である。

  $I(Y)$の生成元を$\beta$で送ったものが$I(\ol{Y})$を生成するとは限らないこと: $(\beta(y-x^2), \beta(z-x^3)) = (yw-x^2, zw^2-x^3)$である。これは$\setmid{(0,0,s,u)}{s,t \in k \setminus \single}$を零点にもつが、$I(\ol{Y}) = J$はこの点を零点として持たない。よって$J$は$(yw-x^2, zw^2-x^3)$と一致しない。
\end{description}





\bfsubsection{演習問題 2.12}
\barquo{
点$P=(a_0, \ldots , a_n)$を$\rho_d(P) = ( M_0(a), \ldots , M_N(a))$ (単項式$M_j$に$a_i$を代入して得られる点)に送ることにより
}
\begin{rem}
\textblue{誤植であると思われる。}正しくは$\rho_d(P) = ( M_0(P), \ldots , M_N(P))$である。そのあとのコメントは意図不明。
\end{rem}



\bfsubsection{演習問題 2.12}
\begin{description}
  \item[(a)] $\grt$は斉次元の次数を$d$倍するので$\fraka$は斉次イデアル。また送る先が整域なので$\fraka$は素イデアル。
  \item[(b)] まず$\rho_d(\P^n) \subset Z(\fraka)$を示そう。こちら側は極めて易しい。$T = k[y_0, \cdots , y_N]$, $S=k[x_0, \cdots , x_n]$とおく。$y \in \rho_d(\P^n)$とする。そうすると$y=\rho_d(P)$なる$P \in \P^n$が存在する。このとき$f \in \fraka \cap T^h$に関して$f(y)=f(\rho_d(P))=(\grt f)(P)= 0$である。したがって
  $y \in \wt{Z}(\fraka \cap T^h) = Z(\fraka)$である。これで$\rho_d(\P^n) \subset Z(\fraka)$がいえた。

  逆を示そう。いま$m = (m_0, \cdots , m_N) \in Z(\fraka)$が与えられたとする。$0 \leq i \leq n$について
  \[
   M_{\grs(i)}=x_i^d
  \]
  と定めよう。このとき任意の$M_{\grd} = \prod_{k=0}^n x_k^{p(\grd, k)}$ $(0 \leq \delta \leq N)$に対して
  \begin{align*}
    (M_{\grd})^d &= \prod_{k=0}^n x_k^{p(\grd, k)d } \\
    &= \prod_{k=0}^n (M_{\grs(k)})^{p(\grd, k)} \\
    &= M_{\grd}(M_{\grs(0)}, \cdots , M_{\grs(n)} )
  \end{align*}
  が成り立つ。したがって、
  \[
  y_{\grd}^d - M_{\grd}(y_{\grs(0)}, \cdots , y_{\grs(n)}) \in \fraka \cap T^h
  \]
  であるから、任意の$0 \leq \grd \leq N$について
  \[
    m_{\grd}^d = M_{\grd}(m_{\grs(0)}, \cdots , m_{\grs(n)})
  \]
  が成り立つことがいえる。$m \in \P^N$より、ある$m_k$は$0$でないから、$0$でない$m_{\grs(i)}$が存在することが判る。ここで$0 \leq i \leq n$, $0 \leq j \leq n$について
  \[
M_{\grs(i,j)}=x_i^{d-1}x_j
  \]
  とおく。すると
  \begin{align*}
    (M_{\grs(i)})^{d-1}M_{\grd} &= x_i^{(d-1)d} \prod_{k=0}^n x_k^{p(\grd, k)} \\
    &= \prod_{k=0}^n x_i^{(d-1)p(\grd,k)} x_k^{p(\grd, k)} \\
    &= \prod_{k=0}^n (x_i^{d-1}x_k)^{p(\grd, k)} \\
    &= \prod_{k=0}^n (M_{\grs(i,k)})^{p(\grd, k)} \\
    &= M_{\grd}(M_{\grs(i,0)}, \cdots , M_{\grs(i,n)})
  \end{align*}
  が成り立つ。これは
  \[
  y_{\grs(i)}^{d-1}y_{\grd} - M_{\grd}(y_{\grs(i,0)} , \cdots , y_{\grs(i,n)}) \in \fraka \cap T^h
  \]
  を意味するので、これにより
  \[
  m_{\grs(i)}^{d-1}m_{\grd} = M_{\grd}(m_{\grs(i,0)} , \cdots , m_{\grs(i,n)})
  \]
  である。とくに$m_{\grs(i)} \neq 0$なる$i$をとって固定すると
  \[
\f{m_{\grd}}{m_{\grs(i)}} = M_{\grd}\left( \f{m_{\grs(i,0)}}{m_{\grs(i)}}, \cdots , \f{ m_{ \grs(i,n) }}{  m_{\grs(i)}  }   \right)
  \]
  である。ゆえに
  \begin{align*}
    m &= (m_0, \cdots , m_N) \\
    &= \left( \f{m_0}{m_{\grs(i)}} , \cdots , \f{m_N}{m_{\grs(i)}} \right) \\
    &= \rho_d \left( \f{m_{\grs(i,0)}}{m_{\grs(i)}}, \cdots , \f{ m_{ \grs(i,n) }}{ m_{\grs(i)} } \right) \\
    &= \rho_d ( m_{\grs(i,0) }, \cdots ,  m_{\grs(i,n)} )
  \end{align*}
  が成り立つ。すなわち、$m \in \rho_d(\P^n)$がいえた。
 \item[(c)] 全射であること: (b)によりあきらか。

 単射であること: $P,Q \in \P^n$について$\rho_d(P)=\rho_d(Q)$であるとする。このとき任意の$0 \leq i \leq n$について$p_i^d = q_i^d$である。そこで$p_i \neq 0$かつ$q_i \neq 0$となる$i$をとって固定することができる。すると任意の$0 \leq j \leq n$について$p_i^{d-1} p_j= q_i^{d-1} q_j$であるから、両辺を割って$p_j/p_i=q_j/q_i$
 を得る。よって$P=Q$であり、単射性が言えた。

 連続であること: $Y \clsub Z(\fraka)$が与えられたとする。すると$Y = Z(J)$なる斉次イデアル$J \subset T$がある。このとき$P \in \P^n$について
 \begin{align*}
   P \in \rho_d^{-1}(Z(J)) &\iff \rho_d(P) \in Z(J) \\
   &\iff \forall f \in J \cap T^h \quad f(\rho_d(P))=(\grt f)(P)= 0 \\
   &\iff P \in {\wt Z}(\grt (J \cap T^h)) \\
   &\iff P \in Z(\grt (J) S)
 \end{align*}
 が成り立つので、$\rho_d^{-1}(Z(J)) = Z(\grt (J))$である。よって$\rho_d$は連続。

 閉写像であること: $Z(I) \clsub \P^n$とする。$\rho_d(Z(I)) = Z(\grt^{-1}(I))$を示そう。$\rho_d(Z(I)) \subset  Z(\grt^{-1}(I))$は(b)の前半と同様なので、逆を示す。いま$Q \in Z(\grt^{-1}(I))$と$f \in I \cap S^h$が与えられたとする。(b)での結果により、$\rho_d$は全射なので$Q = \rho_d(P)$なる$P \in \P^n$がある。また、$f^d = \grt(g)$なる$g \in T$がある。すると
 $g \in \grt^{-1}(I) \cap T^h$であって、
 \begin{align*}
   f^d(P) &= (\grt g)(P) \\
   &= g(\rho_d(P)) \\
   &= g(Q) \\
   &= 0
 \end{align*}
 である。ゆえに$P \in {\wt Z}(I \cap S^h) =Z(I)$である。よって$Q \in \rho_d(Z(I))$がいえたので、$\rho_d$は閉写像、したがって同相であることが結論される。
 \item[(d)] $M_0 = x_0^3$, $M_1 = x_0^2x_1$, $M_2 = x_0x_1^2$, $M_0 = x_1^3$とする。このとき
 \[
 \rho_3(U_0 \cap \P^1) = \setmid{(1,t,t^2,t^3) \in \P^3}{t \in k} = X
 \]
 であるから
 \[
 \rho_3(\P^1) =  \rho_3(\ol{U_0 \cap \P^1}) = \ol{\rho_3(U_0 \cap \P^1)} = \ol{X}
 \]
 が判る。
 \end{description}



 \bfsubsection{演習問題 2.14}
 \barquo{
 $\psi$の像が$\P^N$の部分多様体であることを示せ。
 }
 \begin{rem}
   部分多様体(subvariety)という語は演習問題3.10で定義されている模様。
 \end{rem}




 \bfsubsection{演習問題 2.14}
 $\psi$がwell-definedであること: あきらか。

 $\psi$が単射であること: $\psi(a,b)=\psi(a',b')$とする。任意の$0 \leq i \leq r$, $0 \leq j \leq s$に対して$a_ib_j = a'_ib'_j$である。$b_j \neq 0$なる$j$を固定すると
 \[
 a_i = a'_i \cdot \f{b'_j}{b_j}
 \]
 である。$a_i \neq 0$なる$i$があるので、このとき$\f{b'_j}{b_j} \neq 0$である。よって$a = a'$がわかる。同様に$b = b'$である。

 $I(\Im \psi)$の構成: $\P^N$の座標環を$k[z_{ij}]$とする。準同形
 \[
 \grs \colon k[z_{ij}] \to k[x_0, \cdots , x_r, y_0, \cdots , y_s] \st \grs(z_{ij}) = x_i y_j
 \]
 を考え、$\fraka = \Ker \grs$とする。

 $Z(\fraka )=\Im \psi$であること: $Z(\fraka ) \supset \Im \psi$は、次の図式の可換性、つまり斉次元$f \in k[z_{ij}]$について$f \circ \psi = \grs f$が成り立つことによる。
 \[
 \xymatrix{
 \P^N \ar[r]^f & \{ 0, 1 \} \\
 \P^r \tm \P^s \ar[u]^{\psi} \ar[ur]_{\grs f} & {}
 }
 \]
 ここで$\grs f \in k[x_0, \cdots , x_r, y_0 , \cdots , y_s]$は、$x$と$y$のそれぞれについて斉次元であるため、$\P^r \tm \P^s$上の関数としてwell-definedであることに注意する。

 逆を示そう。$P = (P_{ij}) \in Z(\fraka)$が与えられたとする。このとき任意の$0 \leq i_0, i_1 \leq r$, $0 \leq j_0, j_1 \leq s$について
 \[
 z_{i_0 j_0} z_{i_1 j_1} -  z_{i_1 j_0} z_{i_0 j_1} \; \in \fraka
 \]
 が成り立つことから、
 \[
 P_{i_0 j_0} P_{i_1 j_1} =  P_{i_1 j_0} P_{i_0 j_1}
 \]
 である。さてここで$P_{lm} \neq 0$なる$l$, $m$が存在するので、ひとつ取って固定する。$a = (P_{0m}, \cdots , P_{rm})$, $b = (P_{l0}, \cdots , P_{ls})$とおく。このとき
 \[
 a_i b_j = P_{im} P_{lj} = P_{ij} P_{lm}
 \]
 が成り立つので、$\psi(a,b)=P$である。ゆえに$Z(\fraka ) \subset \Im \psi$がいえた。

 $\Im \psi \subset \P^N$が部分多様体であること: $\fraka$は整域への準同形の核だから素イデアル、したがって$\Im \psi$は既約集合であり、かつ$\P^N$の閉部分集合。ゆえに部分多様体。
